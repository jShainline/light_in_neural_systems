\subsection{\label{sec:history}Historical Context}
As described in the introduction, concepts of cognitive computing lead to the consideration superconducting optoelectronic hardware as a primary candidate for large-scale neural systems capable of general intelligence. Superconducting optoelectronic technology for neural computing resides at the confluence of multiple disciplines. The subject is derived in large part from the foundations of communication theory and computation, as established by Turing \cite{tu1936}, von Neumann \cite{ne1945}, Shannon \cite{sh1948} and others. Yet the nature of information processing in neural systems is not wholly captured by the mathematical analysis formalized to describe serial communication and computation with digital signals. Concepts from neuroscience dating back to Ramon y Cajal \cite{ra1908}, Mountcastle \cite{mo1978}, Edeleman \cite{ed1978}, and many others indicate that clean, serial data streams have much to gain from the network concepts of neural systems with complex spatial and temporal behavior, particularly if one seeks a machine that can think like an intelligent being. If one seeks complexity of performance, one must provide complexity in hardware. If one seeks comprehension in the face of ambiguity, the computation must be able to handle shades of gray. Probability and statistics are inherent to neural systems. 

Conceptually, neural systems contain threads of communication theory, digital logic, probability and statistics. In practice, these systems combine the physics of many devices. The systems envisioned here utilize electrons to compute and photons to communicate. They leverage semiconductors to make light and superconductors to compute. At this confluence of physics and computing, it is possible to conceive of systems with complexity from the chip scale to the globe, employing the logic of neural systems for cognitive information processing across broad reaches of space and time. 

Why do this now? Why this way? Since the inception of the EDVAC, its limitations were anticipated \cite{}. Yet the microelectronic march delayed the consequences by many decades. The scaling first charted by Moore \cite{mo1965} left little impetus for revolutionary concepts. Now that scaling is reaching physical limitations \cite{}, there is an appetite for what new hardware may have to offer. And as computing machines have become deeply integrated in society, we are ready for a sea change in what we ask our machines to do. They can answer any question of fact, but that is no longer all we ask of them. We seek intelligent machines, computers that think, not superficially, but with as deep a wisdom as we can manage to construct. That is why we seek to combine the strengths of superconductors and light. In conjunction, we think, they will lead to the most intelligent machines. To put this discussion in context, we must revisit the origins of computing.

\input{__history_of_digital_computing}

\subsubsection{\label{sec:neuroscience_history}}
\begin{itemize}
\item neuron doctrine, tissue comprised of cells in 1839, Ramon y Cajal, Golgi, and others in 1890s describe neurons as cells of nervous system
\item Mountcastle/Hubel and Wiesel identify columns and slabs of cortical cells as cortical building blocks
\end{itemize}

\subsubsection{Waves of Interest in Artificial Intelligence}


%Early paper on reservoir computing concept:
\vspace{3em}
%Harnessing Nonlinearity: Predicting Chaotic Systems and Saving Energy in Wireless Communication
\cite{jaha2004}

%also cite Maass paper here

\subsubsection{Present Context and Aim of this Article}
During WWII communication and information technology played many roles not present in prior conflicts. The technological and scientific climate was ideal for new advances in computing, particularly for general-purpose computers that could solve a variety of differential equations and thus contribute to many technical pursuits. Specifically, ... (list pursuits)

At the present moment, the technical areas driving evolution in machine learning and AI are myriad, far more diverse that during the 20th century. These driving influences include: consumerism, online advertising, and personal devices; financial security and financial trading; medical imaging, disease diagnosis, and drug development; data analysis and experimental control in scientific research; and information warfare. Also central to the present context are the unprecedented global transformation induced by silicon microelectronics, the associated availability of computational resources as well as large data sets, and the recent emergence of the Internet, which has led to the generation of large data sets and ensured that much of human activity involves engagement with an interconnected computational system.

This context is driving machine learning and AI to evolve in particular ways. Most importantly for the purposes of the present article, the aforementioned application spaces tend to require a computer that identifies patterns in a specific type of data. The system may seek patterns in a consumer's spend or internet browsing habits, or it may seek certain patterns in medical images or internet videos. The contemporary trend is thus to use machine learning to identify certain patterns in specific data types.

The objective of general intelligence requires more than category-specific pattern recognition or analysis of structured data. A system displaying general intelligence must be able to perceive and contextualize many types of data and learn to identify a large variety of patterns characterized by features in many dimensions of the data by observing and object's traits. For example, consider the associative intelligence involved in a mundane scenario encountered by a human. One may observe an object on the ground and based on its texture and shape know that the object is a leaf, and based on its location under a tree at its red and yellow coloring, it can be assumed the leaf has fallen from the tree. The intelligent system will recognize that this makes sense, as it is late autumn, and trees of this variety lose their leaves this time of year. The intelligent system may even conjure a mental image of a bud emerging in spring, demonstrating the recall of pertinent information to completely contextualize the elements under consideration. A model of the world exists within the intelligent system, and new information is continuously compared to this inner model. In turn, the inner model adapts to new information to maintain appropriate correspondence with the information to which it is exposed in a perpetually changing context.

Toward the goal of general intelligence, we must consider hardware that may depart from that which is adequate for domain-specific machine learning. Neuroscience is a valuable guide to the device and system requirements for hardware capable of general intelligence, while the history of computing and silicon electronic provide crucial insights related to the characteristics necessary in a successful, mature technology. This article reviews the neuroscientific principles that inform hardware properties. Several approaches to hardware for machine learning and neuromorphic computing are summarized, and attention is paid to selecting the physical mechanisms and devices that are best matched to the needs of neural computing for AGI. I describe the reasoning that leads me to the thesis that hardware utilizing few photons for communication and superconducting circuits for computation are uniquely suited to performing the functions required for intelligence. A potential architecture for such a system is outlined, extending from die-scale modules up to many-wafer cognitive systems with fiber-optic white matter connecting the system. Application spaces and potential physical limitations are discussed. This aim is to anticipate the hardware that will be utilized in mature, large-scale neural systems achieving general intelligence, and to explain the reasons why we should expect this hardware to differ from silicon microelectronics that have been unmatched in performance for digital computation. At this early stage, much remains uncertain about the feasibility of such a system, but the ramifications for technology and science are so immense that the subject deserves thorough inquiry.