\subsection{Criteria for assessing cognitive neural hardware}
\begin{itemize}

\item network metrics
\begin{itemize}
\item total number of neurons in network
\item degree distribution
\item clustering coefficient on different scales
\item average network path length
\item modularity analysis
\item rentian analysis
\end{itemize}

\item synaptic metrics
\begin{itemize}
\item range of achievable synaptic weights
\item number of achievable synaptic weights
\item max synaptic state retention time
\item analysis of short-term plasticity responses (filter properties, energy consumed, area)
\item analysis of stdp (range of values, update rates, energy required, time window control, area required)
\item analysis of metaplasticity (mechanisms, range of rates, energy required, area required)
\item total size of synapse
\end{itemize}

\item dendritic metrics
\begin{itemize}
\item analysis of dendritic intermediate nonlinear processing (range of time scales, I/O (gain) curves, energy, num synapses per dendrite, operations performed)
\item analysis of dendritic sequence detection (time scales, number of synapses involved, energy)
\item available logic operations (Boolean, time of last firing, multisynaptic)
\item number of values achievable in readout
\item size of dendrites
\end{itemize}

\item neuronal metrics
\begin{itemize}
\item total number of dendrites
\item total number of synapses
\item analysis of the rentian fan-in of the dendrito-synaptic arbor
\item dynamic properties of threshold
\item refractory period
\item integration time (if different from dendritic integration)
\item energy of neuronal firing
\item timing jitter of neuronal firing
\end{itemize}

\item communication metrics
\begin{itemize}
\item total system bandwidth (at neuronal level, also including dendritic level where neuronal data rates are multiplied by information available to first layer of dendrites after any additional synaptic fanout)
\item I/O 
\end{itemize}

\item system operation metrics
\begin{itemize}
\item operating temperature
\item power consumption
\item max spike events per second
\item power consumed during max spike events per second
\item power density
\item temperature variation during operation
\item total system size
\end{itemize}

\item system production metrics
\begin{itemize}
\item equipment required (i.e., technology node)
\item device yield/tolerance to variation
\item time required
\item packaging strategy
\item total material consumed
\item total cost to produce 
\end{itemize}

\end{itemize}

\begin{itemize}
\item unprecedented integration of photonics and electronics in a scalable process that can be implemented with existing infrastructure--change a few implant conditions, swap out a few sputtering targets, improve BEOL dielectrics for photonics
\item communication on various length scales, multi-planar on wafers, wafer-to-wafer vertical and lateral, fiber white matter
\item feasibility of brain scale
\item why si if no transistors?
\begin{itemize}
\item III-V substrates should be pursued as well. Our group is working on this, initial anecdata indicates similar efficiency
\item big problem is fab. wafers are harder to scale, material harder to purify, oxide not as good for waveguide cladding. Similar consideration to mosfet gate. Overall manufacturability
\item may eventually use transistors for perhaps faster refractory period
\end{itemize}
\item ultimate limits
\end{itemize}

\vspace{3em}
