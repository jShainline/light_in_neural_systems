\subsubsection{Waves of Interest in Artificial Intelligence}

\vspace{1em}
\begin{itemize}
\item McCulloch and Pitts
\item Shannon et al after WWII (ten week retreat should get it done)
\item Rosenblatt perceptron
\item Minsky
\item Hopfield
\item Backprop
\item connectionism
\item See also \_\_overview\_of\_machine\_learning.tex, make sure it isn't redundant
\end{itemize}

%Early paper on reservoir computing concept:
\vspace{3em}
%Harnessing Nonlinearity: Predicting Chaotic Systems and Saving Energy in Wireless Communication
\cite{jaha2004}

%also cite Maass paper here

\vspace{1em}
Flying analogy. Why should we make AI like the brain when we don't make airplanes like birds? If we wanted small flying machines that could eat worms and land on branches, we may need to follow biology more closely. We want airplanes to carry large loads long distances. Different objectives require different infrastructure. With AI, sometimes we want systems that do very different things than the human brain. We may not need brain-like devices or architecture. But when we pursue cognitive systems capable of general intelligence, we are seeking an entity that accomplishes much of what the brain does. When we aim to produce true intelligence, we may find the brain has already deciphered mathematically optimal means of accomplishing the goal, at least under the constraints of biology and evolution.