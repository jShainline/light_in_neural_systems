\section{\label{sec:applications}Application spaces}

\begin{itemize}
\item original applications of computing
\begin{itemize}
\item cryptography
\item weather
\item bombs
\item numerical solution to arbitrary diff eqs
\item from Turing, AI
\item now, apply to nearly all aspects of modern life
\end{itemize}
\item what others in the field are pursuing
\begin{itemize}
\item LASSO \cite{dasr2018}
\item fast control \cite{prsh2017}
\end{itemize}
\item here, following neuroscience applications 
\begin{itemize}
\item vision systems
\item language processing
\item motor control
\item may lead to Turing's vision of an AGI one can interact with
\end{itemize}
\item others, unique to large-scale neural systems and/or superconducting optoelectonic
\begin{itemize}
\item internet monitoring/simulation
\item sociological simulation
\item genetic analysis/evo devo
\item neuroscience and dynamical systems
\item quantum/neural hybrid systems (Bayesian discussion)
\end{itemize}
\end{itemize}

\begin{itemize}
\item telescopes
\item particle colliders
\item fusion reactors
\item eventually, perhaps even general commercial application
\item perhaps in the home
\end{itemize}

\subsection{Hybrid Cognitive Systems}

Modules for specialized processing: deep learning, quantum, high-speed inference, sensory organs

\subsubsection{Neural-deep learning hybrid systems}
Deep learning modules may provide types of sensory inputs, visual system design can borrow from CNNs and bio, olfactory with combs, etc. (cite personal communication)

\subsubsection{Classical-quantum-neural hybrid systems}
Quantum systems and neural systems have complimentary information processing capabilities. Quantum systems are fundamentally probabilistic, while neural systems are excellent for sampling probability distributions. Schemes to utilize quantum information are usually statistical, while populations of neurons can perform optimal Bayesian inference on samples drawn from statistical distributions. This reasoning leads us to consider the potential to utilize a neural system to perform quantum state tomography on large-scale quantum systems. The goals of the project are to construct a neural system capable of: 1) measuring the state of a network of qubits at the Heisenberg limit; 2) inverting the physical measurement through Bayesian inference to arrive at a quantum state reconstruction; and 3) reporting the reconstructed state over a classical communication channel as the qubits evolve in time, all implemented in scalable hardware.

Quantum information processing requires the ability to determine an unknown quantum state from a series of measurements performed on an ensemble of identically prepared systems. Performing measurements on many interacting qubits places severe demands on measurement hardware. To characterize a large quantum system, the number of measurements that must be performed can become intractably large if care is not taken to optimize the measurement protocol [1,2]. Additionally, the computational challenge of reconstructing the full quantum state from the set of measurements is formidable for large quantum systems. Developing hardware with classical, quantum, and neural capabilities presents an alternate route to develop scalable measurement techniques to extract Heisenberg-limited information [3] from a complex quantum system, to devise a method for a full quantum state to be efficiently reconstructed from measured data, and to ensure that the hardware implementation of this measurement/analysis procedure communicates efficiently to room temperature. 

At present, the various elements of scalable quantum state tomography are maturing and beginning to combine. In hardware, control and measurement circuits operating at cryogenic temperature are being developed. Josephson circuits capable of detecting single microwave photons present an exciting new avenue for scalable qubit characterization [4], yet racks of control and readout electronics are still employed for interfacing to relatively small quantum systems. Regarding reconstruction of quantum states from measured data, statistical methods involving Bayesian inference have been developed in the context of quantum tomography over the last 30 years [5-9]. It has been shown that by reoptimizing the measurements to be performed as information about the quantum state is acquired, the total number of measurements can be reduced [10]. Modern techniques in machine learning have been applied to the problem, showing that a neural network can perform quantum state tomography on highly entangled states of a hundred qubits [11]. The neural network employed in the tomographic analysis of Ref. 11 is a conventional, feed-forward neural network implemented in software. The related field of spiking neural networks has found that populations of spiking neurons naturally perform Bayesian inference [12-14]. Networks of spiking neurons can be trained so the average firing rate of the population represents the expectation value of an observable, and the variance of the firing rates of the neurons represents the uncertainty. Bayesian analysis has been applied to a series of weak measurements to track the trajectory of a single qubit, but cumbersome measurement hardware infrastructure was utilized [15]. Software neural nets have been used to perform tomography on 100-qubit systems, but the measurements were performed conventionally [11]. The adaptive Bayesian formalism has been applied to a two-qubit system to minimize the number of measurements required for full state reconstruction, but explicit numerical calculations were performed on conventional computers between each measurement [2]. Spiking neural networks have been used to perform Bayesian inference on statistical distributions [16], but the systems under observation have all been classical [14]. It has been shown that neural networks can emulate quantum computation [17-19], can accurately measure quantum systems [20], and can perform quantum state tomography [11]. The proposed hybrid hardware would combine these advances in a measurement apparatus performing Heisenberg-limited measurements, conducting Bayesian inference in real time as information is received, and using all knowledge about the quantum system for optimization of measurement protocol and full state reconstruction. The metrological hardware we propose to develop would prepare highly entangled states of many qubits, perform the measurements and information processing necessary for tomography, and communicate the results of state reconstruction to room temperature with near-infrared photons over optical fiber [21].

This concept remains in the domain of thought experiments, but we will describe a route to make it real. We propose to model and construct the classical-quantum-neural (CQN) hybrid system depicted schematically in Fig. 1. The system comprises a classical control module, a quantum module containing a network of coupled qubits, and a neural module interfacing with both the classical and quantum systems. The envisioned operation of the CQN system is as follows. The classical system prepares the quantum system in a particular state. The quantum state is set by a static many-body Hamiltonian and a series of qubit drive pulses. The classical system also provides the neural system with information representing the static and drive Hamiltonians. The classical system generates microwave signals to probe the quantum system. We envision the measurement signals perform a series of weak measurements on a time scale short relative to the qubit decoherence and relaxation times [15], but projective measurements could be employed as well. The ability of weak measurements to give information as a function of time while a quantum state evolves fits nicely with the dynamical operation of spiking neurons. The output from these measurements is a faint microwave signal, and information about the state of the qubits is encoded in this signal. The neural system comprises an input layer, a computational reservoir, and an output layer. The input layer receives the faint photonic signals, and the dynamical state of the reservoir evolves in response to the varying signals received from the quantum system. To function as proposed, the input layer of the neural network should represent expectation values of the qubits in the quantum system, and the operation of the network should be to invert that information into a hypothesis regarding the density matrix [6,9], encoded in spike trains by the output neurons. Mathematically, we usually assume we know the density matrix and can therefore calculate the expectation value of any operator. In practice, one measures expectation values and infers the density matrix from the data. This is the inversion operation that will be performed by the neural system.

Superconducting circuits appear capable of realizing this CQN system. We propose to develop the quantum system based on transmon qubits operated in the dispersive regime, probed via microwave signals along transmission lines. Josephson arbitrary waveform synthesis will be utilized to generate the microwave qubit control and measurement signals, and a superconducting FPGA based on flux-quantum logic and magnetic Josephson junction memory elements will control the operation of the entire apparatus. The neural system will receive the microwave signals transmitted from the classical system through the quantum system, and therefore the input synapses to the neural system must respond to faint microwave signals to perform Heisenberg-limited observation of the quantum system. As the size of the quantum system grows, so must the neural system. To achieve the required communication across the large neural system, photonic connectivity is required, making superconducting optoelectronic loop neurons [21] promising as device primitives for the neural system. The optical signals from these neurons bring the added advantage that information is transduced to optical, and can be readily coupled to fiber for transmission out of the cryostat for further processing with CMOS circuits.
 
References: 
[1] C. Granade, J. Combes, and D.G. Cory, “Practical Bayesian tomography”, New J. Phys. 18 (2016).
[2] G.I. Struchalin, I.A. Pogorelov, S.S. Straupe, K.S. Kravtsov, I.V. Radchenko, and S.P. Kulik, “Experimental adaptive quantum tomography of two-qubit states”, Phys. Rev. A 93 (2016). 
[3] A.A. Clerk, M.H. Devoret, S.M. Girvin, F. Marquardt, and R.J. Schoelkopf, “Introduction to quantum noise, measurement, and amplification”, arXiv:0810:4729 (2010).
[4] A. Opremcak, I.V. Pechenezhskiy, C. Howington, B.G. Christensen, M.A. Beck, and R. McDermott, “Measurement of a superconducting qubit with a microwave photon counter”, Science 361 1239 (2018).
[5] K.R.W. Jones, “Principles of quantum inference”, Annals of Physics 207 (1991).
[6] K.R.W. Jones, “Fundamental limits upon the measurement of state vectors”, Phys. Rev. A 50 (1994).
[7] R. Derka, V. Buzek, G. Adam, and P.L. Knight, “From quantum Bayesian inference to quantum tomography”, arXiv:quant-ph/9701029 (1997). 
[8] R. Schack, T.A. Brun, and C.M. Caves, “Quantum Bayes Rule”, Phys. Rev. A 64 (2001).
[9] R. Blume-Kohout, “Optimal, reliable estimation of quantum states”, New. J. Phys. 12 (2010).
[10] F. Huszar and N.M.T. Houlsby, “Adaptive Bayesian quantum tomography”, Phys. Rev. A 85 (2012).
[11] G. Torlai, G. Mazzola, J. Carrasquilla, M. Troyer, R. Melko, and G. Carleo, “Neural-network quantum state tomography”, Nat. Phys. 14 (2018).
[12] W.J. Ma, J.M. Beck, P.E. Latham, and A. Pouget, “Bayesian inference with probabilistic population codes”, Nature Neuroscience 11 (2006).
[13] T. Yang and M.N. Shadlen, “Probabilistic reasoning by neurons”, Nature 447 (2007).
[14] J.M. Beck, W.J. Ma, R. Kiani, T. Hanks, A.K. Churchland, J. Roitman, M.N. Shadlen, P.E. Latham, and A. Pouget, “Probabilistic population codes for Bayesian decision making”, Neuron 60 (2008).
[15] K.W. Murch, S.J. Weber, C. Macklin, and I. Siddiqi, “Observing single quantum trajectories of a superconducting quantum bit”, Nature 502 (2013).
[16] M.J. Barber, J.W. Clark, and C.H. Anderson, “Neural representation of probabilistic information”, Neural Computation 15 (2006).
[17] C. Wetterich, “Quantum computing with classical bits”, arXiv:1816.05960 (2018).
[18] C. Pehle, K. Meier, M. Oberthaler, and C. Wetterich, “Emulating quantum computation with artificial neural networks”, arXiv:1810.10335 (2018). 
[19] G. Carleo and M. Troyer, “Solving the quantum many-body problem with artificial neural networks”, Science 355 (2017). 
[20] D.T. Lennon, H. Moon, L.C. Camenzind, L. Yu, D.M. Zumbuhl, G.A.D. Briggs, M.A. Osborne, E.A. Laird, and N. Ares, “Efficiently measuring a quantum device using machine learning”, arXiv:1810.10042 (2018). 
[21] J.M. Shainline, S.M. Buckley, A.N. McCaughan, J. Chiles, A. Jafari-Salim, R.P. Mirin, and S.W. Nam “Circuit designs for superconducting optoelectronic loop neurons”, J. Appl. Phys. 124 (2018).

\subsubsection{SOENs as the central cognitive hub}
This hardware can interface readily with semiconductor electrical or optoelectronic logic or neural systems operating at 300\,K or 4\,K; photonic deep neural networks as well as photonic qubits; and superconducting deep neural deep neural networks as well as superconducting qubits. It is my perspective that the superconducting optoelectronic spiking neural network will serve to integrate information from all these input modalities into a coherent cognitive dynamical state. 

\vspace{3em}
discuss different hardware at different temperature stages

\vspace{3em}
Imaging arrays of SPDs can be used to read out information stored in volume holograms. cite Varun's SPD arrays. ``Millions of bits of information could be read out and transferred at the same time merely by shining an unfocused light beam on a suitably designed optical memory device.'' \cite{abps1987}

\begin{equation}
\label{eq:density_matrix}
\langle \mathcal{O}\rangle = \mathrm{Tr}(\mathcal{O}\rho)
\end{equation}