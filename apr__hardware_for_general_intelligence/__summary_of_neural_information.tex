\subsection{Summary of neural information}
Von Neumann suspected the existence of a more subtle and powerful language of information employed by the human brain. Neuroscience has elucidated many of the principles of this language. Let us attempt here to summarize the salient elements that should guide neural hardware design. Given the complexity of the subject and the rapidly evolving state of neuroscience, we expect time to bring corrections to these concepts, yet the foundations of these concepts do seem well established.

The model from neuroscience informing the hardware presented here is as follows. Each neuron attempts to gain access to as much information as is physically possible about the activities of the other neurons in the network. Each neuron gains access to pieces of this information based on the temporal filter it performs. For example, a given synapse (or pair) can pas on information about the rate, rising edges, falling edges, temporal correlations, or sequences output from any neuron (or pair) in the network. In the temporal domain, we assume the signals can each be given a distinct exponential decay constant. Each synapse then has the information to answer a question, such as, how much has neuron $i$ been firing in the last $\tau_{ij}$ seconds? Or, how much has neuron $i$ been bursting, and then quiescing, and then bursting again in the last $\tau_{ij}$ seconds? 

The answers to these questions must pass through the dendritic arbor. Each dendrite contains information received from one or a number of synapses coupled to the dendrite. The net information contained in the dendrite may be able to answer a question such as, how much have neurons $a$, $b$, and $c$ collectively been firing in the last $\tau$ seconds? Or, how many of a particular subset of 10 neurons in cluster $q$ have stopped firing in the last $\tau$ seconds? 

When under the influence of inhibitory neurons, a dendritic compartment will be quiet. Upon the release of inhibition, the dendritic compartment reports to the neuron the answer to the question it knows how to answer by transmitting an analog signal in the form of current that modifies the neuron's membrane potential. Each segment of the dendritic arbor performs a nonlinear transfer function on the signals from the synapses connected to that segment, and the neuron itself performs a nonlinear transfer function on the signals it receives from across its dendritic arbor. The neuron's nonlinear transfer function is to produce a pulse (an action potential) when the membrane potential of the soma reaches a certain threshold value. This pulse is communicated through the neuron's axonal arbor to all the neuron's downstream connections as a digital signal, wherein the presence of the pulse informs all downstream connections that under the present network conditions, the activities on all that neuron's dendrites were sufficient to induce firing, and the amplitude of the pulse is not used to encode information.

In this picture, the excitatory (pyramidal) neurons are a knowledge base that can be queried by the inhibitory interneuron network. The net objective of the network is to be able to identify as many correlations as is physically possible across space and time. In space, these correlations are limited by network path lengths. In time, correlations are detectable over time constants of synapses and dendrites. To identify correlations over longer times than this (such as the lifetime of the entity), the logic of synaptic plasticity and metaplasticity come into play \cite{fudr2005,fuab2007}. Note that this model of inhibitory query of pyramidal neurons is readily scalable across arbitrary partitions of the network, so the basic informational principles are continuous from the scale of local networks up to the system as a whole. This is uniquely enabled by the fractal use of space and time.

We conjecture that the probability of observing a neuronal avalanche accessing the information in $s$ dendritic compartments scales as $P(s)\sim s^{-\alpha}$. During oscillatory behavior, inhibitory neurons sample specific dendrites in an intentional, controlled manner at a frequency $f_{\theta}$, so that the information contained in the collection of all synapses, dendrites, and neurons active in the functional network resonant at $f_{\theta}$ can be integrated across partitions of the network to be incorporated in computations a higher levels of network hierarchy. The mechanisms for this coherent information access and integration include cross-frequency coupling, wherein local activity occurring at higher frequencies $f_{\gamma}$ is modulated by slower frequencies $f_{\theta}$, with the phase of the higher frequency activity being well-defined relative to the phase of slower frequencies. Such network-wide information integration through multi-scale activity across space and time is thought to be necessary for cognition \cite{bu2006}, perhaps by enabling access to the global neuronal workspace \cite{ba1988,de2014}.

To summarize the summary, a single neuron extracts as much information as it can from its neighbors, and it transmits as much as it can to its neighbors through its activity in various effective network contexts established by the state of the dendritic arbor as configured by the inhibitory interneurons. A cluster in the network attempts to answer as many questions as it can about its inputs, and it attempts to communicate this information across the network to as effectively as possible, and so on up the hierarchy of network partitions. A network of inhibitory neurons samples the information from synapses and dendrites in myriad combinations, in principle answering any question that could be reasonably posed regarding a stimulus that could be physically presented to the entity. 

This model of neural information processing bears a resemblance to a Turing machine. Turing's goal was to make a machine that could answer any question that could be asked within the axioms of its system (universal while not violating G\"{o}del), and the goal here is essentially the same. Yet in addition to the Turing machine behavior, wherein the network acts as an oracle, an intelligent neural system should also be able to ask its own questions by formulating an output that generates a response from an intelligent or inanimate agent so as to gain new information. In addition to generating an entity that is universal in the sense defined by Turing, we aspire to create machines that are intelligent in the sense that they can engage in self-directed learning. Such a machine will be able to answer our questions, but also have a mind of its own.

\vspace{3em}
What is the significance of oscillations and synchronized clusters? Much has been written about this \cite{}, and perhaps much remains unknown. Here I offer one interpretation based on the references presented thus far. Within a minicolumn there are about 100 neurons, and these primarily participate in small neuronal avalanches, power-law distributed from size one to the complete 100, and also extending far beyond the minicolumn. These avalanches can be the result of spontaneous, stochastic activity, or they can be induced by outside stimulus. These transient, synchronized bursts of varying numbers of neurons primarily between locally connected clusters of neurons make up gamma oscillations. Single minicolumns and clusters of minicolumns up to the scale of columns can give rise to a very large number of these dynamical states, each with a distinct set of participating neurons and synaptic connections. In the language of dynamical systems, each pattern of activity is a stable heteroclinic channel, which is a sequence of successive metastable states. The time-scales and resonant frequencies of activity are tuned by synaptic, dendritic, and somatic time constants adapt in response to local applied voltage (due to afferent synaptic activity) as well as neuromodulators that may adjust resonant frequencies of sectors of neurons simultaneously.

Groups of neurons spanning minicolumns to columns for a dynamical basis set \cite{rahu2008,one_from_reading_group} that can represent some range of stimuli. The fastest transient synchronized states occur within minicolumns, and a single minicolumn can produce a large number of different patterns dependent on the stimuli. Each minicolumn is capable of representing a certain class of features or input stimulus, i.e., a certain color, spatial frequency, intensity, etc. Synchronized ensembles at slightly lower frequencies form between groups of minicolumns, forming specific coalitions of neurons representing a distinct state across multiple dimensions of feature space, providing a broader range of insights regarding the nature of a stimulus. 

The pattern repeats at still lower frequencies, where dynamical states of columns communicate amongst each other to form brain regions. These complex structures of neural architecture are supremely equipped to analyze certain aspects of the world, such as faces, language, odors, etc. At the scale of brain regions, oscillations are in the theta frequencies. Activity on the scale of columns organizes on gamma timescales (30\,Hz-80\,Hz), and activity across the cortices, integrating input from activity in many columns, occurs ten times slower, in theta oscillations (4\,Hz-8\,Hz). Neurons in any minicolumn can talk to neurons in any region on theta timescales, provided network path lengths are short. 

Local clusters interacting on gamma timescales with activity modulated by broad network activity on theta timescales illustrates the meaning of the phrase, ``information integration across space and time.'' However, it is to be understood that many spatial and temporal scales are involved in cognition, and this hierarchical nesting of scales achieves the fractal organization of the brain that appears uniquely capable the efficient information integration that enables cognition. I take the principles of neural information summarized here as the principles guiding the design of cognitive systems. Next I describe several approaches to hardware for artificial intelligence.

\vspace{3em}
From an architectural perspective, there is one more level of heirarchy to consider, and that is the thalamo-cortical complex \cite{}. Brain regions input a representation of their activity to the thalamus, and it decides what informaiton to braodcast across the network. The thalamus must be able to receive from and reach all brain regions in a tremendous feat of network communication.

\vspace{3em}
themes common to all three pictures:
\begin{enumerate}
\item space and time intertwined
\item enhanced responsivity to inputs and resilience to noise
\item role of inhibition critical
\item role of/relation to plasticity
\end{enumerate}

\vspace{3em}
Regarding computational devices, each synapse performs filtering operations, each dendrite performs nonlinear transfer functions, and each neuron is a complex processor adapting on multiple time scales and performing network computations withe different transient ensembles engaged to process different information. We should expect hardware capable of general intelligence to incorporate devices and circuits capable of performing these functions efficiently and directly without being emulated by a digital system stepping through differential equations. 

Regarding the communication infrastructure, we seek to enable networks with many nodes and short average path length made possible by neurons making many connections. The network must have a high small-world index, requiring long-distance connections, and the network must have a modular configuration with specialized processors combining their information at progressively higher levels of hierarchy over larger spatial and temporal scales to form representations of stimulus with fine detail as well as comprehensive context. The communication network must achieve this graph structure directly and not via emulation on a wholly different graph of nodes for communication.

We next consider multiple routes to creating hardware achieving these goals of communication and computation in neural systems for general intelligence.