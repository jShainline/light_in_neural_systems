\subsection{\label{sec:superconducting_optoelectronic}Superconducting Optoelectronic Systems}
Superconducting optoelectronic networks were proposed by our group at NIST in 2017 based on two conjectures: 1) communication between neurons in hardware for AI is best achieved with photons; and 2) communication at the single-photon level reaches the energy efficiency limit, and therefore enables scaling to very large cognitive systems. These conjectures led us to propose specific circuits based on semiconductor light sources operating in conjunction with superconducting-nanowire single-photon detectors \cite{shbu2017}. After additional consideration, it became apparent that the use of Josephson junctions in conjunction with nanowire detectors enabled significantly more synaptic, dendritic, and neuronal computational functionality than nanowire circuits alone. The basic conjectures cited above remained in place, and the communication infrastructure described in Ref.\,\onlinecite{shbu2017} remained central to the concept, but a third conjecture began to guide our work: 3) In artificial neural systems utilizing superconducting single-photon detectors and communication with few-photon signals, circuits combining Josephson junctions with flux-storage loops are best equipped to provide the necessary short- and long-term synaptic plasticity functions, elaborate dendritic processing, and neuronal thresholding functionality. A summary of the circuits accomplishing these functions was first presented in Ref.\,\onlinecite{sh2018}, and more detail was given in Ref.\,\onlinecite{sh2018_full}. Dendritic processing was detailed in Ref.\,\onlinecite{sh2019_fluxonic}. This section summarizes our present thinking on the general concepts leading to the hypothesis that hardware combining superconducting electronics with silicon photonics and fiber optics is best equipped to realize artificial general intelligence, and a brief overview of the photonic and electronic circuits is given.

\subsubsection{Basic Motivations}
--``Hybrid optical neural networks may not materialize as commercial products unless each neuron in the network requires more than 1000 interconnections to solve problems.'' \cite{juya1996}

\begin{itemize}
\item from the perspective of single photons for communication leading to spds, JJs, si light sources
\item from the perspective of JJs for neuromorphic circuits, leading to single photon communication, spds, si light sources
\end{itemize}

\subsubsection{computational circuits}
\begin{itemize}
\item synapses
\item dendrites
\item soma
\end{itemize}

\subsubsection{Communication Circuits}
\begin{itemize}
\item axon hillock/transmitter circuit
\item photonic circuits/axonal arbor
\item wafer-scale architecture
\end{itemize}

\subsubsection{Large-Scale Systems}
\begin{itemize}
\item beyond wafer scale/wafers to columns
\item a hierarchy of modules
\item rentian scaling at wafer scale all the way up to many module systems enabled by multiplanar waveguides as well as fiber white matter, scaling of white to gray matter volume, estimate for largest size
\item neuronal pool, velocity of communication
\item power consumption: any competing technology must communicate at the velocity of light with very few photons across systems of comparable scale
\item native environment in space
\end{itemize}

\subsubsection{Misc. SOEN Notes}

\vspace{3em}
-\cite{psbr1990}GaAs neurons: two transistors, a photodetector, and an LED

\vspace{3em}

From Verber's perspective in 1984, ``It is to be expected that optical and integrated-optical methods will become available to perform a significant number of computational tasks. Analog methods are currently being developed and digital optical devices will follow in the near future. The challenge is to recognize their strengths and weaknesses in comparison to electronic devices and incorporate them into computational systems in an optimal fashion.'' \cite{ve1984} However, optical devices have not become widely adopted for any computational tasks, digital or analog. This state of technology is not due to a lack of effort or ingenuity, but rather to a combination of physical and practical reasons \cite{ke1985}. However, optical and integrated-optical methods for communication are still quite promising at the chip \cite{suwa2015} and package \cite{} scales, and are indisputably superior to electrical communication at large scale. 


\vspace{3em}
%Waveguide integrated superconducting single-photon detectors with high internal quantum efficiency at telecom wavelengths
\cite{kafe2015} 

These detectors are wires of superconducting material \cite{mave2013}, and they can be straightforwardly patterned atop on-chip dielectric waveguides \cite{shbu2017b,x,y,z}. The nanowire thickness is 4\,nm-10\,nm, its width is 80\,nm-350\,nm, and the interaction length along the waveguide can be as short as 10\,\textmu m. The wire is current biased in parallel with a resistive load (Fig.\,\ref{fig:snspd}), .

Because the detectors are superconducting, they draw very near zero power in the steady state. 

\vspace{3em}
As Miller points out in considering optical logic devices, ``...coherent interference is mostly a nuisance in logic systems. [A] device at the $\sim$\,100-photon level could still operate in a quasi-classical fashion where optical or quantum coherence is neither necessary nor even desirable.'' \cite{mi2010} By using few-photon optical signals for communication from neurons to single-photon synapses, this is precisely the regime in which we operate. By using light only for fan-out, optical coherence is irrelevant because independent fields from separate neurons never interfere. Miller identifies two major benefits of optical interconnectivity: 1) that ``higher densities of information [are] possible in relatively long connections''; and 2) ``optics could reduce the energy required for communication''. \cite{mi2010} To his second point, I add optics can additionally reduce the complexity of communication hardware while increasing the speed. These considerations are of primary significance in the context of integrated neural systems. Further, while Miller envisions integration with semiconductor photodetectors and therfore concludes that optical signals ``only need to transmit enough energy to charge the photodetector at the receiving end'' (100 - 1000 photons), when superconducting detectors are employed, one need only transmit enough energy to disrupt the superconducting phase of a nanowire detector, and this can be achieved with a single photon. 

\vspace{3em}

-Like superconducting neural systems and free-space optical systems, the goal of nearly all efforts in optoelectronic neural systems and neuromorphic photonics is not to develop general intelligence, but rather to realize neural systems for specific tasks such as inference or control. For most efforts, the motivation for using light is the speed, either of laser cavity dynamics or optical communication. Device and hardware choices toward these ends may be different than for the focus of this article, which is general intelligence. We intend to explain why specific choices are not conducive to the present goal, even if they are suitable for other applications.


\vspace{3em}
Criteria for practical devices \cite{mi2010,ke1985a}
\begin{itemize}
\item Cascadability
\item Fan-out
\item Logic-level restoration
\item Input/output isolation
\item Absence of critical biasing
\item Logic level independent of loss
\end{itemize}

\vspace{3em}
some technologies place high speed as the primary goal, and sacrifice feasibility to attain that speed advantage. other technologies emphasize small size as a significant objective, and practical complications usually result. There are many applications where speed or size is paramount, but if they are not it is often easier to achieve general high performance is compromises can be negotiated between several metrics.

\vspace{3em}
re detectors: compare $CV^2/2$ with $C = 1$\,fF and $V = 1$\,V to $LI^2/2$ with $L = 250$\,nH and $I = 10$\,\textmu A \cite{mi2009}

\vspace{3em}
a primary difference between circuits of normal conductors and semiconductors as compared to superconductors is that in the former one generally deals with capacitances, whereas in the latter on deals with inductances. With capacitance, parasitics involve the energy  $CV^2/2$ and the charging time (integral I dt = Q = CV). In the case of inductors, the energy is $LI^2/2$ and the charging time 

\vspace{3em}
\begin{itemize}

\item general concept: communication between neurons is photonic; when a neuron spikes it must either generate or modulate light; throughout, speed, size, power all co-optimized

\item first key choice: generate or modulate

\item modulate:
\begin{itemize}
\item requires cw light running at all times ($x_{dB/cm} = 1; y_{dB/s} = 100*x_{dB/cm}*c; q_{dB} = 3; t_s = q_{dB}/y_{dB/s}$, for 1\,dB/cm propagation loss, 3\,dB of the light is lost every 100\,ps)
\item requires frequency tuning, most likely
\item cross talk of neurons on the same bus
\end{itemize}

\item generate:
\begin{itemize}
\item requires light source at every neuron
\item requires unprecedented optoelectronic integration, million sources and a billion detectors on a wafer
\item must be very low capacitance
\item seems like only a silicon light source will suffice, but this would require cryogenic operation
\end{itemize}

\item second key choice: establish synaptic weight in the photonic or electronic domain?

\item photonic domain:
\begin{itemize}
\item This choice has several important ramifications for hardware and information processing. Regarding information processing, it is usually assumed that neural communication is digital: the presence or absence of an action potential is a binary one or zero, and the amplitude of the action potential is not encoding information. When adjusting the synaptic weight in the photonic domain, this is not the case. The number of photons reaching a neuron through a synaptic connection becomes an analog variable, and it is subject to shot noise, in addition to any noise mechanisms present in the detector. The signal-to-noise ratio of shot noise improves with $\sqrt(N_{\mathrm{ph}})$, where in this case $N_{\mathrm{ph}}$ is the average number of photons, so establishing weights in the photonic domain introduces an energy/noise tradeoff. Setting weights in the photonic domain also has the disadvantage that photons are discarded by attenuation at weak synaptic weights. Thus, by setting synaptic weights in the photonic domain, we place a burden on light sources to produce large numbers of photons to minimize shot noise, and we discard photons when they are attenuated at weak synapses. In this mode of operation, light is used for communication, but it is also used for the important computational operation of applying the synaptic weight.
\item these objections notwithstanding, to our knowledge, all except one optoelectronic neural approach proposed to date sets weight in photonic domain
\item specific instances: mzi (no STDP, poor spatial scaling, cross-talk); wdm (limited number of channels, cross-talk with rings on master ring, demands on sources); mzi and wdm (thermal tuning hopeless for scaling, no plasticity mechanisms proposed); phase change synapses (at least don't dissipate steady state, still power lost due to variable attentuation, small footprint, Hebbian learning possible, but STDP not likely, meta, short term also doesn't look promising)
\end{itemize}

\item electronic domain:
\begin{itemize}
\item By contrast, if we establish synaptic weights in the electronic domain, light is used exclusively for communication, and communication remains entirely digital. The presence of an optical signal can be used to represent an all-or-none communication event. In this case, the detector and associated electronics must be able to achieve a variable synaptic response to identical photonic pulses based on the configuration of the electronic aspects of the circuits. In this case, we expect that a neuron will send, on average, $N_{\mathrm{ph}}$ photons to each of its downstream synaptic connections. Due to shot noise, each downstream connection will receive $N_{\mathrm{ph}}\pm\sqrt{N_{\mathrm{ph}}}$ photons, and the detector circuit must be configured to implement a synaptic response if a threshold of $N_{\mathrm{th}}$ photons is detected. After detection, the electronic response must vary depending on the synaptic weight, independently of the precise number of photons that was detected. It is in this electronic response that the signal becomes analog again. Whereas setting the synaptic weights in the photonic domain places a larger burden on light sources, setting the synaptic weights in the electronic domain places a larger burden on detector circuits. One must achieve a detector circuit that converts light pulses to electrical current or voltage, and the amount of electrical signal must be largely independent of the number of photons in the pulse, depending instead on reconfigurable electrical properties of the circuit, such as bias currents or voltages. These reconfigurable bias currents or voltages then represent the synaptic weights, and the task of a neuron's light source is simply to provide a roughly constant number of photons to each of its downstream synaptic connections. For energy efficiency, the number of photons necessary to evoke a synaptic response from the detector ($N_{\mathrm{th}}$) should be made as low as possible to make the job of the light source as easy as possible. $N_{\mathrm{th}}$ cannot be made lower than one, as the electromagnetic field is quantized into integer numbers of photons.
\item only know of one system where electronic domain has been proposed: soens
\item basic functionality
\item stdp
\item meta
\item homeo
\item short-term
\end{itemize}

\item neuronal computation: reaching threshold
\begin{itemize}
\item differentiate between state-based and spiking
\item main considerations here are energy/power
\item how much energy is required to generate a pulse or drive a modulator? 
\item how much light must be made/moved to drive all downstream synaptic connections? 
\item how fast can pulses be generated (refractory period)? 
\item how long can neurons remember (leak rate)? 
\item what is range of spike rates? what is expected power?
\end{itemize}

\item somewhere in here, comparison of detectors (going cold costs 500x for carnot, but gains 2000x for detector sensitivity)
\item related, comparison of sources (going cold reduces how many photons must be made, but most importantly, if it means a silicon light source can work for this project, it is a game changer)

\item inhibition, gotta have a plan

\item dendritic processing
\begin{itemize}
\item intermediate nonlinearities
\item direction attention with inhibition
\item sequence detection
\item how can any of this happen in the photonic domain?
\end{itemize}

\item room temp vs cryo
\begin{itemize}
\item sources (cryo enables Si sources. for large-scale integration, process simplicity brings tremendous advantage)
\item detectors (A SiGe photodetectors needs about $10^4$ photons in 100\,ps to respond; efficiency of SNSPDs, low-noise of SNSPDs, simplicity of fabrication, and excellent operation in conjunction with JJs)

\end{itemize}

\end{itemize}

 

\subsubsection{Superconducting optoelectronic neural systems may overcome challenges of digital systems}
Let us refer to these challenges with superconducting electronics as: 1) the memory problem; 2) the clock problem; 3) the I/O problem; and 4) the volt problem. We will argue below that a new type of cryotron, implemented in a compact, on-chip, thin-film device can overcome the volt problem, but only if one accepts slower speed than superconducting digital logic seeks, an acceptable trade-off in a neural system. Solving the volt problem this way enables us to generate light, thereby solving the ``O'' part of the I/O problem. The ``I'' part of the problem is solved by utilizing superconducting photon detectors, compact circuit elements developed since 2000 and integrated with photonic circuits within the last decade. With integrated light sources and detectors, superconducting systems can send and receive near-infrared photons over optical fibers, relieving the heat load of co-axial cables, and solving the volt problem at semiconductor photodetectors on the room-temperature side. The asynchronous nature of spiking neural systems eliminates the clock problem, provided synaptic, dendritic, and neuronal integration times can be made longer than the jitter of the circuits. Finally, while the memory problem is the most severe for superconducting digital systems, the prospects for memory in superconducting neural systems are one of the most exciting aspects of the technology. As discussed in Sec.\,\ref{sec:neuroscience}, memory in neural systems involves short-term plasticity, long-term plasticity (STDP, for example), and metaplasticity (that adapts the rate with which synapses change). We find that JJ circuits very naturally implement these functions, and the distributed nature of synaptic memory avoids the difficulty of creating large arrays of addressable superconducting memory elements. 

We will describe these superconducting optoelectronic circuits in more detail in Sec.\,\ref{sec:optoelectronicNeurons}. Let us first turn our attention to JJ circuits that implement neural functions without optical communication.

\vspace{3em}
The binary nature of communication with these synapses is realized by the all-or-nothing response of an SNSPD \cite{thatpapersomeonepresentedatgroupmeeting}.

\vspace{3em}
The plateau of a WSi or MoSi SNSPD renders device insensitive to minor biasing variations, enabling scalable fabrication and operation, satisfying one of Keyes' requirements. 1000 have been fabricated in an array for imaging with 99.7\% yield. Leads into an associated discussion with consideration of exponential dependence of JJ tunneling barrier, sensitivity, difficult for digital logic to tolerate, neuro more tolerant because we seek a distribution of synaptic and dendritic weights, and all the crucial bias currents can adapt dynamically through various plasticity mechanisms to tune distributions and maintain functional operating points.

\vspace{3em}
Each device is quite simple, but they can be combined to realize circuits with a wide variety of relevant neural functionality. Complexity arises from sophisticated arrangements of robust basic components.


\vspace{3em}
Need to mention here that the Stanford architecture and first neural networks by Psaltis intended to utilize incoherent LEDs because they are simple, scalable, and do not require precise control of phase across many optical paths. The same is true here, with the added element that we do not wish to align the resonant frequencies of laser cavities with the gain bandwidth of silicon light emitters that may be narrow