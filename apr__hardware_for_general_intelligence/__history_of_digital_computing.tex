\subsubsection{The Origins of Digital Computing}
Much of nature is best described by analog values. The radius of a planet's orbit can take any value across a broad range. The light output from a star is a smoothly varying function. Yet upon close inspection, that light is discrete. A photon is either there, or it is not. The ancient Chinese saw such dichotomies as central to the balance of nature, represented symbolically in the yin-yang. This concept led them to invent binary arithmetic as a representation of the interaction between mutually interdependent opposites \cite{http://www.atimes.com/leibniz-chinese-invented-first-binary-code/}. Francis Bacon extended these ideas in 1623 to establish that two symbols were sufficient for encoding any communication \cite{dy2012}, a concept that was matured by Claude Shannon during and after WWII \cite{sh1948}. Bacon was also well aware 400 years ago that optical communication was advantageous, such as when signaling between ships, although he probably did not anticipate the complex fiber-optic networks that now span the globe. Leibniz further developed ideas related to binary representation \cite{http://www.leibniz-translations.com/binary.htm} with the intention to utilize binary arithmetic for computing \cite{https://hal.archives-ouvertes.fr/ads-00104781/document,dy2012}. In the 1640s, Pascal devices a mechanical calculator intended to aid in the arithemtic related to taxes, and in 1679, Leibniz proposed a mechanical apparatus for digital computing based on marbles passing through holes to perform logical operations.

Charles Babbage was able to build upon the work of Pascal, Leibniz, and others to create something more complex. His goal was to create mechanical apparatus to tabulate transcendental functions. The result was the Difference Engine, an early machine that could approximate the solution to differential equations. Babbage intended to go beyond this, and in 1834 he conceived the Analytical Engine, a general purpose computer that could perform different functions based on input instructions. Additionally, it could change its own operations based on the output of its calculations, foreshadowing the Turing machine. The apparatus took the principles of automation that had been so successful during the industrial revolution (1760-1840) and applied them to mathematics. The Analytical Engine was never successfully constructed, but its influence was significant intellectually. Ada Lovelace was the most ardent supporter of this machine, and she anticipated a wealth of applications. She developed algorithms and means of programming the Analytical Engine, and her published notes on the subject constitute a significant early milestone in computing. Among her insights were the realization of the power of a general-purpose computing machine that could perform a wide variety of mathematical calculations but could also work with other types of symbols and non-mathematical constructs. Her work established the foundation of computer programming, and her thinking initiated concepts related to artificial intelligence (AI). In reflecting on whether the Analytical Engine would ever be capable of thought, she concluded that it would not. As she wrote, ``The Analytical Engine has no pretensions whatever to \textit{originate} anything. It can do whatever we know how to order it to perform...but it has no power of anticipating any analytical relations or truths.'' \cite{is2014}   

%From the earliest intellectual explorations into the digital domain, it has been clear that one of the great strengths of digital information processing is the simplicity of the binary format. Information provides the ability to answer questions, and a bit allows one to answer a single question. This or that? Yes or no? Despite the simplicity, a more rigorous mathematical foundation (provided by Alan Turing) as well as the motivations of war would be required before Leibniz's vision of a digital computer would be realized.

It would be nearly 100 years before further developments in computing occurred. In 1936, Turing published his seminal work ``On Computable Numbers'' \cite{tu1936}, wherein he introduced the concept of a universal computing machine. The machine\textemdash provided sufficient time and memory\textemdash could execute any computation that is possible, in principle, within the framework established by G\"{o}del \cite{}. Turing's contribution was of fundamental mathematical importance in that it further strengthened the arguments made by G\"{o}del regarding the existence of unsolvable problems (the ``entscheidungsproblem'', as laid out by Hilbert), but Turing's contribution was also of practical importance in that it provided a specific vision for the realization of a universal computer. Turing would go on to design a hardware manifestation of such an apparatus, but his system would never be built. Like the ancient Chinese, Bacon, Leibniz, Babbage, and many others to come, Turing was convinced of the necessity of implementing computation with binary symbols. As he stated in 1947, ``Being digital should be of more interest than being electronic.'' \cite{tu1947} 

%Bardeen, Brattain first demonstration of transistor in germanium 1949 at Bell
\cite{babr1949}

%first Si MOS transistor in 1960 at Bell
\cite{ka1976}

Turing's work on decrypting messages encoded by the enigma machine during WWII led to advances in computing hardware, in part by leading to the construction of Colossus, a British machine used for analyzing wartime communications, but also through the influence Turing's ideas had on John von Neumann. Von Neumann appreciated the significance of Turing's universal computer, both for the insights regarding issues of completeness raised by G\"{o}del, and also for the utility of enabling a single machine to be capable of performing any possible calculation. During the war, von Neumann found it imperative that the United States develop an atomic weapon, and he appreciated the necessity of numerically modeling various aspects of the detonation of nuclear weapons. Beginning with the Manhattan Project in Los Alamos, and continuing at the Institute for Advanced Study at Princeton after the war, von Neumann was dedicated to creation of universal calculators in the mold of a Turing machine, the objective being to numerically analyze arbitrary differential equations, represented as difference equations with binary arithmetic. This objective led to the construction of apparatus with a centralized computing unit that could follow arbitrary (clearly articulated) instructions, and could read and store information in a separate memory unit. The memory unit could contain initial conditions, the results of calculations, and instructions that could be modified during the execution of the computation. 

In 1953, a machine at the Institute for Advanced Studies employing this ``von Neumann'' architecture was primarily used for five types of calculations: nuclear explosions, shock waves, meterology, biological evolution, and stellar evolution. Already at that time, the universal machine was analyzing problems across 25 orders of magnitude in time and roughly the same number in space. During the subsequent 70 years, the von Neumann architecture has been used to answer questions regarding essentially every conceivable subject of any concern whatsoever to human beings. This trajectory represents tremendous success regarding von Neumann's initial goal. Nearly all modern computing is based on digital (binary) information processed in a von Neumann architecture, a scheme devised as a means to realize a Turing machine in electronic hardware. 

Over the course of this time, hardware for computation has shifted from vacuum tubes, in which information is represented by the deflection of an electron beam under the influence of a voltage, to silicon microelectronics, in which information is represented by the flow of electrical current through a transistor under the influence of a voltage. This evolution of hardware has been perhaps as important as the original conception of the architecture  in terms of enabling sustained technological evolution. While hardware has advanced tremendously, the underlying concepts regarding the form of computation performed have remained relatively static. In particular, two key traits of the von Neumann architecture have descended to this day from Turing ancestry: serial information processing, and separation of computation and memory. 

John von Neumann would likely be surprised by the constancy of the architecture. As he stated in 1949, ``There is reason to suspect that our predilection for linear codes, which have a simple, almost temporal sequence, is chiefly a literary habit, corresponding to our not particularly high level of combinatorial cleverness, and that a very efficient language would probably depart from linearity.'' \cite{ne1949} Von Neumann did not see the sequential processing performed by Turing's original conceptual apparatus as an ideal way to go about computing, but rather as a convenient way to get started, and certainly a useful tool for numerical investigation. Other limitations of the Turing machine and von Neumann architecture were also well known shortly after their conceptions. Julian Bigelow, the chief engineer of the Electronic Computer Project at the Institute for Advanced studies, articulated that, ``If you actually tried to build a machine the way Turing described it, you would spend more time rushing back and forth to find places on a tape than you would doing actual numerical work or computation.'' Regarding the von Neumann architecture as implemented with vacuum tubes, Bigelow commented, ``The design of an electronic calculating machine...turns out to be a frustrating wrestling-match with problems of interconnectivity and proximity in three dimensions of space and one dimension of time.'' Integrated silicon microelectronics enabled tremendous advances in computer, in large part because of the ability of lithographically defined wires to achieve extraordinary interconnectivity. Yet the challenges of routing information through space and time still form the core motivation for pursuing neural systems with optical interconnectivity.

Beyond hardware constraints, those at the dawn of electronic computation identified logical limitations as well. Describing generations of computers following the EDVAC, Bigelow stated, ``The modern high speed computer, impressive as its performance is, from the point of view of getting the available logical equipment adequately engaged in the computation, is very inefficient indeed.'' Von Neumann went further, beginning to articulate his vision for new types of logic for computation. He claimed, ``Whatever language the central nervous system is using, it is characterized by less logical and arithmetical depth than what we are normally used to.'' \cite{ne1958} Von Neumann anticipated the highly parallel computation performed by neural systems, and was likely influenced by the work of McCulloch and Pitts nearly a decade prior. In 1943, they showed that perceptrons (devices with computational properties loosely based on the nonlinear behavior of neurons in the brain) could be used to represent any logical expression, much like universal computation performed by a Turing machine \cite{mcpi1943}. 

While some were conceiving of how to utilize principles of neural information processing for computation, Alan Turing was contemplating whether any artificial machine could eventually embody the intelligence demonstrated in the brain. In 1950, Turing published ``Computing Machinery and Intelligence'', in which he argued that artificial intelligence should indeed be possible, and he offered a means to determine if it had been achieved, now referred to as the Turing Test \cite{tu1950}. After 70 years of development of hardware, with devices now defined at the nanometer scale, and architectures such as RISC-V that standardize instructions across Turing machines, silicon microelectronic hardware still appears far from achieving general intelligence. Turing was interested from the beginning in constructing a machine that could think, but the Turing Machine as originally conceived is not adept at the task. A silicon digital supercomputer can provide the answer to very hard math problems, and it can search databases, route information over switching networks, and control robots and automobiles. But such an architecture is not conducive to forming a broad concept of what is going on in the world. The digital system struggles with context, with the inter-relations between quantities. It is the intelligence Turing was attempting to emulate that is most difficult to achieve artificially with the serial operation of the Turing machine. 

To achieve the intelligence of a human being, we must depart from the serial operation of a Turing machine, we must step away from the framework of the von Neumann architecture, and we must begin to speak in languages beyond binary. As expressed by von Neumann, ``A new, essentially logical, theory is called for in order to understand high-complication automata and, in particular, the central nervous system. It may be, however, that in this process logic will have to undergo a pseudomorphosis to neurology.'' \cite{ne1951} This statement in 1951 anticipates the development of neuromorphic computing in calling for a form of complex information processing modeled after neural systems. After his untimely death in 1958, little effort was made to pursue this vision. As historian George Dyson explains, ``The reliability of monolithic microprocessors and the fidelity of monolithic storage postponed the necessity for this pseudomorphosis far beyond what seemed possible in 1948.'' \cite{dy2012} The extraordinary success of silicon technology has made innovation in architecture and logic lack urgency.

The aspirations to achieve general intelligence and to incorporate the processes of the brain to enable new forms of computing have persisted. In 1990, Carver Mead introduced the concept of using silicon transistors to perform the operations of neurons \cite{me1990}. Mead's idea was not to use digital logic to numerically step through the differential equations describing neurons, but rather to use the physics of transistors in the sub-threshold regime to embody an approximation of the dynamics inherent to neuronal operation. Such an approach operates MOSFETs as analog integrators. This idea has been important conceptually to establishing the field of neuromorphic computing, but in practice it has been difficult to achieve high performance, due in part to the challenge of achieving consistent performance from analog devices, the same challenge that led to the victory of digital over analog computing for numerical analysis. 

Nevertheless, Mead's perspective marked the beginning of a trend that has been exponential since. The majority of the field of neuromorphic computing at present attempts to combine the silicon hardware that has been so successful for digital computing with the highly parallel, spike-based computation that is observed in biological neural systems. The goal is to use hardware that can be manufactured economically at large scale to implement the new logic von Neumann anticipated when vacuum tubes were still in use. In this paper we argue that such hardware is not well equipped for neural information processing, but that with a different use of silicon microelectronic manufacturing, new devices and systems can be achieved economically that are tailored to this form of computation. Turing, von Neumann, and Lovelace did not live to see the silicon revolution, nor were they privy to insights from modern neuroscience. From the perspective of the present day, we are in a much better position to design technologies that speak the language of the human brain to achieve the vision of machines that think.

%At present, work in in the field of neural computing is largely divided into two categories. One category involves creation of devices that perform neural functions, such as establishing synaptic weights or achieving relaxation oscillations like a spiking neuron. The other category 
 
\vspace{4em}
Turing/von Neumann/digital: I'm going to give you these instructions and this input. I want you to generate an output that is the one and only (existence and uniqueness theorem) correct answer to a mathematically well-posed question.

brain/AGI: I'm going to give you an enormous amount of information regarding the world and all its parts. I want you to distill the essence, identify the salient relationships, and conceive of it all simultaneously across various spatial and temporal scales. And I want you to explain this world to me.

These are very different objectives. A Turing machine is a universal computer in the sense that if a procedure exists to calculate a given number, a Turing machine can do it. However, two aspects of the Turing model make it inefficient for the latter task. First, information processing is serial. Second, memory must be accessed as an independent step from change of internal state (processing).

\vspace{3em}
Goodfellow, Bengio, and Courville state, ``We simply do not have enough information about the brain to use it as a guide.'' On this important point, I respectfully disagree. It is certainly fruitful to continue developing deep learning even without incorporating further principles  of neural information processing, but there is still much more we can implement in hardware based on the wealth of insights gained throu over a century of neuroscience. We do not have a complete and unambiguous understanding of all aspects of the brain and cognition, but collectively we have learned a great deal about the principles of information processing in the brain. Event the simple architecture of a feed-forward neural net with high connectivity borrowed from the first stages of visual cortex have proven extremely useful for deep learning. Such networks are based on the spatial configuration of certain networks in the brain, and we further know that temporal activity is central to neurosystems. Most deep learning ignores the temporal domain altogether. This is certainly reasonable considering the impressive performance that can be obtained without utilizing the temporal processing strategies of cognitive systems. Nevertheless, the fact that most deep learning to date neglects insights from the brain regarding the time domain means the crucial insight of neural information processing\textemdash that space and time are intertwined\textemdash has gone underutilized by the artificial intelligence community. 

However, it is true that our knowledge of the brain remains incomplete, and we cannot yet build a general-purpose neuromorphic computer based on the operation of the brain that is as trainable and broadly useful for solving important computational problems as deep learning. Computing based on the principles of neural information processing remains a scientific endeavor. 

This article reviews principles of neuroscience as well as neuromorphic hardware, beginning with a summary of neuroscientific literature. The summary of neuroscience is an attempt to identify the basic outline of an intelligent computing architecture. I am a device physicist, not trained in the cognitive sciences, so the picture sketched here is undoubtedly incomplete. Still, I argue that broad strokes of the emerging vision of cognition are sufficient to inform our design of artificial hardware capable of general intelligence. 

\vspace{3em}
bit of history on neural nets, deep learning, disambiguate machine learning, deep learning, reservoir computing, neuromorphic computing



Since the dominance of silicon began in the 1960s there has been a steady stream of attempts to find another platform to surpass it. This has involved III-V semiconductors as well as superconductors, optical approaches, and more recently graphene and 2D materials. There is often a particular feature of a new material or device that is attractive, and, in the face of challenges, it is often assumed that with sufficient time and financial resources, technical problems can be solved, as they were for silicon. To date, no competing technology has proven capable of outperforming silicon when factors including resource availability, processing and manufacturability, device performance, and cost are considered. In this article I argue that hardware based on silicon and leveraging many of the same strengths as silicon microelectronics, but with devices and circuits significantly different than CMOS, holds potential to outperform the transistor approach specifically when functioninig in large cognitive architectures. With this historical context in mind, the burden of significant evidence is on the challenger. Sufficient theoretical and experimental evidence to compel a sea change in the silicon industry is not presented in this paper. Yet I hope the context and reasoning presented here provide motivation for an interdisciplinary community of researchers to further investigate the confluence of superconducting electronics with integrated photonics for neuromorphic supercomputing.


%Silicon: the Semiconductor Material
\cite{heza2004}

\vspace{3em}
Need discussion of Shannon, mutual information, information integration and related metrics

\vspace{3em}
A computation provides the answer to a question. In digital numerical simulations, the question is, ``Given this starting state and these rules for spatio-temporal evolutions, what state will be present at time $t$?'' In vision, the question may be, ``What am I looking at?'' ``Is there a face before me?'' ``What colors do I see?'' At the device level, a synapse may be computing the answer to the question, ``How long has it been since I received an input spike?'' A neuron may compute the answer to the question, ``Is the sum of my inputs above my threshold?''