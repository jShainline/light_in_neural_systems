\section{\label{sec:outlook}Outlook}

\begin{itemize}

\item circle back to Turing and von Neumann, their interests in machine intelligence and modeling computation after the brain
\item circle back to digital vs neural, superconducting optoelectronics brings communication and spiking nonlinearities

\item why go to all the trouble?
\begin{itemize}
\item this technology will only be pursued if it can do something that nothing else can do
\item but it can, and what it can do is very important
\begin{itemize}
\item exceptional complexity for experiments in network information, neuroscience models
\item quantum/neural hybrid systems
\item scaling beyond what is possible with other methods, perhaps the smartest machines on the planet
\item computing has shaped economy and society since its inception
\item powerful scientific tool
\item foundational questions about thought and consciousness amongst the most intriguing and important in modern science
\end{itemize}

\end{itemize}

\end{itemize}

\vspace{3em}
If we find in the long term an alternative hardware platform outperforms cmos for neuromorphic, it is not likely to be simply because another device can provide a better sigmoidal transfer function, but rather because of a suite of considerations from the device to system levels.

\vspace{3em}
SOENs will be able to interface with CMOS (digital or neuromorphic) thorough multiple means. CMOS circuits are likely to be essential for controlling the biases and drive currents to the superconducting network, and in that regard will play important roles in establishing the state of neurotransmitters and affecting activity rates and learning rates in various regions of the network dynamically. CMOS can also drive silicon photonic devices and external light sources to shape photonic signals input to soen synapses via superconducting detectors. Photonic communication directly to and from 4\,K is especially compelling due to the high bandwidth and low headload of fiber optics. 