\subsection{\label{sec:photonic_neural_systems}Photonic Neural Systems}
The strengths of optics in computing have been acknowledged for decades and have led to many attempts to design digital computing architectures around optical phenomena and devices. Like AI and superconductors, there have been several waves of photonics in computing. The first wave of interest began in the late 1970s and was triggered by the seminal paper from Goodman, Dias, and Woody regarding the use of optics for performing Fourier transforms with incoherent light \cite{godi1978}. This work generated a great deal of interest because it showed the power of optics for performing matrix-vector multiplication with complex quantities in a parallel manner using incoherent light that does not have phase as an accessible parameter. While other Fourier transform devices had been proposed, including in slab-guided modes in integrated photonic systems \cite{shha1968,anbo1977}, the work by Goodman, Dias, and Woody combined several key ideas at the right time to generate significant excitement around the use of optics in computing. The optical approach to matrix-vector multiplication introduced in Ref.\,\ref{godi1978} is now referred to as the \textit{Stanford architecture}, as the authors were working at Stanford University at the time of the publication.

Themes raised in the Stanford architecture persist to the present. The primary there is the use of light for highly parallel computation. Each light source can straightforwardly fan its signal out along many optical paths, and parallel computation can occur along each path independently. More generally, the primary strength of optics lies in communication\textemdash the ability to rapidly move information from one place to another without degradation. This strength is fundamental to the uncharged, massless, bosonic nature of photons which allows them to propagate without interaction. This strength is utilized from kHz and MHz frequencies for radio communication up to hundreds of THz for intercontinental data transmission over optical fiber.

Another theme raised by the Stanford Architecture is the use of optics for special-purpose hardware accelerators that perform a specific task with performance unmatched by electronic approaches. Goodman et al. focused on the Fourier transform, but matrix-vector multiplication is the underlying operation. The non-interacting nature of light makes it excellent for parallel data transmission, which enables efficient matrix-vector multiplication. However, to construct a general purpose computer in the model of Turing, nonlinear operations are necessary. Efforts to augment the strength of light for communication with nonlinear optical devices for \textit{all-optical} computing surged in the 1980s following the work of Goodman, Dias, and Woody. These efforts can be separated roughly into two categories, with one track being focused on displacing silicon electronics for digital logic and general-purpose computing, and another track maintaining the purpose of augmenting silicon electronics with task-specific hardware accelerators that were envisioned to work in conjunction with electronic digital computers. The introduction of the Stanford architecture was shortly followed by a wave of interest in neural networks, and the use of optics for neural networks based on a similar architecture was quickly introduced \cite{psfa1985}.

In the following subsection, we briefly review efforts to utilize light in digital computing from the 1970s until the present to establish the context for the evolution of photonic neural systems.

\subsubsection{Optical Logic Elements}
As stated above, the strengths of optics for information processing have been known for some time and have inspired many efforts to conceive of and construct systems for all-optical computing as well as hybrid optoelectronic architectures. The primary strengths of light for information processing are the potential for massive fan out and parallelism, low latency, and high bandwidth. These attributes motivated the Stanford architecture as well as a great deal of other research. The weakness of optics for information processing is that precisely the same as its strength: because photons are uncharged bosons, they do not interact, and therefore they do not have nonlinear responses, and are therefore limited in the input-output transfer functions they can instantiate for computation. Efforts to leverage the strengths of optics for computing have paid significant attention to techniques and devices to give rise to optical nonlinearities that can be leveraged for digital logic gates.

Devices demonstrating optical bistability have been a primary direction of research to enable optical devices that can be used to construct logic gates. The state of the field was reviewed comprehensively in 1982 by Abraham and Smith \cite{abms1982}, and the authors summarize the subject: ``Optical bistability is a general title for a number of static and dynamic phenomena that result from the interplay of optical non-linearity and feedback.'' The 1985 book by Gibbs also summarizes the field at that time \cite{gi1985}. The primary goals of optical bistable devices are to construct optical transistors and optical memory elements. An optical transistor is a device wherein one optical beam controls the propagation of another optical beam. Optical memory is most often implemented by using an optical signal to change the state of a solid-state element, which can then be interrogated optically. It is essentially impossible to induce light to halt its motion, and it is difficult to extend the decay lifetime of photons propagating within a cavity beyond the nanosecond scale, so optical memory based on the storage of photons in a particular location is not feasible.

The first demonstration of optical bistability was achieved by Gibbs, McCall, and Venkatesan in 1976 \cite{gimc1976}. The physical system achieving the bistability comprised a Fabry-Perot interferometer filled with sodium vapor. The sodium vapor provided a nonlinear dispersive medium so that the transmitted intensity was a nonlinear function of the incident intensity. Miller, Smith, and Johnston leveraged a similar effect in a Fabry-Perot cavity formed from an InSb crystal in 1979 \cite{mism1979}. In this work, the nonlinear refraction was based on electronic nonlinearity related to states below the semiconductor band gap, and the use of a semiconductor crystal rather than an atomic vapor made the system more likely to be useful in complex computing systems. Reference \onlinecite{mism1979} demonstrated optical differential gain and bistability. Because the system enabled a weak beam to modulate the transmission of a powerful beam, this system provided the first example of an optical transistor. The overall small signal power gain was greater than six. Subsequent work integrated devices leveraging this principle as a 2D array of etalons \cite{jele1986,vewi1986} for the purpose of performing logical operations. 

An alternative approach to bistable optical devices was based on the self-electrooptic effect device. Work on self-electrooptic devices was also carried out by Miller, beginning in 1984 \cite{mich1984}. The principle of operation is based on adjustable absorption controlled with an electric field applied perpendicular to the plane of the quantum well that shifts the band edge, known as the quantum confined Stark effect \cite{mich1984b}. In a self-electrooptic effect device, the quantum confined Stark effect as well as optical detection are present in a single structure, leading to optoelectronic feedback that produces bistability. A number of configurations of such devices have been implemented, with the symmetric self-electrooptic effect device introduced by Lentine et al. being the most successful \cite{lehi1988,lehi1989}. In this implementation, two $p$-$i$-$n$ diodes with quantum wells in the intrinsic regions are combined, each serving as the load for the other. As Lentine states, ``This device has complimentary outputs whose switching point is determined by the ratio of the two optical input powers and acts as a set-reset latch.'' One strength of this device is that it has time-sequential gain, meaning once set with low-power beams, it can be subsequently read out with high-power beams. This results in good input-output isolation because the large output occurs at a later time than the inputs. Additionally, while previous electrooptical devices with optical bistability required critical biasing very close to a nonlinear threshold, no such biasing was required.

A switching network comprised of symmetric self-electrooptic-effect devices was demonstrated by McCormick et al. in 1993 \cite{mccl1993}. This free-space optical system comprised a 32$\times$32 switching node array of these devices. Light propagated perpendicular to the plane of the device array, and 16 input channels were connected to 32 output channels with each channel transmitting 32 bits in parallel. The input data to be routed entered on optical fibers and was coupled to free space. Six stages of electrooptic devices were present, each with input control lasers, electronic control through an electronic computer, and with optics between the stages. The output was again coupled to fibers. (consider showing figure 2 from \cite{mccl1993})

Other approaches to the creation of optical transistors have continued to be proposed as nanoscale patterning of photonic devices has become commonplace. One example is the design of an optical transistor based on a photonic crystal cavity \cite{yafa2003} that was presented in 2003 by Yanik, Fan, Solja\v{c}i\'{c}, and Joannopoulos. In this on-chip structure, two waveguides lead into a photonic crystal cavity, and two waveguides lead out. In this case, the photonic crystal cavity plays the role of the Fabry-Perot utilized by Miller et al., and a control beam input into one of the inputs shifts the resonance of the cavity through the Kerr nonlinearity. With this beam present, transmission of the signal beam is high, while without the control beam, transmission of the signal beam is low. The authors simulated this operation with material parameters corresponding to AlGaAs, and found that with a modest $Q$ factor of 5000, the contrast ratio between on and off states could be as high as 10 with a few milliwatts of input power. 

Another optical transistor leveraging nano-scale optical phenomena is based on switching a single molecule between internal electronic states and has been demonstrated by Sandoghdar's group in 2009 \cite{hwpo2009}. Other recent efforts to realize bistable optical devices for computing and telecommunications include the use of graphene \cite{wawu2016,guru2017}, semiconductor quantum wells \cite{suzh2019,saeb2018}, and semiconductor quantum dots \cite{lili2019}. 

After 40 years of effort in optical logic devices, no candidate looks promising to displace the electronic transistor for large-scale digital computing. As Miller stated in 2010, ``Only one device has apparently ever satisfied [the success] criteria well enough to allow large logic systems to be constructed.'' \cite{mi2010} Miller was referring to the self-electrooptic-effect devices of Lentine \cite{lehi1989} and the switching systems of McCormick \cite{mccl1993}. Difficulties with optical logic devices do not mean the field of photonics has stalled in the past several decades. To the contrary, tremendous progress has been made. Much of this progress has been enabled by the birth of silicon photonics. Let us put this discussion of optical logic devices to the side while we discuss integrated silicon photonics. Then we will attempt to synthesize what the past 40 year of research in optical technologies can teach us regarding the design of hardware for cognitive systems.

%Based on this brief summary of optical logic devices, can we identify common trends that can guide future thinking on the use of optics in computing in general and for cognitive hardware in particular?

\subsubsection{Integrated Silicon Photonics}
By the mid 1980s, silicon microelectronics was well established as the supreme technology for computing. The scaling predictions made by Moore in 1965 \cite{mo1965} had held for two decades, while other material platforms for electronics and photonics had not met with the same success. Within this context, a new perspective on the role of light in computing was put forward in a series of papers by Soref, Lorenzo, and Bennett from 1985 to 1987 \cite{sole1985,sole1986,sobe1987}. The state of the field was reviewed by Soref in 1993 in Ref.\,\cite{so1993}. While other materials (primarily compound semiconductors) had been developed for integrated optical components prior to the work by Soref et al., this series of papers pointed to the potential for integrated optical components to be incorporated with silicon microelectronics monolithically. As the authors stated, ``Silicon is a `new' material in the context of integrated optics even though Si is the most thoroughly studied semiconductor in the world. There is reason to believe that Si can serve as the medium for a variety of guided-wave optical components in much the same manner as III-V semiconductor compounds, while at the same time avoiding the inherent complexities of binary, ternary, and quarternary alloys.'' \cite{sole1986} They further describe their two primary motivations: 1) to utilize the fabrication processes that have been developed for the Si electronic circuit industry in the production of photonic devices; and 2) to monolithically combine silicon electronic circuits with guided-wave optoelectronic components.

In the first paper of the series, passive waveguides and power dividers were demonstrated at $\lambda = 1.3$\,\textmu m, a wavelength at which optical fibers are highly transmissive, indiciating the potential for silicon photonic components to interface with both electronic computational infrastructure as well as fiber-optic communication infrastructure. Optical confinement was achieved in the vertical dimension by epitaxially growing intrinsic silicon on a heavily $n$-type doped silicon substrate, which has a slightly lower index of refraction \footnote{The index change due to free carriers can be calculated with the classical dispersion formula given in Ref.\,\cite{sole1985}: $\Delta n = -(q^2\lambda^2/8\pi c^2n\epsilon_0)(N_em_{ce}^*+N_hm_{ch}^*]$, where $q$ is the charge of an electron, $\lambda$ is the optical wavelength, $n$ is the refractive index of the intrinsic material, $\epsilon_0$ is the permittivity of free space, $c$ is the velocity of light in vacuum, $N_e$ ($N_h$) is the concentration of donors (acceptors), and $m_ce^*$ ($m_ch^*$) is the effective mass of electrons (holes).}. In the lateral dimension, confinement was achieved by etching a fraction of the depth of the epitaxial intrinsic silicon, leading to a so-called rib waveguide configuration \footnote{A rib waveguide results from partial etching of the high-index layer, while a ridge waveguide results from etching completely through the high-index layer. Foundations of optical waveguide theory can be found in Refs.\,\cite{snlo1983} and \cite{hu2009}.}. The waveguide structure is shown in Fig.\,\ref{fig:soref_waveguide}(a) (use Fig. 2 from \cite{sole1985}). The passive power splitter is shown in Fig.\,\ref{fig:soref_waveguide}(b). These are the first silicon photonic components.

In the second paper on the subject, Soref and Lorenzo describe active silicon photonic components based on free-carrier dispersion effects \cite{solo1986}. Because the silicon lattice is centrosymmetric, a linear electrooptic effect is not present in bulk crystals, and therefore many had not considered silicon a candidate for active optical components. The insight by Soref and Lorenzo to utilize free-carrier effects instead opened many opportunites that would lead to over three decades of technology development. In Ref.\,\onlinecite{solo1986}, silicon-germanium compounds were also proposed as complimentary materials for waveguiding, and silicon-on-insulator (SOI) structures were considered as candidates for a layer structure capable of optical confinement.

At the time of the original work by Soref et al., SOI technologies were just beginning to be developed, and epitaxial intrinsic silicon grown on heavily doped silicon gave the best waveguiding results. As silicon-on-insulator technology improved\textemdash particularly through device-layer transfer rather than epitaxial growith\textemdash the situation changed dramatically. The ability to fabricate a thin layer of intrinsic crystalline silicon on top of an electrically insulating layer with lower index of refraction has proved consequential for both electronic and photonic devices. The basic layer structure of SOI is shown in Fig.\,\ref{fig:soi}.




\vspace{3em}
Goal was to follow the model of the integrated circuit, utilize lithographic fabrication and system complexity that can be achieved through integration of components on a chip. 

\vspace{3em}
discuss fibers as an intermediate step
\paragraph{Components Required for Integrated Photonics}
\begin{itemize}
\item passives: waveguides and routing, beam splitters and power taps (y-junctions, evanescent couplers, adiabatic 50-50 beam splitters), spectral filters (microrings \cite{ra2007} and gratings), waveguide crossings
\item sources
\item detectors
\item electrooptic: modulators and phase shifters (rings, MZIs \cite{ohno1975}), SAW transducers for imparting a phase shift
\end{itemize}

\paragraph{Materials for Integrated Photonics}
\begin{itemize}
\item III-V
\item LiNbO$_3$
\item silica
\item silicon
\item SiN
\end{itemize}

\vspace{3em}
Active integrated photonic components were beginning to be developed in the mid 1970s \cite{ohno1975}, and by the mid 1980s many elements of the field of integrated photonics were taking shape \cite{ve1984}. Primary applications included RF signal processing and analog numerical processing \cite{ve1984}. Proposals for all-optical digital computers were made based on the key element of a bistable optical cavity. However, the important sub-field of silicon photonics had not yet emerged.

\vspace{3em}
In 1987, Soref and Bennett introduced the concept of using the shift in index of refraction that results from free carriers in silicon to achieve active optical components based on silicon waveguides \cite{sobe1987}. This insight would have to wait until the development of silicon-on-insulator wafers in the early 2000s to be put into practice. Since then, an explosion of activity has occurred in the rapidly developing field of silicon photonics. Electro-optic effects have been used to make a variety of modulators \cite{rema2010} operating into the 10s of GHz based most commonly on Mach-Zehnder interferometers \cite{lisa2005} or microring resonators \cite{xuma2007}. In addition to the free-carrier electro-optic effects, in 1993 Soref also pointed to thermo-optic effects as a means to make dynamic photonic components on an optoelectronic chip \cite{so1993}. The combination of electro-optic effects for fast index perturbation and thermo-optic effects for slow resonance tuning, in conjunction with etched silicon waveguide structures in silicon-on-insulator substrates, established a foundation of active components capable of signal switching, filtering, and modulation. In his 1993 paper, titled \textit{Silicon-based optoelectronics}, Soref presented a more expansive view of the potential for what he termed ``superchips'' that combine the strengths of photonics and electronics monolithically on a single silicon chip. Silicon had long been the material of choice for integrated microelectronics, but Soref had identified a path to make silicon also a powerhouse in photonics as well.

To make use of silicon as a waveguiding medium so that the active components described above can be implemented, one must utilize light with photon energy less than the band gap of silicon ($E_{\mathrm{g}}=1.17$\,eV/$\lambda = 1.06$\,\textmu m at 0\,K; $E_{\mathrm{g}}=1.11$\,eV/$\lambda = 1.12$\,\textmu m at 300\,K). The buried oxide of silicon-on-insulator wafers becomes absorptive for $\lambda \gtrsim 2$\,\textmu m. Thus, the transparency window of silicon-on-insulator waveguides enables operation with wavelengths below 1.2\,\textmu m, and includes the important telecom bands (O-band:1260\,nm-1360\,nm; C-band: 1530\,nm-1565\,nm), whose significance results from the very low attenuation of optical fibers at these wavelengths. Thus, silicon integrated photonic components can be interfaced with optical fibers for communication across long distances. 

Yet if a material is transparent, it is not efficient for detecting light. To create photodetectors in silicon waveguides, two approaches are taken. One approach is to utilize SiGe regions patterned in Si waveguides, as the band gap of Si is narrowed by the incorporation of Ge. Germanium is present in many contemporary CMOS processes for strain engineering, and can be economically incorporated in the foundry because, like silicon, it is a group IV element, and therefore shares process compatibility and does not act as a dopant in Si. Waveguide-integrated \cite{} and resonator-integrated \cite{} SiGe detectors operating at the O-band and C-band. These detectors have been demonstrated with high efficiency approaching 1\,A/W. The other approach is to introduce defects the silicon lattice, either through ion implantation or the use of poly-crystalline or amorphous silicon. These defects introduce absorptive states within the band gap. Detectors based on this principle have been demonstrated with $x$\,A/W responsivity \cite{meor2014}. 

%The Introduction of Silicon-on-Insulator
SOI in early 2000s, guiding light on a chip is a different ballgame

\vspace{3em}
Integrated Silicon Photonics for Communication above a certain length scale in digital electronics
-monolithic with processors?
-in package?
-off-chip light sources?
-WDM


%promising materials were GaAs and InSb

%Fast and reliable storage using a 5 bit, nonvolatile photonic memory cell
\cite{liyo2018}

%integrated-optical approaches to numerical optical processing
\cite{ve1984} In 1984, Verber stated, ``A problem which has been common to almost all efforts to perform numerical computations by optical techniques has been the accuracy limitation imposed by the intrinsically analog nature of the devices. Although the ultimate solution to this problem is generally accepted to be in the application of optical bistable devices to produce fully binary optical systems, the long development times anticipated for these systems has led to several other approaches to higher accuracy computation using analog optical techniques.'' Approaches to improving the accuracy of analog optical computations have been proposed \cite{psca1980,arha1984}, but neither analog nor digital approaches to all-optical computing have been successful. In 2019, these problems have not been solved, and binary optical systems have not proven capable of displacing CMOS for digital computing. 

The inability of all-optical systems to displace silicon microelectronics for digital computing does not mean light has no role to play in advanced computing generally and neural systems in particular. However, this history should be carefully considered when choosing the specific role for light in cognitive hardware.

\vspace{3em}
After decades of research, the phrase ``all-optical'' should now set off alarm bells. One must ask if the advantage gained (usually speed) is worth neglecting to utilize the myriad competencies of electrical circuits. (or) One must ask why it might be advantageous to neglect to utilize the myriad competencies of electrical circuits. Optics can be employed in addition to electronics without omitting electrical circuits completely. The goal of all-optical is usually speed and parallelism, and introducing electrical components can reduce bandwidth. Still, for systems as complex as required for cognition, it is unlikely an all-optical solution will surface.



\vspace{3em}
``...it is the need to limit power dissipation that largely constrains clock rates in current electronic devices\textemdash lower operating voltages give slower speeds but correspondingly lower energies per operation.'' \cite{mi2010}.

\vspace{3em}
``...we have to remind ourselves that the field of integrated optics is still in its infancy, still in its research stage, and still searching for its proper role.'' \cite{ko1981}
``The arguments for optical wiring are understood even down to the chip-level, but chip-scale optical interconnect technology is still in its infancy.'' \cite{mi2010}

Displacing Si electronics has failed for the reasons Keyes has pointed out.

At present, a major goal of photonics is to augment CMOS electronic hardware to aid in communication. Optical communication shows indisputable advantages over long distances, as exemplified in global fiber optic networks as well as local-area networks. On the chip scale, the advantages of optical communication must contend with the challenges of optoelectronic hardware integration. 


\vspace{3em}
Need to mention here that the Stanford architecture intended to utilize incoherent LEDs because they are simple, scalable, and do not require precise control of phase across many optical paths.

\subsubsection{Free-Space Optical Neural Nets}
The late 1980s 

%Optical information processing pased on an associative-memory model of neural nets with thresholding and feedback
\cite{psfa1985}
-focused on Hopfield model 
-seeking to use parallelism and interconnectability of optics (linear) with bistable optical devices (nonlinear)
-this paper and \cite{faps1985} should be discussed together
-this paper is a theoretical proposal of the system implemented in \cite{faps1985}
-GaAs LEDs with photodiodes and electronics for nonlinearity and feedback to be replaced by optical bistable devices
-such bistable optical devices have still not reached sufficient maturity to enable large-scale systems or commercial products, but some neuromorphic proposals are still based on them
-according to \cite{juyu1996}, this paper is ``generally considered to be the first paper to introduce the idea of the optical implementation for a neural network.'' \cite{juyu1996}

%Optical implementation of the Hopfield model
\cite{faps1985}
-optical implementation of content addressable associative memory
-nonlinear iterative feedback to a vector-matrix multiplier
-numerical and experimental results
-emphasize strengths of optics: parallel processing and massive interconnection capabilities
-regarding Hopfield model, the authors state, ``A remarkable property of the model is that powerful global computation is performed with very simple, identical logic elements (the neurons).''
-also cite \cite{psfa1985}
-LEDs proposed as one means of implementing the logic elements: LED off, state = -1; LED on, state = 1
-LED proposal uses free space propagation, photodetection, thresholding, gain, and feedback (nonlinear feedback)
-for optical vector-matrix multiplier circa 1978, cite \cite{godi1978}
-array of LEDs represents input vector, and an array of photodiodes detects the output vector
-``The output is thresholded and fed back in parallel to drive the corresponding elements of the LED array.''
-multiplication by the connection matrix uses horizontal imaging and vertical smearing on an optical mask with an anamorphic lens system
-this is the downfall: the free-space optical setup is not conducive to fieldability or programmability. a software engineer cannot sit down at a computer and try new things. one needs a phd in optics to have any fun.
-aspired to eliminate electronic feedback and replace with optical feedback, thereby complicating things further, but they envision a monolithic setup
-LEDs were envisioned to be replaced by bistable optical devices (cite Keyes and his refutations)
-experimental demonstration of 32 neuron system with 32 LEDs
-with soens, we must agree: such hardware will not be adopted instead of silicon microelectronics if the goal is moderate-scale Hopfield networks or similar machine learning systems. to displace CMOS, any new hardware must do something CMOS cannot do.
-also cite \cite{jaju1988} for optical Hopfield

%Optical implementation of the {H}opfield model for two-dimensional associative memory
\cite{jaju1988,}
-two-state neurons are implemented with liguid crystal optical switches
-three-dimensional holographic interconnections are also implemented with liquid crystals
-demonstrated 4x4 array

%Architectures for optoelectronic analogs of self-organizing neural networks
\cite{fa1987}
-seeking to add artificial intelligence to conventional computer controllers
-focused on supervised learning by: ``1) computing the interconnectivity matrix for the associations that they are to learn and 2) changing the weights of the links between their neurons accordingly.''
-``Such self-organizing networks therefore have the ability to form and store their own internal representations of the associations that they are presented with.''
-seek to use optics to speed up learning of Hopfield and Boltzmann networks, specifically Boltzmann
-similar architecture to \cite{faps1985}, still LED array
-computer-controlled spatial light modulator to apply weights
-learning is done with a particular algorithm implemented with detection, electronic and computer processing, and updating the SLM.
-key quote: ``[T]he addition of such a module to a computer-controller through a high-speed interface can be viewed as providing the computer-controller with artificial intelligence capabilities by imparting to it neural net attributes.''

%Coeherent optical eigenstate memory
\cite{an1986}
-associative memory
-free-space optical cavity
-``A gain medium internal to the resonator amplifies the field belonging to the eigenmode that most resembles the injected field; the other eigenmodes are suppressed through a competition for the gain.''
-optical resonator with spherical mirrors remembers Hermite Gaussian fields
-stored information can be programmed
-volume hologram playing the role of a mirror can be programmed to store what the user wishes
-experimental demonstration with two orthogonal patterns

%Overview of hybrid optical neural networks
\cite{juyu1996}
-review circa 1996
-``Neural networks that involve optical implementation are commonly called optical neural networks, although they should be correctly called hybrid optical neural networks....[T]hey always require non-linear functions that are difficult to implement optically. Most proposed optical neural networks leave this non-linear function to be implemented electronically.'' 
-``From a practical point of view, we do not intend to make an artificial brain; instead we are attempting to construct an information processing system that mimics some behaviour of the brain and differs significantly from the von Neumann type computer architecture and algorithm.''
-They cite two motivations to study neural networks in the 1980s: ``1) Although a computer can perform complex calculations precisely and quickly, when the task is to recognize an object the computer fails or performs poorly, and is extremely slow compared with the human brain or even an animal brain. 2) A computer needs an explicit set of instructions to perform a given task. Thus, a computer is like a loyal and strong slave but who lacks intelligence. Nevertheless, computer users also need intelligent workers, in addition to slaves, that can perform the task without receiving detailed orders.''
-distilled to recognition capability and learning capability
-limit discussion to feed-forward, binary neurons
-emphasize that the key attribute of optics in feed-forward NNs is (incoherent) matrix-vector multiplication, referencing Goodman \cite{godi1978}
-implementation with LEDs still prominent
-discuss both incoherent and coherent implementations
-Merits of optics: 1) The velocity of optical signals is independent of the number of interconnections; 2) optical signals are immune to mutual interference effects; 3) Optical signals can propagate in three-dimensional free space; 4) the interconnection can be altered properly using spatial light modulators; and 5) optical signals can be easily converted into electronic signals.
-``The principal reason for using optics in neural networks, is that the very large interconnection or fan-in fan-out, of the order of $10^3$ to $10^6$, is quite impractical for VLSI technology because it uses a unique discrete channel for each input or output \cite{caki1989}. On the contrary, we may conceive that an optical neural network is not significant unless each neuron has more that 1000 interconnections with other neurons.''
-learning in free-space optical setups is not easy and requires supervision. by contrast, implementing learning with electronic circuits is far more tractable.
-``The architecture of a neural network is characterized by interconnection and non-linear operation. In principle, optics is naturally suitable for implementing the interconnection. On the other hand, the non-linear operation may be implemented by electronic means, including digital computers.''
-``Hybrid optical neural networks may not materialize as commercial products unless each neuron in the network requires more than 1000 interconnections to solve problems.''

%Optical Neural Networks
\cite{caki1989}
-review circa 1989
-``Optical neural networks are the offspring of two parents: optical information processing and neural network theory.''
-On strengths and weaknesses of brains relative to digital machines: ``Formal reasoning is hard, but inspired guessing is easy.''
-``This is a vast and growing field. We will attempt a fair and balanced overview of the field but certainly not a complete literature survey.''
-``Very large fan-in/fan-out ($10^3$ - $10^6$) is quite impractical for VLSI technology because it uses a unique discrete channel for each input or output.''
-``Most optical neural networks use unconfined or free-space interconnections which avoid these problems by allowing freely crossing and overlapping interconnection paths.''
-``[L]arge numbers of neurons and interconnections are the natural domain of optics.'' 
-identify six optical interconnection methods present at that time: 1) fiber optic fan-in/fan-out; 2) holographic in-plane connection in integrated optics \cite{fees1988}; 3) optical parallel matrix-vector multipliers \cite{faps1985}; 4) lenslet-array multiple imaging \cite{fami1986}; 5) thick holographic associative networks; and 6) fixed hologram arrays \cite{ca1987}.
-holograms can store a ton of information, but limit of interconnectivity is set by size of photodetectors on a surface

%The interconnectability of neuro-optic processors
\cite{jast1986}
-focused on the optical interconnect network with free-space optics
-focused on holographic technique
-illustrated the complexity of the optics required to utilize this approach 
-very high storage density, but impractical to implement compared to integrated systems on a chip
-they find in a feasible system the potential to realize $10^8$ neurons and $10^{16}$ synapses that can be trained on $10^6$ inputs

%Parallel optoelectronic realization of neural networks models using CID technology
\cite{agne1988}
-``Further advances in this field are...limited due to the absence of efficient and reliable hardware realizations of NN models.'' Perhaps, but advances in CMOS proved efficient for the deep learning boom.
-``incorporate optics where microelectronics fails...the interconnectivity problem.''
-Hopfield model with binary neurons
-salient point is that SLM above detector array achieves connectivity matrix
-not clear how sources are integrated

%Parallel $N^4$ weighted optical interconnections
\cite{ca1987}
-main point here is NxN input plane can be fully connected to NxN output plane realizing $N^4$ connections.
-uses NxN SLM as well as NxN hologram array
-limited to two layers
-in principle capable of $N^4 = 10^{12}$ connections

%Optical implementation of the {H}opfield neural network using multiple fiber nets
\cite{itki1989}
-uses optical fibers for interconnection between nodes in a Hopfield network
-``The coupling ratio from fiber to fiber represents the synaptic weight of the connection between units.''
-intensity of light encodes synaptic weight
-5x5 array demonstrated, $25^2$ connections
-laser diodes and silicon photodetectors
-one network for positive elements, one for negative 

%Holography in artificial neural networks
\cite{psbr1990}
-review article summarizing neural networks implemented with semiconductor light sources, photorefractive materials for hologram that serves as host to a multitude of diffraction gratings that apply the synaptic weights and direct the light beams to realize connections, and semiconductor photodetectors to receive signals
-``In recent years there has been a marked resurgence of interest in artificial neural networks.'' 1990
-they list three primary motivations to move toward neural architectures: massive parallelism; dense interconnections; learning; all these years later, the motivations are still the same, but the implementations are different
-arguments against: free-space routing only moves light in straight lines, hard to make dense interconnectivity of recurrent networks, everything is most straightforwardly laid out in a series of planes, with some planes having detectors and sources, potentially with computational circuits, and other planes have diffraction gratings, but the source/detector planes block light from being able to reach beyond them, so it really only works for feed-forward networks wherein connections only occur between adjacent layers; potentially useful for deep learning, not useful for cognition
-uses encoding in optical intensity
-only passive synapses and dendrites
-synapses share the entire volume of the holographic medium, so they cannot be changed independently, much like MZI network
-``Distributed connections have advantages and disadvantages\textemdash the disadvantage compared to the localized implementation is the reduced control of individual synapses. The adjustment of the strength of one synapse may inadvertently affect other synapses as well. Accommodating this limitation is perhaps the most crucial research issue in this field....''
-``The advantages of the distributed holographic synapses are high storage density and ease of fabrication.''
-This shared storage/synaptic cross talk precludes stdp; these challenges have not been solved.
-$10^{12}$ synapses per cm$^3$ in principle, $10^9$ in practice, but it doesn't matter because you cant use them
-``[T]he learning approach can provide a method for transferring some of the burden of programming the computer from the user to the computer.''
-in the 90s, it was assumed that optical approaches referred to free space; integrated photonics and especially silicon photonics had not yet launched
-semiconductor detectors, apply the nonlinearity in the electronic domain, generate light amplitude proportional to electronic output; can do all three on a monolithic chip
-GaAs neurons: two transistors, a photodetector, and an LED
-writing the appropriate gratings in photorefractive materials requires optics with phase-conjugate mirrors, which reflect light back along their original path
-this is a beautiful use of sophisticated nonlinear optics, but has not proven to be sufficiently advantageous to merit the incorporation of sensitive optical apparatus in mainstream computing technology
-strength of synapse determined by optical amplitude during writing
-establishing synaptic weights requires cumbersome optical procedures, and writing each new synaptic weight can inadvertently others, so the whole thing must be done self-consistently
-photorefractive LiNbO3 crystal
-Hebbian-type, local, associative learning, not backprop in 1990
-photorefractive crystals: photoconductive, electro-optic, and have traps that can be optically ionized
-``The recorded hologram is stored in the spatial distribution of the ionized traps in the crystal.'' ``[W]e must...design a sequence of exposures that can load the appropriate weight values in the finite pool of trap sites that are available.''

\vspace{3em}
\cite{waps1987}
holographic gratings in photorefractive crystals store associative memories and use backprop for training \cite{waps1987}, similar to recent Shanhui Fan, nonlinear etalon for sigmoid response; intended for image classification, not much different than deep learning today with sigmoid activation functions and backprop. ``This architecture combines the robustness of the distributed neural computation and the backpropagation learning procedure with the hight speed processing of nonlinear etalons, the self-aligning ability of phase conjugate mirrors, and the massive storage capacity of volume holograms to produce a powerful and flexible optical processor.'' Potential reasons this technology didn't catch on: free space optics, bulky, difficult to package, not fieldable, requires experience with optics to operate, difficult to scale to large systems, nonlinearity utilizes bistability \cite{gi1985} which has a history of being difficult to manage, based on photorefractive effect which is volatile, free-space interconnectivity is bulky and difficult, requires many sources, high power for nonlinear effects for hologram and etalon, how to get data in?, fan-out limited by gain of nonlinear devices and dictates an information-collapsing network architecture

\vspace{3em}
\cite{psbr1988}
-photorefractive crystals as media for holographic interconnections in neural networks
-derive fundamental limitations to connections, but practical limitations are very different
-feed-forward neural nets
-``There is a nice compatibility between simple (multiplicative) Hebbian learning and holography; the strength of the connection between two neurons can be modified by recording a hologram with light from the two neurons.'' 
-``...in an optical implementation each grating corresponds to a separate interconnection between two neurons...''

\vspace{3em}
%Optical Neural Computers
\cite{abps1987}
``It is the ability to establish an extensive communication network among processing elements that primarily distinguishes optical technology from semiconductor technology in its application to computation.'' \cite{abps1987} However, in that and other related work at the time, the objective was to utilize optical signals propagating in free space to interconnect neural processing elements. Using light in this manner has the advantage that ``multiple beams of light can pass through lenses or prisms and still remain separate,'' \cite{abps1987} yet routing of free-space optical signals brings new challenges for complex neural systems. To achieve connectivity graphs corresponding to neural networks with feed-forward, feed-back, and recurrent connections, light cannot travel only in straight lines, but rather must branch and change direction many times. Construction of complex networks with free-space optics, mirrors, and lenses quickly leads to issues related to scaling. 

As Goodman pointed out in 1985, ``quote from Goodman's paper'' The benefits of light for fan-out can be difficult to harness in free space. We may expect the situation to improve with on-chip waveguides. ``Two beams of light, unlike a pair of current-carrying wires, can cross without affecting each other.'' \cite{abps1987} At the same time that free-space optical neural computers were being developed based on holographic memory, the field of integrated photonics was emerging.

%optical implementation of associative networks
\cite{fili1987}
-comparison of several optical associative networks
-emphasis on multimodule approach, actually referring to network layers
-layers are nonlinearly connected
-each module (layer) is a complete, adaptive associative memory which can be modified by exposure to sets of associated information patterns, eg feature vectors
-presentation of one pattern u should result in recall of paired pattern v
-focused on optical hardware implementations for associative modules
-argues that optics is better for low-precision analog arithmetic, but soens uses optics for binary communication, with analog operations performed in superconducting domain based on high beta L loops.
-interesting summary of arguments in favor of optics for various computational tasks
-proposals for optical holographic memories and optical perceptron neural networks go back to the 1960s (cite fili1987 and references therein)
-dont need to discuss this paper in detail, just a review circa 1987 of optical approaches to associative memories, basically a published study by the navy

\vspace{3em}
%Optical associative memory model based on neural networks having variable interconnection weights
\cite{maar1987}
-Modification to Hopfield network
-Simulated results
-No physical optical system built
-Optics for readout in vector-matrix multiplication

%Adaptive multilayer optical neural network with optical thresholding
\cite{safi1995}


\vspace{3em}
Need to mention here that the first optical neural networks by Psaltis intended to utilize incoherent LEDs because they are simple, scalable, and do not require precise control of phase across many optical paths.

%Coherent optical neural network that learns desirable phase values
\cite{kahi2003}
-coherent optical neural network that learns a desired output phase as a function of frequency
-uses liquid crystal SLM
-very application-specific, intended to be used in optical communications or adaptive coherent optical image processing
-parameters of neuron (inputs, outputs, weights) given complex values corresponding to light-wave properties
-it is not clear how to utilize these wave properties of light in complex neural systems, as they require precise control at every synapse. we often would rather not have to keep track of all these parameters with control knobs, and unsupervised means of utilizing them are not obvious.
-more likely to me that specialized modules may utilize color and phase, but central cognitive module will be much simpler to ensure scalability
-learning with complex-valued Hebbian rule, updates achieved with a computer changing SLM settings

\vspace{3em}
Switching elements made from nonlinear optical materials require too much power to scale, and often require III-V materials. This leads to discussion of optical transistors, in which one optical beam controls the transmission of another. ``Each element can be either a purely optical switch or an optoelectronic combination of light detector, electronic switch and light emitter.'' \cite{abps1987}

%more recent

%Reinforcement learning in a large-scale photonic recurrent neural network
\cite{buma2018}
-Free-space optics; scalable to order $10^5$ nodes; demonstrated reinforcement learning with 2025 nodes; each node is a pixel of a spatial light modulator; connections established with diffractive optical element; learning with a digital micromirror device. All weights are positive and all output weights are binary.
-The general architecture is not conducive to very large scale. The elements that perform the computation are based on integration of digital electronics with mechanical mirrors. The routing is not conducive to general networks with hierarchical construction and Rentian scaling. It is hard to get light out of the module. The power per node is high because light levels are high (compared to single photons). Optical setup sensitive to alignment. Requires external control for learning procedure. Readout of network state performed with a camera. The optoelectronic system steps through discrete time with a specified update equation. In the 2018 demonstration, update of 900 nodes occurred at 5\,Hz, limited by software control of the spatial light modulator. Diffractive optical element establishes synaptic weights. Scaling limit due to ``the imaging setup's field of view and not by the concept itself.''
-This is not a model for spiking neural networks. Synaptic weights are static after the training phase and do not permit any of the short- or long-term plasticity mechanisms nor dendritic processing that are central to the manner in which spiking neurons utilize time. This network has binary outputs that are readout in parallel. Internal node states are represented by optical field intensities, and therefore such a concept cannot be extended to the single-photon domain. Memory reconfiguration DOE is passive, but light is always on.
-``We demonstrate a network of up to 2025 diffractively coupled photonic nodes, forming a large-scale recurrent neural network.'' Terms like ``large-scale'' need to be defined in introduction (large scale means at least 1000 logic gates, tens of thousands of transistors per chip, VLSI still refers to chips with tens of billions of transistors per chip.)

\vspace{3em}
%All-optical machine learning using diffractive deep neural networks
\cite{liri2018}
-Limited to feed-forward, non-spiking, no synaptic or dendritic complexity
-Inference with light, training/design done on external computer, backprop
-Demo: MNIST, actual experiment, performed with 400 GHz light; also fashion MNIST
-Multiple planes of diffractive elements cascaded
-Claim that inference is performed at the speed of light, but data received by photodetectors has to be processed on standard digital machine, and new inputs have to be generated by electronics controlling input optical fields.
-No synaptic computation, just usual f(sum(wij xj))
-Bulky, table-top implementation. Difficult to imagine this will displace CMOS for deep learning
-Making same arguments as 1980s for photonics: ``Optical implementation of machine learning in artificial neural networks is promising because of the parallel computing capability and power efficiency of optical systems.''
-Not clear how this is really new as compared to Psaltis, Wagner et al. They apply the diffractive masks differently, use 3D printers, THz rather than optical, but basically the same concepts.

%Large-scale optical neural networks based on photoelectric multiplication
\cite{hasl2018}
-Also using free-space optics, feed-forward, non-spiking, no synaptic or dendritic complexity
-Senior authors of Ref.\,\cite{shha2016} identified spatial scaling as a fatal problem with networks of on-chip MZIs, and are presently championing a free-space approach instead, while the lead authors have spun out start-ups based on the on-chip approach
-this free-space approach uses homodyne interference to apply synaptic weights (still just f(sum(wij xj)))
-extraordinary claim that an optical neural network can accomplish a multiply-accumulate operation with less than one photon of energy. they arrive at this number by assuming unity efficiency of photon generation, detection, no loss anywhere in the system, and some MACs can fail to occur without loss of classification accuracy due to the redundancy of the network
-simulated MNIST
-signals from photodetectors read out serially, nonlinearity applied electronically, so really just using a complex system of free-space optics to perform wij*xj, with the sum and nonlinearity performed electronically
-serialization a form of time-multiplexing
-not intended to be an approach to dynamical systems
-they argue the approach is scalable to $10^6$ neurons

%hybrid optical-electronic convolutional neural networks with optimized diffractive optics for image classification
\cite{chsi2018}
-use a free-space optical neural network prior to electronic processing
-they ``incorporate a layer of optical computing prior to either analog or digital electronic computing, improving performance while adding minimal electronic computational cost and processing time.''
-incorporate convolutional layers
-``By pushing the first convolutional layer of a CNN into the optics, we reduce the workload of the electronic processor during inference.''
-``[A]n imaging scenario where the input signal is already an optical signal easily allows for propagation through additional passive optical elements prior to sensor readout.''
-incorporate an optical convolutional layer optimized for a specific classification problem (compare to facial recognition in the brain)
-assume light is incoherent and monochromatic
-results in more hardware complexity, as now a CNN would require a free-space optical setup; probably only useful in artificial vision systems where free-space light is the signal, not when digital images need to be processed.



\subsubsection{Deep learning with silicon photonics}
Like superconducting neural systems, the goal of nearly all efforts in optoelectronic neural systems and neuromorphic photonics is not to develop general intelligence, but rather to realize neural systems for specific tasks such as inference or control. For most efforts, the motivation for using light is the speed, either of laser cavity dynamics or optical communication. Device and hardware choices toward these ends may be different than for the focus of this article, which is general intelligence. We intend to explain why specific choices are not conducive to the present goal, even if they are suitable for other applications.

We consider deep learning to be based on feed-forward networks of non-spiking neurons trained through a supervised algorithm such as backpropagation. While markedly different from the recurrent networks of dynamical nodes that learn from experience through local plasticity mechanisms, the relative simplicity of deep learning makes it a natural place to begin utilizing principles of neural information processing. Feed-forward neural networks have been studied with free-space optics since the height of optical computing excitement in the late 1980s and early 1990s, and after the developments in silicon photonics following Soref, similar principles were developed in an integrated context. 

%plunked this paragraph down after the rest
Matrix-vector multiplication has been a draw toward optics for some time \cite{godi1978,ve1984,maar1987}, with an early proposal appearing on pg.\,1 of vol.\,2 of Optics Letters \cite{godi1978}. Other approaches to matrix-vector multiplication have emerged over the years as technology has evolved, and the approach that is currently receiving the most attention is based on the concept of implementing a unitary operator with an array of Mach-Zehnder interferometers by Reck et al. in 1994 \cite{reze1994}. With the addition of loss or gain, any matrix can be represented through its singular-value decomposition \cite{st2016}. Because additional MZIs can be used to discard light, the full singular-value decomposition and matrix-vector multiplication can be performed using MZIs. Implementation of such MZI networks with silicon photonic waveguides is currently being pursued \cite{mi2015_fix_ref,shha2017}, and it has been argued that efficient training through backpropagation can be implemented in optoelectronic integrated circuits with tunable MZIs and photodetectors working on conjunction with CMOS logic at each interferometer \cite{humi2018}.

%another out-of-context comment
hybrid devices utilizing an artificial nonlinearity implemented with an electro-optic effect and electrical feedback have been studied since the early 1980s \cite{sm1980,ko1981}

The operation of synaptic weighting in deep learning reduces to matrix-vector multiplication. Such an operation can be achieved with an array of Mach-Zehnder interferometers. A recent demonstration accomplished this using thermo-optic phase shifters with silicon waveguides \cite{shha2016}. A network with four inputs and outputs was trained to classify four vowel sounds. The effort led to two start-up companies attempting to commercialize the technology to compete with specialized CMOS processors (such as tensor processing units) for deep learning. The photonic approach demonstrated so far made use of off chip light sources and detectors, and applied the nonlinearity in software. For such an approach to be competitive, significant system integration is required. The two senior authors of Ref.\,\cite{shha2016} have more recently moved back to a free-space approachin to deep learning \cite{}.

The approach of using 2-D arrays of interferometers for routing and synaptic weighting pursued in Ref.\,\cite{shha2017} is incompatible with large-scale cognitive systems for several reasons. One reason is that the index shifts induced by thermo-optic phase shifters are small, and power dependent, leading to either large structures, high power consumption, or both. Cross talk between thermal elements necessitates placing the waveguides far apart, and it is difficult to utilize the vertical dimension interferometer arrays, so attempting to scale results in networks that are sprawling in the plane. Further, as described in Sec.\,\ref{sec:neuroscience}, an important mechanism of learning in spiking neural systems is through STDP, wherein the activity of the two neurons associated with a synapse leads to memory adaptation. With interferometer arrays, changing a single phase in the network will, in general, modify several synaptic weights. Therefore, while backpropagation can be implemented with such a network \cite{humi2018}, STDP cannot. 

%Design of optical neural networks with component imprecisions
\cite{fama2019}
-focused on feed-forward MZIs for deep learning
-their goal is to design networks with built-in phase shifts for specific classification problems, then mass-produce a bunch of these at once
-in this modality, tolerance to fabrication errors is very important
-these are static MZIs, no dynamic reconfigurability, no relevance to learning machines for general intelligence
-standard singular-value decomposition in the linear layers
-they mention graphene saturable absorbers for nonlinearity, discarding entirely the benefits of scalable fabrication
-they're considering optical bistability in microring resonators and two-photon absorption as alternatives, so they don't have a plan for that
-just a simulation study, no experiments

\subsubsection{Photonic Reservoir Computing}
-most are actually with delay systems, but some are not
-Bienstman (Ghent) and others have done theory related to reservoir systems without delay using both SOAs and Si microrings in the swirl topology to avoid waveguide crossings
-first delay system paper was electronic, \cite{apso2011}
-first photonic delay system paper \cite{laso2012}
-similar work using MZIs and SOAs for nonlinearity followed
-big names are Fischer (Spain), Soriano (Spain), Brunner (Spain), Massar (Ghent)


%Advances in photonic reservoir computing
\cite{vabr2017}
-review of photonic reservoir computing
-``much of the promise of photonic reservoirs lies in their minimal hardware requirements, a tremendous advantage over other hardware-intensive neural network models.''
-two main approaches: multiple discrete optical nodes and single node with delayed feedback.
-this review covers the material i have already covered here, so cite it as a summary
-also a few more refs: using sesam for nonlinearity \cite{dedu2014}; others using semiconductor lasers with feedback \cite{ngve2014}
-cite this article (\cite{vabr2017}) and say to see their bibliography for complete references regarding photonic reservoir computing
-consider using Figs. 9,10 to summarize delay system

\vspace{3em}
%---------------------
%Ghent/Bienstman
%---------------------

%Toward optical signal processing using {P}hotonic {R}eservoir {C}omputing
\cite{vadi2008}
-proposed photonic reservoir computing for large scale pattern recognition problems
-network of semiconductor optical amplifiers as the basic building blocks
-``Rather than simulating a nonlinear element using a software algorithm, we propose to implement such an element using a photonics device. This could have advantages in terms of speed and power efficiency.''
-utilize dynamics resulting from interactions between photons and electrons in semiconductor lasers
-numerical simulations
-photonic reservoir, electronic readout
-not initially obvious that coupled SOAs will make a good reservoir
-seek to avoid waveguide crossings and are limited to one plane of waveguides, so simulated connectivity is limited

%Parallel Reservoir Computing Using Optical Amplifiers
\cite{vada2011}
-follow on to \cite{vadi2008}
-still simulations, investigating design parameters and significance of process variations
-word recognition task
-most important design parameters are ``the delay and the phase shift of the system's physical connections.'' this makes sense in the system under consideration (SOAs) because the time constants are very fast and the transfer function is phase sensitive
-sensitive to SOA noise
-three reasons to choose SOAs: 1) they have gain, so no separate component is needed to compensate for loss; 2) they are broadband, so fabrication tolerances are relaxed as compared to resonators; 3) their stead state characteristic resembles the upper part of a sigmoidal activation function, so knowledge gained from software may be applicable
-memory of elements is set by carrier lifetime, 100-300\,ps
-4x4 swirl topology dictated by the desire to avoid waveguide crossings
-9x9 network simulated
-``The result of the tanh network without leak rate at its optimal delay, is actually comparable to a tanh with leak rate and no (or minimal) delay. This reinforces the view that delay is an alternative approach to introducing memory next to leak rate.''

%Cascadable excitability in microrings
\cite{vafi2012}
-theory and experiment
-using free-carrier/thermal dynamics
-si microrings
-only keep free-carrier concentration and temperature so they can perform phase-plane analysis
-cascadable excitation demonstrated between two rings

%Experimental demonstration of reservoir computing on a silicon photonics chip
\cite{vame2014}
-gets rid of SOA that were the focus of their previous work
-uses linear, passive silicon photonics
-nonlinearity is implemented at the readout photodetectors
-taking the weighted sum of the states is currently being done offline, but it is ``conceptually easy to also implement this linear combination of states in the optical domain, where a set of variable optical attenuators or modulators implement the weights.''
-16 node square mesh that contains multiple feedback loops
-connections are 2cm spirals with 1.2 dB loss per spiral
-280ps delay per edge
-these delays bring the timescales down to the point where electronics can manage them
-demonstrate Boolean operations with memory
-XOR, which they argue is non-trivial
-2-bit XOR with one bit from current time step and one from previous
-for XOR of bits from different delays, need to train different readout

%All-Optical Reservoir Computing on a Photonic Chip Using Silicon-Based Ring Resonators
\cite{desc2018}
-simulation
-swirl topology
-4x4 grid of nodes
-ring resonators as nodes, utilizing nonlinearity due to interplay between optical mode, free carriers, and thermal coupling to the environment
-xor as task
-can perform xor at 20 Gb/s with error rates lower than 1e-3 using 2.5 mW

%---------------------
%end Ghent/Bienstman
%---------------------

\vspace{3em}
Recurrent neural networks can approximate the trajectory of a dynamical system \cite{funa1993}
Recurrent neural networks are Turing equivalent \cite{kisi1996}

%Reservoir computing concept introduced in:
\vspace{3em}
%Harnessing Nonlinearity: Predicting Chaotic Systems and Saving Energy in Wireless Communication
\cite{jaha2004}

\vspace{3em}


\vspace{3em}
%Advances in photonic reservoir computing
\cite{vabr2017}


%Delay Systems
General comments:
-Not modular or hierarchical without multi nodes, which requires complex timing between nodes, ends up with similar problems to time multiplexed interconnects

\vspace{3em}
%Information processing using a single dynamical node as a complex system
\cite{apso2011}
-first demonstration of reservoir computing with a delay system
-implemented electronically
-Delay systems are defined as ``Nonlinear systems with delayed feedback and/or delayed coupling.''
-good introduction to delay systems and reservoir computing
-spoken digit recognition and dynamical system modeling
-400 nodes implemented with bulk analog electronics for spoken digit recognition

\vspace{3em}
%Tutorial: {P}hotonic neural networks in delay systems
\cite{brpe2018}
-``Inspired by a strongly simplified interpretation of the human brain's structure, large numbers of simple nonlinear elements (neurons) are connected (synaptic links) into large networks. Information processing in ANNs usually relies on numerous simple nonlinear transformations and large scale linear matrix multiplications. The implementation of these operations in von Neumann architectures is highly inefficient, as it requires massive parallelism.''
-``[E]ven the recent astonishing developments cannot mask the fact that currently no ideal ANN-specific hardware, which fully implements physical hardware neurons and physical \textit{synaptic links}, exists. With such a novel platform, many orders of magnitude could be gained in speed and energy efficiency.''
-``Reservoir computers offer a compromise between performance and an implementation-friendly ANN topology.''
-``[S]implicity is bought in expense of time multiplexing and de-multiplexing in the input and readout layer.'' 
-``[T]emporal multiplexing results in a reduction of the system's overall processing bandwidth by the number of neurons $N$.''
-The authors emphasize that in a delay system, ``[A]ll nonlinear transformations carried out by the virtual spatiotemporal network rely on the same physical component.'' The is the primary limitation to achieving complexity in such systems. The single node of the system quickly becomes a bottleneck, as all spatial degrees of freedom are represented as temporal degrees of freedom. Such technology is capable of implementing various types of artificial neural networks, which is the goal of the research, but it is not a promising route as a computational primitive for hardware intended to achieve cognition.
-To increase the effective dimensionality of the system, the total delay time must be increased to allow inclusion of information from a larger number of virtual nodes. This reduces the speed with which a given computation can occur.
-Used backpropagation through time.
-Time-delay systems are an approach to reservoir computing that minimize the demands on hardware by extending computation across time rather than across space. A single nonlinear node can perform a computation similar to that achieved by a neural network, but the computation is spread out across time rather than space \cite{brpe2018}. The state of a node during a time window in a delay system corresponds to the state of a node at a spatial location in a neural network. Such an approach is appealing in the context of emerging hardware because a system comprising only a single node can perform a useful calculation, albeit with a slowdown due to time multiplexing proportional to the number of emulated nodes. Using the time domain in this way is somewhat reminiscent of information processing in biological neural systems wherein each node is engaged in different transient ensembles at different times, although the specific mathematical formalism describing delay systems as manifestations of recurrent neural networks are not nearly as general as the information processing principles of the brain, nor do they aspire to be.

\vspace{3em}
%Photonic information processing beyond Turing: an optoelectronic implementation of reservoir computing
\cite{laso2012}
-first photonic delay reservoir paper
-LiNbO3 MZI modulator for nonlinearity
-400 virtual nodes
-21us delay time

%Optoelectronic reservoir computing
\cite{padu2012}
-first experimental reservoir computer based on an opto-electronic architecture (actually tied with \cite{laso2012} to within five days)
-experimental details: laser diode input; LiNbO3 MZI modulator; output splits 25\% to detector, 75\% to EDFA and then fiber delay (50.4 km) for echo-state network (247.2 us delay); delayed light is detected and signal affects modulator; additional input signal drives modulator
-the authors acknowledge theirs is similar concept to \cite{apso2011} except using photonics instead of electronics for nonlinearity and delay

%All-optical reservoir computing
\cite{dusc2012}
-same folks as \cite{padu2012}, but instead of using MZI, this paper uses semiconductor optical amplifier for nonlinearity
-similar to theory of Beinstman, but using a single dynamical node with delay
-cite \cite{luja2009,taya2019} for reservoir computing reviews circa 2009,2019
-``The architecture we use is based on fiber optics delay loop with a single nonlinear node and off-line training. The nonlinearity is provided by the saturation gain effect in a SOA.''
-so, same as \cite{laso2012}, but using SOA instead of MZI, so they call it all optical and consider this a strength.

\vspace{3em}
%Parallel photonic information processing at gigabyte per second data rates using transient states
\cite{brso2013}
-semiconductor laser with delayed self-feedback
-demonstrate spoken digit and speaker recognition and chaotic time-series prediction at data rates beyond 1 Gbyte/s
-break delay cycle into N different, time-multiplexed transient states, typically with $100\le N \le 1000$.
-data preprocessing and readout are carried out off-line
-with N = 388, total loop delay time is 77.6 ns =$\tau_D = T_2$
-The time allocated to each ``neuron'' is $T_2/N = 200ps = T_1$, but really one must use $T_1 = 200$\,ps because that is the characteristic time of relaxation oscillations of the laser, so you have $T_2 = NT_1$. This gives 5,000 neurons in a microsecond, 5 million neurons in a millisecond, and the scale of cortex would require a second, meaning gamma oscillations could not be faster than a second. Of course, there are many other reasons why information integration would not be possible at this scale.
-For a network with N = 388, 10 mJ per digit of spoken digit recognition, better than desktop computer by 100x.

\vspace{3em}
%Brain-{I}nspired {P}hotonic {S}ignal {P}rocessor for {G}enerating {P}eriodic {P}atterns and {E}mulating {C}haotic {S}ystems
\cite{anha2017}
-reservoir computing
-goal is to achieve long-term prediction of time series by feeding the output from a reservoir back to itself
-seek to understand the means by which biological circuits generate time series
-also seek to enable technological generation of time series with photonic reservoir computing
-optoelectronic delay system utilized, similar to \cite{apso2011,padu2012,laso2012}
-readout uses FPGA, high-speed dedicated electronics
-electronics demonstrated in \cite{andu2016}, but here they are used very differently: to feed the reservoir output back into the reservoir

\vspace{3em}
%All-{O}ptical {R}eservoir {C}omputing on a {P}hotonic {C}hip {U}sing {S}ilicon-{B}ased {R}ing {R}esonators
\cite{cosc2018} 


\vspace{3em}
%Conditions for reservoir computing performance using semiconductor lasers with delayed optical feedback
\cite{bubr2017}
-semiconductor laser with delayed feedback
-connect its injection locking, consistence, and memory properties to RC performance in a non-linear prediction task
-partial injection locking achieves a good combination of consistency and memory
-``experimental identification of the best operation conditions for time series prediction tasks in a semiconductor laser based reservoir computer.''
-one challenge in similar systems is that the dynamics of semiconductor lasers is very complex when signals are injected. stability is an issue. finding optimal operating points can be challenging. may be okay with a single dynamical node, but difficult for neurons based on excitable lasers

\vspace{3em}
%A Unified Framework for Reservoir Computing and Extreme Learning Machines based on a Single Time-delayed Neuron
\cite{orso2015}
-identifies the similarities between extreme learning machines and echo state networks
-implements both on a single hardware platform
-switching between the two by activating or deactivating one physical connection
-same implementation as \cite{padu2012} and \cite{laso2012}, single time-delay neuron

%delay systems
In delay systems time is used to emulate space. This makes efficient use of hardware in that very few nodes can behave as many. This delay technique is a form of time multiplexing, and as such scaling results in significant latency. Such systems are useful in situations where power and hardware resources are at a premium, but this use of space and time is not conducive to the fractal use of space and time associated with cognition.

%Fischer et al
\cite{laso2012,apso2011,orso2015,brfi2015,brso2013,bubr2017,buma2018,brpe2018}

%Delay dynamics of neuromorphic optoelectronic nanoscale resonators: {P}erspectives and applications
\cite{rofi2017}

\subsubsection{Spiking Neurons with Semiconductor Excitable Lasers}
While the interferometric approach to deep learning discussed above makes use of static neurons, several approaches to spiking neurons have been pursued as well. One class of spiking photonic neurons leverages the carrier dynamics in compound semiconductor laser cavities. It has long been known that the equation governing lasers with gain and saturable absorber regions are isomorphic to the leaky integrate-and-fire neuron \cite{dukr1999}, with the number of excited carriers in the laser playing the role of the membrane potential. This correspondence has led to several designs \cite{nata2013} and experimental efforts (see Ref.\,\cite{prsh2017} and reference therein) to leverage this behavior to make spiking neurons that sum optical signals and produce optical pulses when a threshold has been reached. This work began in Er-doped fibers, and continues with on-chip implementations with III-V photonic systems, with much of the work being done in the Prucnal's group at Princeton. The refractory period of such neurons is set by the cavity photon decay time and is on the order of 10\,ps, while the integration time is set by the carrier relaxation time, and is on the order of 100\,ps. This short refractory period means such neurons can fire up to $10^9$ times faster than biological neurons, yet the short integration time means temporal correlations amongst neuronal firing events is forgotten rapidly. 

While the goal of these efforts in excitable lasers is to perform neuro-inspired computing very rapidly with small networks, and not to achieve brain-scale systems, we nevertheless point out two features of this approach to using light in neural systems that are not conducive to achieving large-scale systems. The first is power consumption. To properly set the threshold of these neurons, the gain region must be continuously pumped. This requires between 100\,mW and 1\,W per neuron, even when the neuron is not firing. For a system of $10^{10}$ neurons, a gigawatt would be consumed, even with the system at rest. The second limitation regards computation. As discussed in Sec.\,\ref{sec:neuroscience}, neural information processing leverages many complex computations in synapses, dendrites, and neurons. In excitable lasers, all the computation occurs in the interaction between photons and carriers in the laser cavity. Multiply-accumulate operations can be performed with leak and threshold, but no path toward short-term synaptic plasticity or dendritic processing have been proposed. By relying on the exponential decay constants of photons and carriers, one is unable to tune the range of temporal information processing or supply the dendritic arbor with information across a wide range of temporal scales. These computations and time constants are more readily achieved in the electronic domain with circuits that can be engineered to perform complex functions rather than relying on material parameters, a point we revisit below.

%artificial neuron based on integrated semiconductor quantum dot mode-locked lasers
\cite{meka2016}
-InAs/InGaAs semiconductor quantum-dot passively mode-locked laser
-extensive bibliography of similar neurons, but argue theirs is unique because it can implement inhibition (\cite{alva2013} creates inhibition with phase)
-2 mm Fabry-Perot cavity
-``inhibition and excitation are associated with waveband switching effects triggered solely through optical injection from another optical neuron.''
-the inhibitory neuron emits in an alternate band, which results in suppression of the activity in the excitatory band
-sensitive to biasing conditions
-will be extremely difficult to find biasing conditions where all neurons in a large network can work well together and be tuned to excite/inhibit appropriately
-5V bias
-optical isolator between two neurons to isolate signal from one neuron from reflecting back to itself (Keyes)
-extremely fast 20GHz repetition frequency
-high power 60 mW peak power
-they want to avoid utilizing inhibition in the electronic domain because it is too complex
-the scheme does not lend itself to dense integration and scaling, perhaps possible up to a few neurons

%Ultrafast all-optical implementation of a leaky integrate-and-fire neuron
\cite{krro2011}

%Recent progress in semiconductor excitable lasers for photonic spike processing
\cite{prsh2016}
-big review article by Prucnal et al
-72 pages, you have not read this. likely redundant with book

%A leaky integrate-and-fire laser neuron for ultrafast cognitive computing
\cite{nash2013}

%SIMPEL: Circuit model for photonic spike processing laser neurons
\cite{shna2015}
-photodiodes accomplish inhibition

%Spike processing with a graphene excitable laser
\cite{shna2016}

%Controllable spiking patterns in long-wavelength vertical cavity surface emitting lasers for neuromorphic photonics systems
\cite{huja2015}

%Excitability in optically injected microdisk lasers with phase controlled excitatory and inhibitory response
\cite{alva2013}
-Ghent/Bienstman
-make comparison to pulsing si microrings used for reservoir computing
-demonstrate class I excitability in optically injected microdisk lasers and propose a possible optical spiking neuron design
-threshold and integrating behavior
-optical phase control can be used to generate inhibitory response
-input pulse power around 1uW for 0.2 ns for threshold
-no transfer of excitation between disks demonstrated

%Optical neuron using polarisation switching in a 1550nm-{VCSEL}
\cite{huhe2010}

%Solitary and coupled semiconductor ring lasers as optical spiking neurons
\cite{coge2011}

\vspace{3em}
%Ultrafast photonic reinforcement learning based on laser chaos
\cite{nate2017}
-application specific to a particular reinforcement learning problem
-using laser chaos as a source of randomness that outperforms pseudorandom number generators

%Reconfigurable semiconductor laser networks based on diffractive coupling
\cite{brfi2015}
-provide a means to couple many semiconductor lasers
-8x8 array of single-mode VCSELs, square lattice, pitch of 250um


\subsubsection{Wavelength-Division Multiplexing for Routing and Synaptic Weighting} 
In addition to the work on excitable lasers as spiking neurons, the Princeton group has also pioneered the use of concepts from wavelength-division multiplexing for both signal routing and synaptic weighting \cite{tana20142,tafe2017}. Within this framework, each neuron within a cluster produces or modulates light at a distinct wavelength upon firing. The signals from all neurons within the cluster are multiplexed onto a single broadcast waveguide, and all other neurons tap all colors from this waveguide and apply synaptic weights based on the frequencies of microring resonances relative to the neuron wavelengths. For a cluster of $N$ neurons, $N$ different colors of light must be generated, $N$ microring filters must be used to multiplex these signals onto the broadcast waveguide, and each neuron must have $N-1$ microring filters to receive and weight the signals from all the other neurons. Thus, a cluster of $N$ neurons requires $N^2$ microring resonators. This approach to communication between neurons is referred to as ``broadcast-and-weight'', and is closely related to the operation of wavelength-division multiplexing in fiber communication networks.

Again, the goal of the work from the Princeton group is not to achieve brain-scale systems, but rather to ``...find out the minimum ensemble of behaviors that are necessary to harness similar processing advantages.'' \cite{prsh2017} Nevertheless, adopting wavelength-division multiplexing concepts from larger-scale communication networks down to the chip scale is intuitive and aesthetically appealing, so it is worth pointing out why it ends up not being conducive to reaching large-scale cognitive systems. To begin, it is important to distinguish between using the wavelength of light for multiplexing multiple signals on a broadcast bus and the use of microring resonators to establish synaptic weights. The Princeton group uses both techniques, but it is possible to employ one or the other independently. When using wavelength for multiplexing, the advantage is that space can potentially be saved. Instead of each neuron having an independent axonal arbor to reach its downstream connections, many neurons share a single distribution waveguide. However, the area saved is significantly reduced by the fact that $N^2$ microring resonators must be employed. More important than area is power. Because microring resonances are so sensitive to minor variations in fabrication, each of the $N^2$ resonators must be actively aligned to the appropriate wavelength corresponding to the emission from the associated neuron. This typically requires on the order of 1\,mW. For a brain-scale system of $10^{14}$ synapses, 100\,GW would be required just to align the communication network. The power consumed for alignment limits scalability, but so does the procedure for carrying out the alignment. Each of the microrings must be aligned, and if thermal tuning is employed, significant cross-talk will occur. Implementing such alignment for systems of more than a few neurons becomes quite cumbersome. Additionally, the wavelengths of the neurons can only be spaced so closely if cross talk is to be avoided, and the gain bandwidth of the light sources is limited, so a limit of roughly 200 neurons within a cluster is encountered. One may think of such a cluster as analogous to a mini-column in the brain, but unfortunately communication between mini-columns is hindered by the use of wavelength for multiplexing. In order to communicate between mini-columns, a neuron must first communicate from its local cluster up to a higher level of hierarchy where the same colors are re-used, and then down again to the target cluster. Such a communication protocol severely limits the graph structures and path lengths that can be achieved (see Sec.\,\ref{sec:neuroscience}. It is intuitive to leverage wavelength multiplexing in photonic neural systems to maximize use of bandwidth, but when used in this way wherein each neuron is uniquely identified by a color, scalability is severely hindered.

These considerations pertain to using wavelength for multiplexed routing, but there are independent reasons why using microring resonators to establish synaptic weights is not conducive to scaling. One challenge associated with microring weight banks is the fact that by changing a certain parameter (power delivered to heater, for example) the synaptic weight first increases, then saturates, the decreases as the resonance passes the target wavelength. This makes it very difficult for supervised or unsupervised learning to occur. Additionally, the shape of the resonance is nonlinear with very steep sections. Thus, to achieve uniform changes in synaptic weight, a nonuniform change in drive must be applied, and across much of the range of weights, the synaptic weight will be very noisy.

Microring weight banks and Mach-Zehnder interferometer networks have two things in common: they both require implementing phase shifts in photonic components (which usually draws power, even in the steady state), and neither is capable of implementing STDP or other unsupervised learning techniques. To achieve the largest-scale neural systems, it is highly advantageous if storage of a synaptic weight draws no power. For a system at the scale of the brain, if each synapse draws even 10\,nW in the steady state, the system will consume 1\,MW just to remember what it has learned. 

%Demonstration of WDM weighted addition for principal component analysis
\cite{tach2015}

%A silicon photonic modulator neuron
\cite{tafe2019}

\subsubsection{Phase Change Materials for Synaptic Weighting and Neural Thresholding}
One technique for establishing synaptic weights between neurons signaling with light is to leverage phase-change materials \cite{chri2017}. Such materials have the property that the coefficient of optical absorption is different between the two phases. Therefore, a variable attenuator can be devised wherein the crystallization state of a small patch of phase-change material integrated on a waveguide determines how many photons are transmitted through the synapse. Reference \cite{chri2017} showed that such a synapse could be used to implement a form of Hebbian learning, wherein two pulses incident closely in time could strengthen the synaptic weight by adjusting the crystallinity of the material and reducing absorption. 

Such Hebbian update in this system represents a novel route toward synaptic weighting in photonic neural systems. Unfortunately, the material studied in Ref.\,\cite{chri2017} requires billions of photons for Hebbian update, thereby exceeding the communication energy limit of a single photon by at least nine orders of magnitude. Additionally, the patch of phase-change material has no way of keeping track of the order in time or even the source of input pulses, so anti-Hebbian synaptic weakening cannot be achieved, and a route to full STDP has not been proposed. 

\subsubsection{Synaptic weights in the electronic domain}
We have discussed here three approaches to establishing synaptic weights in photonic neural systems: interferometric networks; microring resonators; and phase change materials. These approaches all have one thing in common: they treat the synapse as a variable attenuator, and change the weight by varying the number of photons that pass through the synapse. Communication in biological neural systems is binary, and the synaptic weight is enacted based on how much post-synaptic current is generated, and is independent of the amplitude of the action potential reaching the pre-synaptic terminal. By contrast, if one establishes the synaptic weight in the photonic domain, communication is analog, and the number of photons in the pulse\textemdash analogous to the amplitude of the action potential\textemdash now carries information. This has two detrimental consequences. First, it requires that each neuron produce more photons that would be necessary for binary communication, and many photons are discarded at weak synapses. This is a power penalty. Second, setting the synaptic weights in the photonic domain means that any noise on the transmitting neuron light sources results in additional noise received by the neuron. This is an information-processing penalty.

The alternative is to set the synaptic weights in the electronic domain. The synaptic response is independent of the number of incident photons, and the synaptic weight is stored and implemented by an electronic circuit. Provided a synaptic terminal receives a photonic signal surpassing a certain threshold, a synaptic event is induced. The physical limit on the amplitude of this threshold signal is a single photon. Establishing the synaptic weight in this manner is most straightforward if each synapse is equipped with an independent photodetector. For integration with CMOS, the waveguide-integrated SiGe or defect detectors described above are good candidates. Logic circuits based on MOSFETs are the clear choice to implement synaptic, dendritic, and neuronal computations, and transistors operated in analog may play a role. Upon reaching threshold, the transistor circuits would drive a pulse through an on-chip laser, and the light thus produced would fan out to downstream connections. At those connections, as long as a number of photons greater than the threshold were received, the synaptic response would ensue, thus eliminating the effects of any noise on the photonic communication signal. The challenge here is the same at that mentioned above: it is hard to integrate light sources on silicon. If a million III-V or SiGe sources can be integrated on a 300-mm silicon optoelectronic wafer in a cost-effective manner, such an approach to optoelectronic networks will be viable.

To reach the physical limit of single-photon synaptic threshold, superconducting-nanowire single-photon detectors (SPDs) can be used. We will describe these detectors in more detail in the next section, but for the present discussion we point out that these detectors respond to single photons, and their response is nearly identical \cite{} if one or more than one photon is detected. Thus, neuronal communication using these detectors enables the lowest possible communication signal level, and sources must produce only enough photons per synaptic connection so that even with noise, each synapse receives at least one photon, with a chosen tolerable error rate. Such communication appears to saturate a physical limitation for neuronal signaling with photons of a given wavelength. Whereas transistors were used for computation in the hardware example above, if SPDs are used for detection, circuits of JJs the clear choice for computation. Because SPDs and JJs both require operation near 4.2\,K, optoelectronic hardware operating in this modality has the potential to utilize silicon light sources, potentially bringing a tremendous advantage in cost and scalability. In the next section we will describe the synaptic, dendritic, and neuronal functions of these circuits. 

\subsubsection{Where are the Light Sources?}

\paragraph{Considerations for Digital Communication}
\vspace{3em}
For digital communication it makes sense to modulate a CW laser because the photons are only discarded half the time. With neurons firing sparsely, tapping a CW light stream every time a neuron fires is wasteful. Almost all the light\textemdash which is the system's most valuable resource\textemdash is simply thrown away. If many neurons are multiplexed on the same light source, they suffer cross talk, which becomes particularly problematic when the fire synchronously. By using light sources that only generate light when the neuron fires, photons are not wasted during quiescent periods. Because each neuron has its own emitter, cross talk does not occur. 

In the case of silicon light sources based on point defects that are currently under investigation for this application, emission occurs in a sharp zero-phonon line with 0.3 nm bandwidth with x\% emitted in broader phonon assisted sidebands with 10 nm bandwidth. All emitters are identical, so design of passive components is straightforward. Whereas the presence of the phonon sideband is problematic for quantum application requiring pure quantum states, light in the phonon sideband is still useful in this application and does not represent an impediment to operation. Most importantly, the utilization of silicon light sources enables scalable fabrication unlike any other approach. These light sources are suitable for this application because the technology only requires incoherent pulses of light with 10\,ns emitter lifetime and 1\,\% efficiency. Because single-photon detectors can be utilized, the emitters do not need to be particularly bright, generating only one to 10 photons per connection to compensate for propagation loss and Poisson noise. The sources are as simple as possible, as are the detectors, and the result is a highly scalable hardware platform tailored to this type of information processing.


\paragraph{Silicon Light Sources: the Great Achilles' Heel}
So if photonic switches, modulators, filters, and detectors can all be implemented in silicon, why do all silicon microelectronic chips not have photonic components? There is one reason: a simple, inexpensive light source integrated with silicon waveguides operating at room temperature does not yet exist. Silicon has an indirect band gap, so optical emission requires a phonon for momentum conservation. This three-body process (electron, hole, phonon) is rare, so non-radiative recombination dominates. Regardless, if silicon is to be used as a passive and active waveguiding material for routing, switching, and modulation, a source emitting at a longer wavelength must achieved, just as detectors must absorb at longer wavelength, as described above. If detectors can be made to accomplish this, why is the same not true for sources? Despite efforts for decades \cite{shxu2007}, an economical, efficient, room-temperature, waveguide-integrated light source on silicon has not been discovered. To understand the source challenges, let us briefly consider three means by which researchers have attempted to create silicon light sources. More comprehensive surveys can be found in the literature \cite{li2005,shxu2007,libo2010,zhyi2015}.

%\begin{figure} 
%    \centering{\includegraphics[width=8.6cm]{silicon_absorption_emission.pdf}}
%	\captionof{figure}{\label{fig:silicon_absorption_emission}Caption.}
%\end{figure}
Like the case of detectors, two approaches to creating light sources on silicon are band gap engineering with Ge alloys and introduction of states in the gap via lattice defects. While detectors based on SiGe have shown decent performance without extensive process development, the same cannot be said of SiGe sources. Poor material quality is not as problematic if the goal is to make an absorber, whereas non-radiative recombination pathways introduced by material defects greatly limit the efficiency of SiGe as a light source and lead to high threshold current for lasing \cite{zhyi2015}. Thus, despite the process compatibility of SiGe with CMOS, SiGe lasers to date have not high enough performance with low enough cost to find a market. 

Similarly, light sources based on defects in silicon have been studied extensively for decades as the silicon microelectronics industry has matured \cite{da1989}. While defect-based detectors have demonstrated useful performance and low cost at room temperature, defect-based light sources have not. To understand why, consider a three-level model of the processes of absorption and emission, as shown in Fig.\,\ref{fig:silicon_absorption_emission}. The three levels involved are the ground state ($E_0$, electron in valence band, hole in conduction band), the first excited state ($E_1$, electron and hole bound to defect), and second excited state ($E_2$, electron in conduction band, hole in valence band). At room temperature, the two phonon mediated processes ($E_2$ $\rightarrow$ $E_1$ and $E_1$ $\rightarrow$ $E_2$) are both fast, with few-picosecond time constants (check Davies). The electric-dipole transition ($E_1$ $\rightarrow$ $E_0$) is comparatively slower, with nanosecond to millisecond transitions depending on the specific defect \cite{}. In detection, the dipole transition ($E_0$ $\rightarrow$ $E_1$) is pumped by the signal to be detected, and the excited electron-hole pair quickly transitions from $E_1$ to $E_2$, where a reverse-bias field sweeps the carriers out of the junction, resulting in detection. By contrast, in the emission process one pumps the $E_0$ $\rightarrow$ $E_2$ transition (through electrical carrier injection in a $p-n$ junction), and the excited carriers quickly transition to $E_1$, but before they can make the slow transition from $E_1$ to $E_0$, they make the fast transition back from $E_1$ to $E_2$, and eventually recombine non-radiatively through a variety of pathways without making the slow, dipole transition required to generate light. Crucially for our story, this is not the case at low temperature. The $E_2$ $\rightarrow$ $E_1$ transition involves emission of a phonon, so it remains fast, while $E_1$ $\rightarrow$ $E_2$ involves absorption of a phonon. At liquid helium temperature (4.2\,K), the relevant phonon states have low occupation, and the rate of the optical transition from $E_1$ to $E_0$ can be faster than the rate of transition back to the band edge, making silicon light sources possible based on this mechanism when operating at the same temperature required to enable superconducting circuits based on Josephson junctions. 

In addition to these two approaches to light sources on silicon, a major effort has been undertaken in the last 15 years to achieve hybrid integration of III-V light sources on silicon. Process incompatibility and lattice mismatch make it difficult to grow III-V gain media directly on silicon. Independent processing of Si and III-V substrates followed by wafer bonding is being pursued, but contemporary CMOS is very comfortable at 300-mm-wafer scale, while III-V processing has stayed at 150\,mm or below. Many such subtleties and complexities of process and materials integration have limited hybrid system performance and kept costs high. Many of the challenges are practical rather than fundamental, but nevertheless place real limits on the technologies that are achieved.

\paragraph{Hybrid Materials Integration}

\paragraph{Systems with Off-Chip Sources}
Considerable work continues in the development of light sources on silicon. At present, many efforts are proceeding to demonstrate exciting systems on chip with optical communication based on external III-V lasers fiber-coupled to silicon optoelectronic processors. Such work began commercially with the founding of Luxtera in 2001 with the goal being to utilize integrated silicon photonics for network interconnects. More recently, the effort led by Stojanovic, Popovic, and Ram has led to the development of silicon photonic systems implemented in existing CMOS processes with zero changes to the process. This effort has demonstrated basic components, such as waveguides, filters and modulators in 45-nm \cite{orma2012,shor2014,meor2014} and 32-nm technology nodes \cite{}. This ``zero-change'' approach (initially funded by DARPA \cite{}) has matured to the point where all-optical communication with 11 wavelength-division multiplexed channels was used between a processor and DRAM \cite{suwa2015,suwa2017} in the same 45-nm silicon-on-insulator process that was used to create the IBM Power 7 processor (Watson, PlayStation 3). This feat represents a significant milestone in the technological trajectory connecting global photonic networks down to optoelectronic systems on a single chip, perhaps fulfilling Soref's vision of a superchip. In this work, the III-V light sources are external to the silicon chip with fiber coupling between. Some in the field contend this will remain the most tractable solution in the long term. 

Significant commercial interest has persisted in this field since the founding of Luxtera, including major efforts by Intel \cite{}, and continuing with start-ups spinning out of the zero-change work \cite{}. All of these efforts attempt to use light as a means to communicate digital signals between electronic processors, whether it be at scale of a single chip \cite{suwa2015}, a server rack \cite{} or a data center \cite{}. As has been the case in semiconductor electronics and superconductor electronics, the device and hardware infrastructure developed for digital information processing is now being explored for neuromorphic information processing. 

\vspace{3em}
Regarding the primary motivation for optical devices, Kogelnik wrote in 1981, ``...the available speed is almost limitless, and it will be a challenge to exploit this speed.'' \cite{ko1981}

\paragraph{Silicon Light Sources Work at Low Temperature}

\vspace{3em}
When using light for computing in addition to communication in neural systems, information is often encoded in the amplitude of the light, and neurons often integrate the optical signal over time. There are multiple reasons why these uses of light are problematic: 1) encoding information in light amplitude is inefficient because it requires higher light levels to achieve dynamic range and overcome shot noise; 2) source noise is convoluted with synaptic weight; 3) integrating light intensity as the dynamical variable internal to the neuron (membrane potential) is limited because it is very difficult to store light in a cavity for longer than a few hundred picoseconds; 4) inhibition is difficult because light intensity can only be reduced with an additional optical signal if the phase between the two signals is controlled, which is extremely difficult in large systems with many synaptic signals.

\vspace{3em}

%Machine Learning with Neuromorphic Photonics
\cite{fepe2019}

%Neuromorphic Photonic Integrated Circuits IEEE JSTQE
\cite{pena2018} 

%Toward fast neural computing using all-photonic phase change spiking neurons
\cite{chsa2018}

%Progress in neuromorphic photonics
\cite{fesh2017}

%Computing matrix inversion with optical networks
\cite{wuso2014}

%High-speed all-optical pattern recognition of dispersive fourier images through a photonic reservoir computing subsystem
\cite{mebo2015}

%Machine Learning with Neuromorphic Photonics
\cite{lipe2019}

%A silicon photonic modulator neuron
\cite{tafe2018}

%A high performance photonic pulse processing device
\cite{rokr2009}

%other

%Reconfigurable optical neuron based on photoelectret materials
\cite{mo2000}
-sandwich structure combining electro-optic medium and photoelectret
-exotic material/physics
-no feasible route to system scalability

%An all-optical neuron with sigmoid activation function
\cite{mots2019}
-WDM input and weighting scheme
-logistic sigmoid activation function realized with a ``deeply-saturated differentially-biased Semiconductor Optical Amplifier-Mach-Zehnder Interferometer (SOA-MZI) followed by a SOA-Cross-Gain-Modulation (XGM) gate.''
-the goal seems to be a very close fit with a sigmoid, no matter the cost

\vspace{3em}
Did you emphasize that the Stanford architecture intended to utilize incoherent LEDs, as did the first optical neural nets by Psaltis et al?