Computers are increasingly being used to aid in information-processing tasks that lack a well-defined mathematical construction and instead require the ability to assimilate information across a broad range of content categories and use that information to make decisions, sometimes with regard to ambiguous criteria. Digital computers were not invented to address these tasks, and attempts to adapt digital hardware for applications requiring general intelligence point to opportunities for significant performance improvements if new hardware can be devised. These opportunities have led to a Cambrian explosion of many types of devices and approaches to beyond-CMOS hardware for artificial intelligence, predominantly in the domain of neural computing. In this article I review the history of both computing and neuroscience in order to establish the context for the present burgeoning of new hardware evolution, and I summarize past and present efforts in neural hardware with particular attention to semiconducting, superconducting, and photonic neural systems. I argue that a hardware platform combining the strengths of light for communication with superconducting electronics for computation has unique promise to embody the principles of neural information in systems of very large scale. Further, I argue that a superconducting optoelectronic network may serve as a central cognitive hub in an intelligent system employing specialized modules based on semiconductors, superconductors, or photonics and performing specialized computations such as numerical analysis, deep learning, or quantum information processing. The analysis indicates the physical possibility to achieve intelligent systems modeled after but potentially far exceeding the capabilities of the human brain.