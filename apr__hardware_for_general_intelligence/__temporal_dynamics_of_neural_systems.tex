\subsection{Temporal Dynamics of Neural Systems}
We emphasize that neural systems make use of space and time in a coordinated manner. We have described the spatial properties of neural systems in the preceding subsection, and here we describe the temporal dynamics. In the following subsection we describe how neural systems process information across space and time to enable cognition. 

Due to the complexity of neural information processing, multiple different perspectives have emerged that emphasize different aspects of observed phenomena. These perspectives are not necessarily mutually exclusive, and a complete understanding of neural systems may require all of these concepts. This situation is analogous to the parable of the blind men and the elephant, in which several individuals attempt to understand the full nature of an elephant with access to only a subset of the relevant physical data. This analogy leads us to identify the neural elephant, depicted in Fig.\,\ref{fig:neural_elephant}. This analogy illustrates that the challenge of understanding neural information processing at the scale of the human brain is challenging enough that multiple perspectives are required to grapple with the phenomena observed to date. Here I discuss three primary perspectives that have been brought to bear on the problem. These three perspectives are: 1) oscillations and synchrony; 2) the mathematical framework of dynamical systems; and 3) neuronal avalanches and criticality. All three of these perspectives emphasize the interplay between space and time, and in Sec.\,\ref{sec:cognition} we will argue that they are all related phenomena, but like different parts of the same neural elephant.
%\begin{figure} 
%    \centering{\includegraphics[width=8.6cm]{neural_elephant.pdf}}
%	\captionof{figure}{\label{fig:neural_elephant}Caption.}
%By Hanabusa Itchō - This is a retouched picture, which means that it has been digitally altered from its original version. Modifications: rotated, cleaned, cropped, centerline fold/crease removed (original Ukiyo-e print was on two pages), and scan fog haze removed., Public Domain, https://commons.wikimedia.org/w/index.php?curid=2265247
%\end{figure}

Before exploring these three perspectives on dynamical activity in neural systems, it is necessary to review basic behaviors at the device level. In the next subsection we describe neuronal device dynamics, including relaxation oscillations, neuronal communication mechanisms, synaptic plasticity mechanisms, and computations performed by synapses and dendrites. 

\subsubsection{Device Dynamics}
A neuron is a dynamical entity. It receives input from many afferent synapses, identifies coincidences and sequences between the activities on multiple synapses, and integrates various inputs over time. In a biological neuron, activity on a synapse results in a post-synaptic current into the receiving neuron. This current reduces the magnitude of the voltage across the neuron's cell membrane, usually with an exponential decay in time, and if that voltage is reduced below a threshold, the neuron produces an action potential, often referred to as a spike or pulse. This dynamical process of signal accumulation followed by bursting activity qualifies a neuron to be considered a relaxation oscillator. Before describing the temporal dynamics of neural systems in more detail, let us consider for a moment why relaxation oscillators are particularly well suited for cognition.

\paragraph{Relaxation Oscillators}
As we have mentioned, a defining aspect of cognitive systems is the ability to differentiate locally to create many sub-processors, but also to integrate the information from many small regions into a cohesive system, and to repeat this architecture across scales. A network of many dynamical nodes, each with the capability of operating at many frequencies, gives rise to a vast state space. As computational primitives that can enable such a dynamical system, oscillators are ideal candidates. In particular, relaxation oscillators \cite{st2015,mist1990,soko1993,lued1997,huya2000,bu2006,gile2011,vepe1968,cacl1981} with temporal dynamics on multiple time scales \cite{soko1993} have many attractive properties for neural computing, which is likely why the brain is constructed of such devices \cite{ll1988}. We define a relaxation oscillator as an element, circuit, or system that produces rapid surges of a physical quantity or signal as the result of a cycle of accumulation and discharge. Relaxation oscillators are energy efficient in that they generally experience a long quiescent period followed by a short burst of activity. Timing between these short pulses can be precisely defined and detected \cite{bu2006}. Relaxation oscillators can operate at many frequencies \cite{huya2000,more} and engage with myriad dynamical interactions \cite{lued1997}. The oscillator's response is tunable \cite{huya2000}, they are resilient to noise because their signals are effectively digital \cite{stgo2005}, and they can encode information in their mean oscillation frequency as well as in higher-order timing correlations \cite{pasc1999,thde2001,sase2001,stse2007,brcl2010,haah2015}.

\paragraph{Information Coding by Relaxation Oscillators}
The spiking nature of relaxation oscillators enables them to encode information in several ways. Most simply, they can encode information in their average firing rate, which gives rise to the standard expression used in deep learning and neural networks:
\begin{equation}
\label{eq:standard_neural_network_activation} 
y_i = f(\sum_{ij}w_{ij}y_j),
\end{equation}
where $y_i$ represents the ``activation'' of the $i$th neuron, $w_{ij}$ is the synaptic weight from neuron $j$ to neuron $i$, and $f(\cdot)$ is a nonlinear function representing the input-rate-to-output-rate transfer function of a simple point neuron. Equation \ref{eq:standard_neural_network_activation} is an extremely simple model of a neuron in the sense that only the average firing rate is considered, so no information regarding the precise times of spikes is retained. Further, $w_{ij}$ is assumed to be independent of frequency, so the model assumes synapses and dendrites perform no spectral filtering. Also, all input synapses are assumed to terminate on a single integrating body, assumed to be the soma, so no information regarding the spatial location of the synapse on the dendritic tree is retained. Nevertheless, Eq.\,\ref{eq:standard_neural_network_activation} has proven remarkably useful as a starting point for deep learning. 

Beyond the rate-coded, point neuron model, relaxation oscillators can also encode information through several other means. These include the time-to-first-spike after the onset of a stimulus, the phase of a spike relative to sinusoidal background oscillations, and the time interval between the firings of two or more neurons (see Sec. 4.5 in Ref.\,\cite{geki2002}). Thus, while information coding in digital computing is simple (bits are transmitted one at a time on a clock), information processing in neural systems with relaxation oscillators as computational primitives is complex. Neurons and networks do not speak a single, simple language, but rather send various types of spike-based messages with different information encoding and decoding in different contexts. We will explore these various strategies for coding in more detail shortly, but first we note one simplifying factor: the spikes that are used to communicate between neurons are binary. No information is conveyed in the amplitude of a single spike.

\paragraph{Binary communication and post-synaptic response}
We expect that any cognitive computing platform will be based on spiking neurons that behave as relaxation oscillators. Communication between these relaxation oscillators is effectively binary\textemdash all or nothing. When a neuron produces an action potential, it propagates down the axon and branches throughout the axonal arbor. The signal propagates as a section of depolarization between the interior of the axon and the surrounding extracellular fluid. This depolarization opens pores in the membrane of the axon, allowing the flow of ions from the extracellular fluid into the axon, thus providing the electrical signal that will reach the synapses. Each time the action potential is generated, the behavior is nearly identical: the speed of propagation of the signal is set by the physical properties of the axon; the number of pores that open is very large, so the signal propagating down the axon is not noisy; and the signal that reaches the synapses is very similar from pulse to pulse. Significant variability arises when an action potential meets a synapse, but this relates to the information processing occurring at the synapse (discussed in more detail below), not the nature of the encoding of the signal. 

To realize the digital nature of neuronal communication, the role of the action potential propagating down the axon is not to provide current to the post-synaptic neuron, but rather to begin a chemical cascade within the synapse that controls the post-synaptic signal amplitude. When the action potential reaches the synapse (pre-synaptic cleft), the action potential may trigger the release of neurotransmitters into the synaptic cleft. These neurotransmitters diffuse through the fluid of the cleft, and bind to receptors on the post-synaptic cleft. These receptors then trigger the flow of current through the dendrite on which the synapse resides. This post-synaptic current carries the information that will be processed, first by the dendrite, then the dendritic tree, and finally the soma. The action potential arriving at the synapse initiates the synaptic cascade in a binary, all-or-nothing manner (either vesicles are released from the pre-synaptic cleft or they are not), but the amount of current flowing into the post-synaptic dendrite depends on the state of the synapse, and it can take a continuum of values. Thus, communication in neural systems is binary, yet information processing is analog. The synapse performs a digital-to-analog conversion, and the state of the synapse (which depends on many factors) determines the analog value entering into the computation performed by the dendrites and soma. 

When a synapse receives an action potential, it has a certain probability of releasing vesicles containing neurotransmitters that will then be detected on the receptors of the post-synaptic cleft, generating a current that will propagate a certain distance and decay with a certain time constant. The time-course of the response is a continuous function, specifically an exponential decaying in space and time. The length and time constants depend on: the type of synapse; the morphology of the dendrite to which it is connected; the concentration of various neuromodulators; and the local membrane potential, which is determined by local synaptic activity. Thus, communication between neurons is well modeled by binary events that trigger a post-synaptic current with some probability $P_{\mathrm{s}}$, while the post-synaptic current is highly variable and shaped by a number of physiological and dynamical factors. We will further explore the relevance of this post-synaptic response in the coming paragraphs. 

\paragraph{The Neuron as a Complex Information Processing System}
For multiple reasons described above, we expect relaxation oscillators to be the computational primitives of complex cognitive systems. But amongst relaxation oscillators, neurons are unique in the myriad complexities. Our understanding of the information-processing capabilities of neurons has evolved considerably over time. Early models treated a neuron as a point that passively integrated many inputs and produced a signal upon reaching threshold. This integrate-and-fire model was proposed as early as 1906 by Sherrington in a series of lectures \cite{sh1906}, supporting the view championed by Ram\'{o}n y Cajal \cite{ra1908}. It was that same year that Ram\'{o}n y Cajal shared the Nobel Prize in Physiology or Medicine with Golgi for their work on the structure of the nervous system. Within this point-neuron view, a neuron simply integrates signals over time, so information regarding the timing or location of synaptic activity is lost. Further, the complex, dendritic tree is assumed to passively transmit synaptic signals to the neuron cell body. 

This view of the neuron as a point integrator was highly influential, and remains the dominant model guiding many efforts in deep learning and even neuromorphic hardware. However, further experimental evidence as well as theoretical arguments have deepened our understanding of how neurons work and led to a picture of neural information processing that is far more subtle and powerful. 

The modern picture of neural information processing reveals that many operations in addition to integration are being performed, and that these computations are performed at synapses, through the dendritic tree, and in the neuron as a whole. Some of the additional operations now believed to be performed by neurons include: coincidence and sequence detection; nonlinear thresholding performed in dendrites; temporal filtering of pulse trains; and bursting to overcome noise and enable selective communication between subsets of neurons in an ensemble. We now review each of these concepts and explain their significance for neural information processing.

\paragraph{Coincidence and Sequence Detection in Active Dendrites}
As discussed above, after a synapse is triggered and a post-synaptic response is induced, the time-course of the post-synaptic response is a decaying exponential with a time constant that can be shaped by a number of factors. Whether a neuron simply integrates signals or detects coincidences depends on this time constant of the post-synaptic potential relative to the average neuronal interspike interval. If the time constant is on the order of the interspike interval of the neuron, temporal integration is performed during the entire interspike interval, and the neuron is well-modeled as an integrator. If the time constant is short compared to the interspike interval, neurons perform coincidence detection, meaning only spikes that are coincident on the neuron within a short time window relative to the interspike interval can induce the neuron to produce an action potential. If integration is performed exclusively, then the timing of spikes conveys no information. If coincidence detection occurs, information is conveyed in spike timing, neurons respond preferentially to synchronized inputs, and a neuron's output reflects the input pattern.

Work in the 1990s led K\"{o}nig, Engel, and Singer to argue that cortical neurons are better modeled as coincidence detectors than integrators \cite{koen1996}. Their arguments were based both on the physiological evidence that many synapses can have time constants much shorter than the average interspike interval as well as theoretical arguments related to information processing. Regarding the latter, multiple benefits can be identified \cite{sase2000,sase2001}. First, speed of response can be improved if coincidences can be utilized. Processing speed is determined by the latency between the time of arrival of a signal and the time of a generated response. As stated in Ref.\,\onlinecite{koen1996}, ``Because only a small subset of all afferent [post-synaptic potentials] are relevant (namely those which actually conicide and trigger an action potential), the mean time-lag between relevant input and output signals is very short\textemdash only a fraction of the interspike interval. Thus, at identical interspike intervals neuronal systems utilizing coincidence detection can process information much faster.'' Second, coincidence detection provides benefits with regard to error propagation and noise. Errors due to stochastic processes in the environment are effectively filtered out by coincidence detection, as they would only contribute to neuronal firing rarely when they coincide with true signal, whereas neurons performing integration sum all incoming activity uniformly, regardless of correlations that may serve to isolate signal from noise. Third, the use of coincidence detection allows neural systems to make use of smaller ensembles of neurons for encoding information, referred to as the size of the `grain' in Ref.\,\onlinecite{koen1996}. If coincidences give rise to larger post-synaptic signals, then synchronized activity of a small group of neurons can have a large effect and drive a neuron to spike. Finally, the main strength of coincidence detection temporal information is not discarded, and synchronized inputs are processed differently than asynchronized inputs on the same set of synapses. The ability to respond preferentially to synchronized inputs has important consequences for information integration and binding, as discussed below in Sec.\,\ref{sec:oscillations_and_synchrony}. If neurons can detect coincidences, several important computation become possible. 

Work elucidating the behavior of dendrites further supports the view that neurons make use of the timing of synaptic activity as well as other intricacies beyond simple integration \cite{stsp2015}. As stated by Koch in 1997, ``...dendrites do much more than simply convey synaptic inputs to the cell for linear summation. Indeed, if this were all they did, it is not obvious why dendrites would be needed at all; neurons could be spherical in shape and large enough to accommodate all the synaptic inputs directly onto their cell bodies. ...the function of this elaborate structure cannot simply be to maximize the surface area for synaptic contact.'' \cite{ko1997} Instead of acting as passive transmission lines, the active properties of dendrites \cite{joma1996} give rise to the ability to detect coincidences. The basic mechanism relies on the nonlinearity of the response of a dendrite. If two synapses are located in close proximity on a dendrite, the response of the dendrite is a nonlinear function of the activities of the two synapses. Whereas the response of a passive dendrite would be the sum of the activities of the two synapses, the true dendritic response is closer to the product of the activities of the two synapses \cite{ko1997}. Further, dendrites themselves can generate spikes upon reaching threshold \cite{hoko2006}, a behavior once thought only to occur in the neuron as a whole. Again considering the example of two synapses on a dendrite, if only one of the synapses fires within the decay time of the post-synaptic potential, it may be insufficient to generate a dendritic spike, and the neuron is not informed of the synaptic activity. However, if both synapses fire within the decay time it may be sufficient to induce a dendritic spike, and the information is passed along the dendritic tree toward the soma. By 2006, the physiological mechanisms of dendritic spikes were more clearly understood \cite{hoko2006}, providing the device-level mechanism to support the coincidence-based information processing described by K\"{o}nig, Engel, and Singer in 1996. The nonlinear response of a dendrite with two synapses can also be compared to a logical AND operation: if synapse one AND synapse two fire within a certain time window, an output pulse is produced. 

The processing capabilities of dendrites extend beyond detection of coincidences between two synapses. Consider now $n$ synapses connected to a single dendrite. Now the dendrite will perform a nonlinear function on the inputs of the $n$ synapses, leading to an intermediate nonlinear transfer function between the synapses and the neuron as a whole. A neuron comprising many such dendrites will behave as a complex processor comprising multiple independent threshold units, and giving rise to myriad different transfer functions depending upon precisely which afferent synapses are active at a given time, just as observed in recent experiments \cite{sava2017}. Further, due to the morphology of a given dendrite, the post-synaptic current will often flow in a particular direction, usually toward the cell body. Consider the case where synapse one makes contact furthest from the cell body and synapse $n$ makes contact closest to the cell body. In this case, firing of synapse one will lead to a post-synaptic current that flows toward synapse two, thereby lowering the local membrane potential at synapse two. If a synaptic firing event at synapse one is followed closely by a firing event at synapse two, dendritic current is more likely to propagate on to synapse three, thereby lowering the membrane potential there. The pattern extend to synapse $n$, and this nonlinear signal propagation along the dendrite provides a means for the dendrite to detect specific sequences of activity: if synapse one fires just before synapse two, and synapse two is just before three, etc., then the dendrite is far more likely to produce a spike that propagates to the cell body than if the synapses fired in a different order. This phenomenon is illustrated schematically in Fig.\,\ref{fig:dendritic_sequence_detection}. The ability of neurons to make use of such sequence detection was developed in 2016 by Hawkins and Ahmad \cite{haah2015}. (say a bit more here) Interestingly, humans appear to be the species with the most elaborate dendritic trees \cite{el2003}.

\vspace{3em}
The total potential generated along a dendrite ``depends therefore on the temporal order of the stimulation of the synapses. An input sequence starting at the far end of the dendrite and approaching the soma is more effective in triggering an output spike than the same number of input spikes in reverse order.'' (\cite{geki2002}, pg. 144.)

%\begin{figure} 
%    \centering{\includegraphics[width=8.6cm]{dendritic_sequence_detection.pdf}}
%	\captionof{figure}{\label{fig:dendritic_sequence_detection}Caption.}
%\end{figure}

I will discuss dendrites further in the context of learning and plasticity shortly. First I review work related to communication between neurons through bursting.

\vspace{3em}
For detection of temporal information in the brain, cite review article from 1993 \cite{ca1993}. You have not read this article, so be careful.

\paragraph{Communication in Bursts}
If neurons encoded information only in their average firing rate, we would expect most neurons to fire relatively consistently, with the exact value of the rate dynamically varying depending on stimulus. By contrast, if neurons encoded information only in spike timing and coincidences, we would expect them to fire single action potentials when presented with relevant stimuli. In practice, these two behaviors are both observed, but a more common mode of activity is the burst. A neuronal burst is a closely spaced sequence of spikes followed by a long quiescent period. 

%bursting for reliability
Bursting activity is thought to play a central role in communication between neurons, perhaps as important as rate or other forms of temporal coding. One function of communication via bursts was appreciated by the 1990s, and that is the increased reliability of communicating with bursts rather than single spikes \cite{li1997}. A single synapse can be unreliable at producing a post-synaptic current in response to the arrival of a single action potential. On the device level, this is because vesicle release by the pre-synaptic cleft does not occur deterministically, but rather with probability $P_{\mathrm{s}}$, which varies considerably across synapses, and can be less than 0.1. However, the arrival of a number of action potentials in close succession performs a facilitating function that increases the probability of vesicle release with the arrival of each successive pulse. While the synaptic response to a single action potential can be unreliable, the response to a burst is much more reliable, increasing $P_{\mathrm{s}}$ above 0.9 simply by using two pulses instead of one in some observations \cite{fomc1991,degu1996}, a phenomenon known as paired-pulse facilitation \cite{li1997}. 

This reasoning of bursts being more reliable than single spikes is intuitive, but it may lead one to suspect that bursts effectively make communication analog in the sense that the generated post-synaptic current will depend on the number of spikes in the burst. Lisman has argued based on physiological evidence that this is not the case. The nuances of vesicle release and neurotransmitter detection are such that the post-synaptic current is nearly identical when receiving a burst, independent of the number of spikes in the burst (within a workable range) \cite{li1997}, thereby enabling bursting to overcome synaptic failure while maintaining binary communication.

Within this picture of neuronal bursting, the low probability of synaptic response becomes advantageous for filtering noise. As Lisman argues, in some brain regions, single spikes are simply the result of noise, but bursts are rarely noise. Therefore, if a synapse responds only to bursts of two or more pulses, it effectively ignores the noise, and responds only to the signal. For example, place cells in the hippocampus \cite{frbu2018} fire preferentially when the organism is in a specific location for which that neuron's tuning curve is maximized. It has been shown that responding only to bursts can define the place field more accurately than when single spikes are considered \cite{otei1991}. Similarly in visual cortex, it is well-known that neurons can have a tuning curve maximized to detect gratings with a specific spatial frequency or orientation \cite{daab2001}. It has been shown that only the bursting component of a neuron's activity is tuned to represent these quantities, while isolated spiking activity only correlates with the contrast of the light and dark regions of the grating \cite{cama1981}. These ideas lead Lisman to conclude ``that the ability of synapses to not transmit single spikes might be a crucial form of filtering.'' \cite{li1997} 

%bursting for frequency-selective communication
The concept that neurons fire in bursts to overcome unreliable synapses and filter noise is well supported by theoretical and experimental evidence. Yet these may not be the only reasons for communication in bursts. Izhikevich et al. have argued another principle is also in play: ``bursts with specific resonant interspike frequencies are more likely to cause a postsynaptic cell to fire than are bursts with higher or lower frequencies. '' \cite{izde2003} Within this model, neurons firing in bursts are like radio transmitters, and some synapses are tuned to receive spike trains of the same frequency. This tuning of synaptic and neuronal responses can be achieved through a combination of high- and low-pass filtering as well as subthreshold oscillations of the membrane potential. These device mechanisms for frequency selectivity are discussed below in Secs.\,\ref{sec:short_term_plasticity} and \ref{sec:subthreshold_membrane_oscillations}. 

The most exciting aspect of this use of bursting is that ``the same burst could resonate for some synapses or cells and not resonate for others,'' thereby providing ``effective mechanisms for selective communication between neurons.'' Such a method of targeted communication is one of many examples of how neurons can utilize one vast structural network to enable myriad functional networks at different times in different contexts. Izhikevich et al. argue that communication via bursts enables targeted, resonant communication \textit{in addition} to making communication more reliable and filtering noise. Beyond these motivations related to improvements in communication, bursts may also be more effective at modifying synaptic efficacy. I discuss the role of bursting in long-term synaptic modification below in Sec.\,\ref{sec:long_term_plasticity}.

\paragraph{\label{sec:coding_strategies_summary}Summary of Coding Strategies}
We have discussed rate coding, coincidence detection, sequence detection, and bursting. One may ask the question, ``Which of these is actually used by the brain?'' The answer is all of them. In some contexts, the average firing rate of a neuron may represent the content of a stimulus. In other contexts, the precise timing between two synaptic events or the order in which multiple synapse events occurs carries information that is utilized by the receiving neuron. In still other contexts, bursts are used to increase reliability, filter noise, and the frequency of pulses within a burst can enable selective communication between resonant elements of the network. All of these coding strategies and communication mechanisms contribute to the extraordinary efficiency and adaptability of biological neural systems, and we should aspire to incorporate all of these principles in the design of hardware for general intelligence. 

We now summarize the mechanisms at the device level that enable these subtle and sophisticated mechanisms for information processing.

\paragraph{\label{sec:short_term_plasticity}Short-Term Synaptic Plasticity}
As described above, the ability of a synapse to respond preferentially to activity at a specific frequency enables additional complexities in communication. At the device level, frequency-selective synaptic response is enabled by filtering mechanisms referred to as short-term synaptic plasticity. 
%Fig. 1 of \cite{izde2003}
%\begin{figure} 
%    \centering{\includegraphics[width=8.6cm]{short-term_plasticity.pdf}}
%	\captionof{figure}{\label{fig:short-term_plasticity}Caption.}
%\end{figure}
The state of a synapse is affected by its activity over short and long time scales as well as external network factors. Neurons often signal in bursts (closely spaced sequences of spikes) \cite{iz2007}, and within a burst, the time between spikes is referred to as the inter-spike interval. Changes of synaptic response over time scales on the order of the inter-spike interval are referred to as short-term synaptic plasticity \cite{abre2004}. One key effect of short-term plasticity is to perform a temporal filter on an afferent spike train. This can be a high-pass, low-pass, or band-pass filter, as shown in Fig.\,\ref{fig:short-term_plasticity}. High-pass filtering results in only the first few pulses of a train being transmitted from the pre-synaptic axon to the post-synaptic dendrite. A synapse performing high-pass filtering reports to the post-synaptic neuron that the pre-synaptic neuron has begun to fire. Conversely, low-pass filtering results in synaptic response after the first several pulses of a train have occurred. A synapse performing low-pass filtering will not be active unless the pre-synaptic neuron produces a pulse train of a certain minimum duration, and therefore this synapse reports to the post-synaptic neuron when the pre-synaptic neuron has sustained bursting activity beyond a certain duration. Band-pass filtering combines these responses. A synapse performing band-pass filtering will only produce a response after an afferent pulse train exceeds a certain duration, and it will fall silent again after if the afferent pulse train continues beyond a certain duration.

These short-term filtering mechanisms enable synapses to report much more information to the neuron than simply the time-averaged rate of afferent activity. A neuron combining the signals from many synapses with various short-term responses has access to information regarding not just the average spike rates of the neurons from which it receives synaptic connections, but also regarding the initialization of bursting and the duration of spike trains.


\paragraph{\label{sec:subthreshold_membrane_oscillations}Subthreshold Membrane Oscillations}

\paragraph{\label{sec:long_term_plasticity}Long-Term Synaptic Plasticity}
%\begin{figure} 
%    \centering{\includegraphics[width=8.6cm]{stdp.pdf}}
%	\captionof{figure}{\label{fig:stdp}Caption.}
%\end{figure}
Over time periods much longer than the inter-spike interval, the response of synapses can also change based on the activity of the two neurons involved in the synapse. A synapse that is more active will strengthen (long-term potentiation), and a synapse that is used less will weaken (long-term depression). This was the essential insight of Hebb in 1949 \cite{he1949}, a concept that developed in the subsequent decades \cite{bipo1998,somi2000} to account for the fact that long-term potentiation only occurs if the pre-synaptic neuron fires just before the post-synaptic neuron, indicating the potential for causality, while long-term depression is induced when the pre-synaptic neuron fires just after the post-synaptic neuron. This spike-timing-dependent plasticity (STDP) \cite{mage2012} plays a central role in memory formation and network adaptation. 

The physical mechanism responsible for STDP involves the growth and decay of neurotransmitter sources and receptors present at the synapse. These synaptic molecular machines (N-methyl-D-aspartate receptors, NMDARs) develop in response to the action potential arriving at the pre-synaptic terminal as well as the back-propagating signal from the post-synaptic neuron. The complex chemistry present at each synapse leads to a remarkable degree of diversity and adaptability in synaptic response.

By strengthening cooperative synapses, STDP adapts the structural network of neurons and their connections into functional networks embodying certain memories or computations learned over time based on the correlations of neuronal firing events. It has been shown that random networks with synaptic weights adjusted over time by STDP adapt into small-world networks \cite{shki2006}, maintaining efficient communication, and adding functional clusters specialized for specific computations. This is one example of the structural network of a neural system can be used to manifest multiple functional networks.

The functional clusters established via STDP have spectral signatures. A given functional cluster of neurons will have a specific pattern of activity, that may repeat in time. The period of this repetition will depend on the specific parameters of the circuit, and a large structural network comprising many highly connected neurons will have the potential to establish a vast repertoire of functional clusters with oscillations at many frequencies. Through STDP, the network can increase the activity of certain oscillations corresponding to highly utilized functional modules. If a certain stimulus has a probability of activating a certain functional cluster, the overtime, the action of STDP will enable the network to correlate that stimulus with the dynamical response of the cluster, and the stimulus will evoke the activity of the cluster with higher probability. In the language of dynamical systems, a specific cyclical response of the functional cluster is referred to as a basin of attraction \cite{iz2007,st2015}, and STDP ensures that relevant stimuli lead the network to the appropriate basin of attraction. This is function of an autoassociator, and it is an important form of long-term memory (\cite{bu2006} pg. 329).

Spike-timing-dependent plasticity makes use of correlations between firing activity of neurons to adapt the network into functional clusters \cite{shki2006}, store memories in dynamical sequences \cite{haah2015}, and strengthen circuits that demonstrate temporal patterns storing sequential memories (\cite{bu2006} pg. 318). But theoretical analysis finds that if STDP is the only long-term synaptic plasticity mechanism, memories are forgotten very quickly \cite{fuab2007}. Experimental analysis finds that synapses have multiple additional means to change how synapses adapt over time and activity to help retain memories \cite{ab2008}. These plasticity mechanisms are referred to collectively as metaplasticity.

%bursting for plasticity
It has been shown that a burst of action potentials comprising four pulses can be sufficient to induce long-term potentiation and depression \cite{huli1995}, dependent on the timing of the burst relative to theta oscillations. A burst arriving at the peak of theta leads to potentiation, while a burst arriving at the trough leads to depression. In this case, we find an interplay between neuronal bursting, synchronized oscillations, and synaptic plasticity. Adapting based on the confluence of all this information leads to efficient evolution toward a network structure that generates constructive activity.

\paragraph{Metaplasticity}
While STDP adapts the strength of synapses (synaptic efficacy), metaplasticity adapts the rate at which synaptic efficacy changes over time and activity. For example, if a pre-synaptic neuron fires just before a post-synaptic neuron, the synapse connecting the two will be a candidate to experience long-term potentiation. But the \textit{probability} that the synapse actually does potentiate can be controlled by chemical signals within the synapse. Additionally, the \textit{amount} by which the synaptic efficacy changes is also subject to chemical modulation. 

The function of metaplasticity is to control which neural circuits adapt at a given time \cite{ab2008}. The receptors mentioned above (NMDARs) can be controlled based on a variety of factors related to network activity so that adaptability may be turned on and off in certain regions at certain times. This functionality is required of plastic synapses to keep them from too quickly losing the trace of a memory that is still needed. Experiments with humans indicate that forgetting occurs as a power-law function of time, \cite{wieb1991,wieb1997}, yet Fusi and Abbott have shown that memories are lost more rapidly than this if plastic synapses are presented with continual stimulus \cite{fuab2007}. They have proposed a model that achieves the observed power-law forgetting by introducing internal complexity to the synapses \cite{fudr2005}. In this model, each synapse has various states of efficacy (weak and strong synaptic weight), but it also has additional internal states with the same efficacy. The difference between these states is the probability with which the efficacy will change due to future plasticity events. 

Metaplasticity provides a network with the means to to enable some regions to adapt at a given time, under a given stimulus, while other regions are unchanged at that time, under that stimulus. Further, metaplasticity provides a means by which some synapses within a region may change very rapidly to adapt to a new stimulus, while other synapses in the same region may change slowly or not at all when presented with the same stimulus. We expect that an intelligent neural system have the capability to immediately learn in response to new information, but also to maintain a lasting representation of all that has been learned through the network's existence even in the presence of continually varying input. Metaplasticity is an important means by which rapid learning in conjunction with long memory retention can be achieved. As stated by Abraham, ``...these metaplasticity processes represent a major form of adaptation that helps to keep synaptic efficacy within a dynamic range and larger neural networks in the appropriate state for learning.''

\paragraph{Homeostatic Plasticity}
To conclude this discussion of synaptic plasticity mechanisms, we note that short-term plasticity adapts based only on the activity of the pre-synaptic neuron, while STDP adapts based on correlations in the activity between pre- and post-synaptic neurons. The mechanism of homeostatic plasticity \cite{cube2012} adapts synaptic weights based only on the activity of the post-synaptic neuron. Homeostatic plasticity (also referred to as the Bienenstock-Cooper-Munro (BCM) model \cite{bico1982}) adjusts the synaptic weights of synapses incident upon a given neuron based on a sliding temporal average of the recent firing activity of that neuron. Such a mechanism provides a means by which neuron and network activity can be maintained within useful limits and dynamic range can be maximized.

\paragraph{\label{sec:dendritic_processing}Dendritic Processing}
We have discussed how STDP leads to the formation of functional clusters within a network based on the history of correlated neuronal activity. But what if the network wishes to isolate specific functional clusters on time scales as short as an inter-spike interval? Or what if we wish to endow a neuron with the ability to respond not only to activity in single synapses, but rather to integrated activity from specific clusters of synapses, or to specific sequences of activity within a cluster of synapses? Dendritic processing enables these functions.

Dendritic processing refers to the intermediate, nonlinear transfer functions performed by dendrites between individual synapses and the neuron as a whole \cite{stsp2015}. The dendritic arbor is a complex, branching structure on which most of a neurons input synapses make their connections. While the dendritic tree was thought to be a passive integrating structure for quite some time, by the mid 1990s many activite dendritic responses were beginning to be understood \cite{joma1996}. The dendritic branches that comprise the arbor have passive and active properties that allow them to perform various computations. Dendrites modulate postsynaptic potentials on their way to the soma as well as generate spike activity \cite{majo2005,hoko2006}. One picture of the dendritic arbor is a network of multiple independent threshold units that integrate signals and produce dendritic spikes upon reaching threshold. This picture was introduced on theoretical grounds based on the enhanced storage capacity of such a dendritic tree in 2001 \cite{pome2001}, and supporting experimental evidence continues to accure \cite{sava2017}.

For example, consider a dendrite with two synapses. If the post-synaptic current into the dendrite is sufficient to produce a dendritic spike, and if this dendritic spike has the same form whether one or both synapses fire, the dendrite performs the OR operation. If both synapses are required to fire to produce a dendritic spike, it performs the AND operation. The current generated in a dendritic spike propagates only a short distance along the dendritic tree, into the next dendritic compartment closer to the cell body, and the current decays with an exponential time constant. Thus, dendrites can perform basic logical operations with a temporal component, and activity closer to the base of the dendritic tree at the cell body integrates information from a larger number of inputs. 

Like many aspects of neural information processing, the myriad roles of dendrites remind us that the devices giving rise to cognition are diverse and multifunctional. Mel, Schiller, and Poirazi emphasize that ``dendrites of different neuron types contribute to the cell's input-output function in markedly different ways.'' \cite{mesc2017} While all of the aforementioned roles of dendrites are observed in biological neural systems, they are not necessarily all present in every dendrite. In the context of designing hardware for neural information processing, these functions must be available, and sophisticated analysis at the level of networks and systems must inform our decisions of when to include each mechanism.

\cite{pome2001}


\paragraph{\label{sec:dendrites_and_plasticity}Dendrites and Plasticity}
In addition to performing nonlinear transfer functions on inputs, dendrites also play a key role in synaptic update and learning. This activity is induced by back-propagating action potentials created during the firing of the neuron \cite{majo2005}. These back-propagating action potentials provide a feedback signal from the neuron to the dendrites and synapses after neuronal firing that performs one of the steps in a timing-dependent plasticity process \cite{nesa2004}.

In addition to back-propagating action potentials playing a role in STDP, dendritic spikes are also involved in forms of synaptic plasticity that do not require synaptic activity. In these forms of synaptic plasticity, it is the timing of the forward-propagating dendritic spikes and backward-propagating spikes that establish the conditions in which synaptic weights can be modified \cite{hoko2004}. This local activity can lead to long-term synaptic potentiation or depression and may be the primary mechanism for rapidly acquiring memories \cite{hoko2006}.

We have mentioned that dendrites are capable of acting as integrators or coincidence detectors depending on the time constants involved. Additionally, a single dendrite can transition between these two roles based on the interplay between synaptic activity and the neuromodulatory environment \cite{hoko2006}. The presence of certain neuromodulators affects the behavior of voltage-gated ion channels, thereby shaping the short-term response of dendrites \cite{majo2005}. As stated by Holthoff et al., ``...the alliance of non-linear and linear integration modes combines the advantages of the excellent signal-to-noise ratio of digital processing with teh speed and complexity of analog processing.'' \cite{hoko2006} 

We have also mentioned that back-propagating action potentials in dendrites play a central role in STDP. The active properties of dendrites can lead to metaplasticity as well \cite{ab2008}. The mechanisms involved again relate to the interactions between synaptic inputs and neuromodulators. The interplay between neuromodulators and synaptic activity affects the time constants that shape the dendritic membrane potential as well as the probability that potentiation or depression will occur in the presence of synaptic input. 

This discussion of dendritic processing is necessarily brief, emphasizing the active role played by dendrites in adaptive and frequency-selective response to spike trains based on voltage-dependent conductances; the nonlinear thresholding properties of dendrites exemplified by the generation of dendritic spikes; and the role of the dendritic tree and back-propagating action potentials in synaptic plasticity. Further detail can be found in the comprehensive review by Sj\"{o}str\"{o}m et al. \cite{sjra2008} From a computational perspective, the dendritic tree is the main information-processing infrastructure of a neuron. Computations are performed as fan-in occurs. Whereas in many computational systems, fan-in is passive and represents part of the communication infrastructure, in neural systems fan-in and computation occur in the same physical structures. This is in contrast with the axonal arbor, which performs fan-out to many synapses, but the axons themselves do not modify information, but rather simply serve to communicate spikes from one neuron to many others. 

\paragraph{Summary of Neural Device Dynamics}
This subsection summarized literature supporting the perspective that relaxation oscillators are uniquely suited to serve as computational primitives in cognitive systems due to the high signal-to-noise ratio of their outputs, their adaptive frequency response, and their ability to make use of many information coding schemes. These coding schemes have been described. The importance of binary communication for mitigating noise was discussed, and multiple communication schemes were explained. These modes of communication include: average rate; coincidence and sequence; phase relative to background oscillations; and bursting for reliability as well as frequency selectivity. The device-level mechanisms that enable these functions were summarized, including short- and long-term synaptic plasticity and active dendritic functions. We now turn our attention to activity emerging from populations of neurons.

\subsubsection{\label{sec:populations}From devices to populations}
We have described some of the basic device operations of neurons and their components, but neural information processing leverages entire populations of neurons with emergent mechanisms for information processing occurring in a network context. As a first example, consider a population of uncorrelated, non-interacting neurons receiving identical stimulus and firing asynchronously. The population activity is defined as the sum over all firings by all neurons, and in the presence of a constant stimulus, the population will fire with a fixed population activity. For a large population of $N\longrightarrow\infty$ neurons, at a given moment there will be some neurons  at all stages of the cycle of integration and firing. If the stimulus is instantaneously increased by a finite amount (perhaps due to the input from another neuron firing), the population activity will instantaneously increase. This instantaneous response is possible because some neurons in the population will be close to threshold, and the finite perturbation pushes them over threshold. Thus, they fire sooner than they would have in the absence of the perturbation, and the population activity increases. Such a response is not obtained, in general, from a single neuron. If a single neuron is perturbed instantaneously, its response will depend on its state before the perturbation. If it is in a refractory period or far from threshold, the perturbation may be insufficient to evoke a response at all, or it may take some time before the effect of the perturbation is observed.

This rapid population response is most clearly observable in the absence of noise, and in the presence of noise, populations perform useful functions as well. For example, one can calculate the signal-to-noise (SNR) ratio of a group of neurons attempting to code a given signal. It is found the SNR increases linearly with the number of neurons in the population (\cite{geki2002}, chapter 7.3.2). Populations of neurons can increase response time and mitigate noise. These two responses do not require interactions between the neurons in the population, yet dynamics become more interesting when the neurons of a population are connected with various graph structures.

\paragraph{\label{sec:oscillations_and_synchrony}Oscillations and synchrony}
When observing brain activity at a coarse scale, such as with electroencephalograms, sinusoidal oscillations in activity are observed. Finer measurements with electrodes also observe this phenomenon on shorter length scales. Oscillations in mammalian cortical neurons are observed across five orders of magnitude in frequency, from 0.05\,Hz to 500\,Hz \cite{budr2004}, and the types of oscillations observed depend on behavior and input stimulus. In rats, 10 frequency bands are observed, and their center frequencies are separated logarithmically, creating what Buzs\'{a}ki refers to as ``a system of brain oscillators''. These different bands are associated with different brain states and different behaviors. At a given moment, activity may be present in several frequency bands, and these activities may compete or interact with each other.   This seemingly simple behavior leads to many layers of complexity with important ramifications for neural system operation \cite{bu2006}. Each neuron is a relaxation oscillator, characterized by spikes that are short in time relative to the inter-spike interval. A single neuron is not well modeled by a sinusoidal time dependence, but the net activity of a population of neurons often takes a sinusoidal form. In neural systems, the devices are relaxation oscillators, while the networks can behave as harmonic oscillators. 

When referring to oscillations in a network of neurons, we are referring to this sinusoidal population activity. The net activity has a well defined phase, and it becomes meaningful to speak of the phase of firing of a single neuron relative to the phase of the sinusoidal population activity. In some cases information is encoded in the timing of a neuron pulse relative to the phase of the background oscillations (\cite{geki2002}, pg. 140).

A related but distinct phenomenon is synchrony. In populations of neurons with recurrent connections, asynchronous firing may or may not be stable depending on factors such as noise, signal transmission time, and rise time of the post-synaptic potential (\cite{geki2002}, Ch.\,8). In the low-noise limit, asynchronous firing is unstable across a wide range of parameter space, and the network will invariably converge toward synchronized firing of action potentials. For neurons sufficiently close in space that communication delays can be neglected, the frequency of synchronized activity is determined by device-level properties, such as membrane time constants. For large groups of neurons sufficiently separated in space so that transmission delays are non-negligible, neurons will naturally separate into clusters, each synchronized internally, but unsynchronized globally. Thus, the net population activity of these cluster states can be at frequencies higher than any one cluster can fire alone. These so-called cluster states have no relation to the identically named quantum states being investigated for their role in quantum computing. Synchrony is quantified through a cross-correlation function of activity taken from different neurons or brain regions.

A neuron is sometimes referred to as a single-cell oscillator, referring to the fact that it may be observed to pulse at a well-defined frequency. This oscillation may be self-induced (autorhythmic) based on the interplay of Na$^+$, K$^+$, and Ca$+$ following an action potential \cite{ll1988}. Oscillation may also be observed due to a driving super-threshold stimulus based on the refractory period of the neuron. A neuron may also be referred to as a single-cell resonator, meaning it responds preferentially to input stimulus within a given frequency range \cite{ll1988}. To achieve a resonant response, the neuron must have both a high-pass response and a low-pass response to generate a band-pass filter \cite{huya2000}, as discussed above in the context of short-term plasticity. Short-term plasticity leads to frequency-dependent synaptic efficacy \cite{rahu2008}. Low-pass response is achieved by parallel leak conductance and capacitance of the membrane that shunts responses at high frequencies. Similarly, there are voltage-gated conduction pathways in the membrane that act as high-pass filters. Nature has engineered the time constants of these responses so that band pass is possible. Further, the low-pass and high-pass cutoff frequencies can be adjusted based on the membrane voltage or the presence of neuromodulators \cite{huya2000,tsma2014}. Voltage tuning of the frequency response of a given synapse may occur based on the activity of nearby synapses, while neuromodulatory adjustments occur over broader areas, usually containing multiple neurons. Resonance in neurons is quantified by measuring the impedance of a single neuron as the driving stimulus is swept across frequencies. It can also be modeled theoretically at the single-neuron level (\cite{geki2002}, pg. 81) and at the population level (\cite{geki2002}, pg. 233). A key result of this line of research is that the frequencies of various oscillations in the brain depend on the structure of the network, but they also depend on the intrinsic response properties of the devices, which dynamically adjust based on network activity \cite{majo2005,tsma2014}. Synapses and neurons are tuned to respond maximally at frequencies that facilitate communication on different spatial and temporal scales. The interplay between network and device contributions to oscillatory activity may be important. As stated by Hutcheon and Yarom in 2000, ``...network connectivity could reinforce the patterns of excitation produced by coupled oscillators.''

It has been proposed that the interplay of rhythmic and arhythmic brain activity is important to cognition. In this model, background oscillations of the network at various frequencies provide the context in which the activities of transient clusters occurs \cite{ll1988,others}. It has been observed that states of oscillation are disrupted upon directing attention to a sudden stimulus \cite{bu2006}, leading to the hypothesis that various network oscillations play a role analogous to the information processing context emphasized by Baars \cite{ba1988}. Oscillations are also thought to be essential to brain-wide information integration. In this context, the thalamus plays a central role. The oscillation frequency of thalamic neurons is tunable based on activity and neuromodulators, and depending on the mode of oscillation, the thalamus may be coordinating the exchange of information between different brain regions at theta frequencies, or it may be inducting sleep through alpha rhythms. Such observations led Baars to identify the thalamo-cortical complex as the anatomical construct responsible for brain-wide information dissemination in the Global Workspace model \cite{ba1988}. It is clear that distinct oscillations play distinct roles in the information processing of the brain. ``The low-frequency resonances in the cortex and thalamus appear suited to support the thalamocortical delta-wave oscillations that are particularly prominent during deep sleep. The higher-frequency oscillatory behavior and underlying resonance in pyramidal and inhibitory neurons of the neocortex might have some involvement with higher frequency rhythms that appear in the cortex during cognition.'' \cite{huya2000}

\vspace{3em}
``...establishment of even a weak resonance makes a neuron a good listener for activity within a specialized frequency band. A host of good listeners, mutually connected, should tune networks to operate in frequency ranges of special biological meaning.'' \cite{huya2000}

\vspace{3em}
Definition of synchrony from \cite{vala2001} box 2

\vspace{3em}
``Functionally, such resonances constrain neurons to respond most powerfully to inputs at biologically important frequencies such as those associated with brain rhythms.'' \cite{huya2000}

\vspace{3em}
``Brain rhythms reflect basic modes of dynamical organization in the brain.'' \cite{huya2000}

\vspace{3em}
Brain oscillations perform numerous important functions

\vspace{3em}
role of inhibition

\vspace{3em}
If all neurons in a network have the same refractory period, and all synapses have the same rise and decay time constants, then signal transmission delay due to the spatial organization of the network is the primary time constant that separates neurons into differentiated clusters. The maximum frequency at which two spatially separated regions can drive each other to synchronize depends on the distance and transmission velocity. This delay limits the scale of networks that can oscillate at a given frequency, constraining high-frequency activity to smaller regions of space. We consider this concept further in Sec.\,\ref{sec:scaling}. 

\vspace{3em}
Neurons in close proximity can drive each other to fire at high frequencies, and such clusters can exchange information across broader regions of the network through slower oscillations, allowing ``brain operations to be carried out simultaneously at multiple temporal and spatial scales.'' \cite{budr2004} This conception of how global activity can modulate and steer local activity is central information processing in the brain. 

\vspace{3em}
must demonstrate resonance in loop neurons in this paper

\vspace{3em}
The net dynamical activity of brain oscillations, synchronized clusters, and neuronal avalanches leads to a $1/f$ power spectral density, which implies that activity occurring at slow frequencies can cause a cascade of activity at higher frequencies \cite{budr2004}. Put another way, slow, sinusoidal activity across a wide region of the network can modulate the fast activity occurring between clusters locally.  

\vspace{3em}
role of oscillations in plasticity \cite{bu2006,budr2004,li1997}

\paragraph{\label{sec:dynamical_systems}The dynamical systems perspective}
We have discussed various forms of neuronal dynamics in neural systems, from the scale of single neurons up to the scale of avalanches and oscillations spanning the network. One prevalent theme is that all activity is transient. Small clusters of neurons form transient synchronized ensembles. Theta oscillations may sustain for some duration until the periodic activity is broken as attention is directed to a stimulus. It is common to think of short-term memories as being represented by dynamical activity stored in a stable attractor, such as in long short-term memory \cite{hosc1997}. However, activity in the brain is never truly cyclic as in a stable attractor, but rather transitions between quasi-stable states. These successions of states are trajectories in a dynamical state space, and a given stimulus can induce a specific trajectory associated with the activities of a population of neurons. Each trajectory is a sequence of successive metastable states \cite{rahu2008}, where a metastable state is a saddle point in state space, as opposed to an attractor which is a local minimum. This picture of neuronal dynamics treats transient activity mathematically as a stable heteroclinic channel \cite{ravo2001,hura2004}. As stated by Rabinovich et al., ``These saddles can be pictured as successive and temporary winners in a nonending competitive game.'' As discussed in multiple contexts thus far, a balance of excitation and inhibition is necessary to result in long-lived transient sequences that do not grow without bound. Unique sequences are triggered by unique inputs, leading to computations that are ``reproducible, robust against noise, and easily decoded.'' \cite{rahu2008}

\vspace{3em}
Reservoir computing
\vspace{3em}
learned/adapted patterns through plasticity, or reservoir without plasticity
\vspace{3em}
information stored in trajectories through phase space that involve sequences of neuronal firings; the dendritic tree can detect these sequences; a specific dendrite may fire preferentially only when a specific dynamic trajectory is excited by stimulus

\cite{iz2007}

\paragraph{\label{sec:criticality}Criticality and the fractal use of space and time}
The concept of self-organized criticality was intrduced in 1987 by Bak, Tang, and Wiesenfeld \cite{bata1987,bata1988,ba1996}. This work argued generally that dynamical systems with spatial degrees of freedom naturally evolved to a critical point with a highly ordered phase on one side and a highly ordered phase on the other side. Similar to general second-order phase transitions, this critical state is characterized by power-law decay of correlations across space and time. However, Bak et al. emphasize that self-organized criticality differs from phase transitions in equilibrium statistical mechanics. While phase transitions in statistical mechanics result from modifying a tuning parameter across a critical value, self-organized criticality emerges in dynamical systems far from equilibrium as the dynamical state converges to an attractor. Dynamical systems at the critical point are marked by $1/f$ power spectrum (sometimes referred to as flicker noise or crackling noise \cite{seda2001} even if the source is not noise at all), indicating correlations over a wide range of temporal scales. 

A $1/f$ power spectrum is characteristic of a fractal process, with self-similar activity observable across time scales \cite{sc1991}. Critical processes are often also marked by self-similar patterns in space, with long-range spatial correlations also identified by power-law decay, as discussed above in the context of Rentian scaling. Power laws have no characteristic scale\footnote{For example, if a system has a power spectral density $p(f)\sim f^{-\gamma}$, then the expectation value of frequency is $\langle p(f)\rangle \sim \int_0^{\infty}fp(f)df = \int_0^{\infty}f^{-(\gamma-1)}df\rightarrow \infty$, indicating that an average frequency cannot be defined. For real physical systems with a maximum operating frequency, this integral will converge. Nevertheless, the power-law distribution results from contributions across a broad range of scales, and it is generally not meaningful to speak of a characteristic frequency or length.} and are intimately related to fractals and indicate a balance between order and chaos. If correlations decay exponentially (in space or time) long-range interactions are impossible, and the system has simple dynamics. If correlations are constant across the system, a small change anywhere perturbs the system everywhere, and chaos results. Power-law distributed correlations mark the fruitful middle ground between these two extremes, giving rise to systems with complexity marked by fractal patterns of self-similarity across spatial and temporal scales.  

This concept of balance between order and chaos was explored in the context of computing by Langton in 1990 \cite{la1990}, where he used the term ``edge of chaos'' to refer to the region of parameter space most capable of information processing. Langton argued that computation requires transmission, storage, and modification of information. Information storage reduces the entropy of a system, while transmission increases the entropy. Because a computer must do both effectively, optimal performance is achieved by operating at a trade-off point between high and low entropy, which occurs in the vicinity of a phase transition. Making the analogy to solid/liquid transitions, solids are highly ordered, so information can be stored, but it cannot be communicated. Conversely, liquids are highly disordered, and information can propagate through a liquid, but the transmission process disrupts the state, and prior information is lost. Computing occurs on the boundary between the ordered solid phase and the disordered liquid phase, and at this operating point information can be transferred over long distances without attenuating. Langton argues that computation and phase transitions are fundamentally connected, an idea that has resonated significantly in the neuroscience community. We now review some of the literature related to criticality in neural systems.

\vspace{3em}
Also explore \onlinecite{sp2010} ch. 12 in this section 

\paragraph{Neuronal avalanches}
Evidence of neural system operation at a critical point can be obtained through temporal or spatial considerations. In the spatial domain, operation at the edge of chaos is observed in the distribution of sizes of sequences of activity. In a wide variety of slowly driven physical systems, the response is marked by discrete events of a variety of sizes \cite{seda2001} (earthquakes, magnets in magnetic fields). The elements these systems hav ein common is that they are ``...driven systems with many degrees of freedom, which respond to the driving in a series of discrete avalanches spanning a broad range of scales....'' \cite{seda2001} Neural systems meet these criteria and display this behavior.

If a population of spiking neurons is mutually interconnected, rich dynamics can result and be shaped by a multitude of factors. Consider a network of $N$ neurons with sparse, random connections, approximating cortex. In the presence of random input, perhaps from other regions, the neurons of the population will be observed to fire, seemingly stochastically. We can ask, if the population is observed for a duration $\Delta t$, what is the probability that $s$ neurons\footnote{Experiments with biological neurons observe local field potentials with spatial resolution more coarse than a single neuron, so the size $s$ is often related to the number of electrodes in an array recording signal during $\Delta t$, but here $s$ is taken to represent a well-defined number of neurons.} will fire during that duration? The result is a power law,
\begin{equation}
\label{eq:neuronal_avalanche_power_law}
p(s) \propto s^{-\alpha},
\end{equation}
with $\alpha \approx 1.5$ \cite{plth2006,be2007}. This relationship is not sensitive to the observation duration, $\Delta t$, and is observed in species from leeches \cite{be2007} to monkeys \cite{peth2009}. Equation \ref{eq:neuronal_avalanche_power_law} is supported by experimental data observing the number of electrodes activated during an observation period, and therefore regards the spatial organization of neuronal activity. An event involving the firing of $s$ neurons is referred to as a neuronal avalanche \cite{bepl2003} of size $s$, and Eq.\,\ref{eq:neuronal_avalanche_power_law} tells us that the ratio of the number of avalanches of size $ks$ to the number of size $s$ is $k^{-\alpha}$, independent of $s$. There is activity across all spatial scales. Local activity of a few neurons is most probable, but large avalanches including wide regions of the network are still statistically significant. Neuronal avalanches of power-law size distribution are another manifestation of fractal scaling in neural systems: ``the relationship of pattern sizes to each other is apparent of every scale.'' \cite{plth2006} 

This fractal organization of activity has important ramifications for system scaling, and therefore should be considered when designing hardware for cognition. Because the interactions of the neurons and the graph structure are such that activity patterns of all sizes can occur, the limit of neuronal avalanche size is not set by the devices, interactions, or dynamics, but rather by the physical structure. As stated by Plenz and Thiagarajan, ``...in a finite system such as the cortex, a power law [] indicates that the underlying dynamics are constrained by the size or physical borders of the system rather than by any intrinsic characteristic of the dynamics themselves.'' \cite{plth2006} The implication of this neuronal activity is that any given neuron or cluster can quickly engage any other region of the network and access information stored at those other sites. These patterns extend beyond single cortical columns, and the engagement of even distant regions can be fast \cite{plth2006}, provided the hardware employed does not introduce communication delays.

It is important that choices we make about hardware ensure this statement remains true of our artificial system. If our neurons can only achieve local connectivity, then network path lengths we become long, and the inability for a given neuron to activate distant neurons within will reduce the probability of large neuronal avalanches. Similarly, if a shared communication infrastructure is employed, and simultaneous requests to send spike events must be arbitrated, avalanches with a large number of neurons will not be achievable within a short temporal window, introducing an undesirable size/speed trade-off.

An important factor shaping neuronal avalanches is the graph structure of the network. Not all graph structures lead to power-law distribution of neuronal avalanche size. In particular, a hierarchical organization of the network is sufficient to lead to power-law distribution of avalanche size as well as duration \cite{frla2013}. As described in Sec.\,\ref{sec:modularity_and_hierarchy}, a hierarchical network is comprised of multiple layers of nested modules. By contrast, a network described by a random graph which will have avalanche sizes that are bimodal, meaning there will be some probability $p$ that a single firing neuron will trigger a large avalanche spanning most of the connected network, and another probability $1-p$ that the avalanche will involve a negligible fraction of the nodes \cite{frla2013}.

A network with hierarchical modularity is conducive to generating neuronal avalanches of power-law-distributed sizes, and this is related to the fact that such networks can efficiently communicate signals across large regions of the network. Similarly, small-world networks are conducive to generating power-law-distributed neuronal avalanches, and small-world networks may or may not have a hierarchical architecture. However, the pattern of making primarily local connections but also important long range connections enables both neuronal avalanches of many sizes. Importantly, networks with initially random connections can self-organize into a small-world network through the adaptation of synaptic weights through spike-timing-dependent plasticity \cite{shki2006}. Even a network initiated with all-to-all connectivity can organize into a functional network with significantly less connectivity based on STDP, enabling a densely connected, random network\textemdash such as that of a newborn\textemdash to adapt into a network demonstrating power-law neuronal avalanches capable of efficient information transfer across all spatial scales of the network. It is important to note that a balance between inhibition and excitation is necessary to enable such avalanche behavior, but only the excitatory synapses need to be plastic to arrive at a scale-free, small-world network.

In addition to sculpting an initially random network into a small-world network, Ref.\,\onlinecite{shki2006} found the resulting degree distribution followed a power law:
\begin{equation}
\label{eq:power_law_degree_distribution}
p(k) \sim k^{-\beta}.
\end{equation}
The interpretation of Eq.\,\ref{eq:power_law_degree_distribution} is that if a node from the network is selected at random, the probability it will have degree between $k$ and $k+\Delta k$ is proportional to $^{-\beta}$. In the computational study of Ref.\,\onlinecite{shki2006}, this degree distribution interprets individual neurons as nodes, but experimental data taken with functional magnetic resonance imaging treating larger volumes of brain matter as nodes also found a power-law degree distribution \cite{egch2005}. In this study, the spatial resolution was limited by the imaging technique, and in general it is quite difficult to assess the degree distribution of actual brain networks in large animals. This information may result from the detailed study of the connectome at the microscopic scale \cite{http://www.humanconnectomeproject.org/}.
   
We have been discussing neuronal avalanches primarily in the spatial domain, and examples from the literature indicate that modular, hierarchical, networks with small-world graph structures lead to a power-law distribution of sizes of avalanches. These structural characteristics are not independent from temporal behavior, and we will find that the temporal distribution of neuronal avalanches leads to consideration of self-organized criticality. But first we describe simpler temporal phenomena, namely oscillations and synchrony.

The fractal distribution of neuronal avalanche sizes has ramifications for information processing and temporal responsivity. Many systems throughout nature obey power-laws taking the form of Eq.\,\ref{eq:neuronal_avalanche_power_law} \cite{ba2005_baks_book,dover_book_on_fractals}, and it often marks behavior at a critical point near a phase transition. Thus, observation of neuronal avalanches obeying power-law size distribution leads many to think the networks of the brain are balanced at a critical point with uninformative, synchronized order on one side, and chaotic disorder on the other \cite{be2007}. This balanced point is referred to as the critical state or criticality, and is related to the general theory of self-organized criticalty as described by Bak, Tang, and Wiesenfeld \cite{bata1987,bata1988,ba2005}. Beggs argued \cite{be2008} that criticality optimizes multiple aspectes on information processing in neural systems. Information transmission is optimized across spatial and temporal scales, as neurons are each able to participate in and initiate local as well as global activity. Information storage of short-term memories is increased, as criticality maximizes the number of stable dynamical attractor states supported by the network. It has also been argued that  By operating in the critical state, the dynamic range of the system is increased \cite{kico2006,shya2009}, the response time is reduced \cite{}.

In addition to showing scale invariance across space, dynamical systems operating at the critical point show a spatial invariance in time and frequency as well. This can be quantified by two metrics: the phase-lock

\begin{itemize}
\item get into a tad of history here, analogous to Turing/von Neumann
\end{itemize}

General intelligence: The ability to place a wide variety of information into a coherent context so that the behavior of the relevant parties can be understood and predicted.

Mention how deep learning neglects essentially all of the important aspects of neuronal information processing discussed here